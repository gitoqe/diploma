\anonsection{Требования к структуре, содержанию, оформлению}
    \href{https://document.vogu35.ru/education-deparment/shablony-makety-dokumentov/gosudarstvennaya-itogovaya-attestatsiya/bakalavriat-spetsialitet-magistratura/1244-vypusknaya-kvalifikatsionnaya-rabota-trebovaniya-k-strukture-soderzhaniyu-i-oformleniyu-1/file}{Прямая ссылка на документ}

    \subsection{ТРЕБОВАНИЯ К СТРУКТУРНЫМ ЭЛЕМЕНТАМ}
    Структура ВКР включает в себя обязательные элементы:
    \begin{itemize}
        \item Титульный лист ВКР (приложения 3-5);
        \item Задание на ВКР (приложения 6-9);
        \item Содержание (оглавление);
        \item Введение;
        \item Основная часть (наименование глав);
        \item Заключение;
        \item Список использованных источников (приложение 2);
        \item Приложения.
    \end{itemize}
    
    \subsection{Оглавление}
    4.3 Содержание (оглавление) ВКР должно включать введение, наименования всех
    глав в арабской нумерации, пунктов (без знаков параграфа), заключение, список
    использованных источников и наименования приложений с указанием номеров страниц.
    Заголовки в оглавлении должны точно повторять заголовки в тексте. Не допускается
    сокращать или давать заголовки в другой формулировке. 
    Последнее слово заголовка
    соединяют отточием с соответствующим ему номером страницы в правом столбце
    оглавления.

    \subsection{Введение}
    4.4 Введение должно содержать актуальность темы исследования, степень ее
    разработанности, цели и задачи, теоретическую и практическую значимость работы,
    методологию и методы исследования.
    
    \subsection{Основная часть}
    Основой ВКР является обоснование теоретической и практической базы, описание
    процесса и результатов исследования, выполненного лично автором. Не допускается
    переписывание учебников, норм, руководств, справочников, монографий и любых других
    источников. Краткое цитирование и обращение к первоисточникам должны выполняться с
    учётом норм профессиональной этики, требований законодательства РФ в сфере
    интеллектуальной собственности и ГОСТ Р 7.0.5-2008.

    Основную часть ВКР следует делить на главы, пункты и подпункты, учитывая, что
    каждый пункт должен содержать законченную информацию. \textbf{Структура, количество глав,
    содержание ВКР устанавливается по решению выпускающей кафедры в соответствии с
    направлением и профилем подготовки /специальности, ФГОС ВО и ФГОС СПО.}

    \subsection{Для технических специальностей}
    4.5.1 В основной части ВКР для технических направлений/специальностей должно
    быть отражено следующее.

    Глава 1 должна содержать:
    \begin{itemize}
        \item  выбор направления исследований или направления разработки;
        \item  описание, способы (методы) совершенствования технологического(их) процесса(ов);
        \item  аналитический обзор источников информации и нормативных данных.
    \end{itemize}

    Глава 2 должна содержать:
    \begin{itemize}
        \item  необходимые инженерные расчеты;
        \item  ТЭО (технико-экономическое обоснование) проектных или технологических
        решений;
        \item  результаты теоретических и/или экспериментальных исследований с описанием
        методики исследований и метрологических характеристик средств измерений, применяемых
        при исследованиях. Обобщение и оценку результатов исследований. Расчетно-пояснительная
        записка для технических направлений/специальностей должна содержать схемы, чертежи и
        графики.
    \end{itemize}

    Последующие главы включаются в ВКР по решению кафедры с целью раскрытия
    необходимой темы. Структура и содержание устанавливается по решению кафедры.

    Глава 3 может содержать организационно-экономическую часть (по вопросам,
    подлежащим разработке в организационно-экономической части, консультируют
    преподаватели кафедр, осуществляющие подготовку по управлению и экономике).
    Глава 4 может содержать анализ и обоснование предложений по безопасности и
    экологичности (по вопросам, подлежащим разработке в главе по безопасности 
    жизнедеятельности, консультируют преподаватели кафедры автомобилей и автомобильного
    хозяйства).
    По решению выпускающей кафедры последние две главы или одна из них могут быть
    переданы руководителю ВКР.

    \subsection{Заключение}
    4.6 Заключение
    Заключение ВКР должно содержать обобщения, выводы по результатам выполненной
    работы, данные о практической эффективности от внедрения, научную, социальную,
    лингвистическую, экономическую и т.д. ценность работы и предложения по
    совершенствованию предмета исследования.
    
    Выводы необходимо формулировать
    конкретно. Можно выделить каждый вывод в отдельный абзац. 
    
    Выводы должны
    соответствовать определенным во введении цели и задачам ВКР. В заключении можно
    указать на вопросы, нуждающиеся в дальнейшей разработке, и наметить перспективные
    направления исследования данной темы.

    \subsection{Список использованных источников}
    4.7 Список использованных источников
    Список использованных источников является необходимой структурной частью ВКР
    и содержит перечень источников, использованных при написании (независимо от вида
    документа и носителя – бумажного или электронного), помещается после основного текста,
    перед приложениями, имеет сквозную нумерацию страниц.
    
    «СПИСОК
    ИСПОЛЬЗОВАННЫХ ИСТОЧНИКОВ» располагается в середине строки, без точки в конце,
    прописными буквами.

    Библиографические записи в списке должны содержать основные сведения,
    достаточные для характеристики и идентификации изданий: автор, заглавие, место и год
    издания, количество страниц и т.д.; оформляются в соответствии с ГОСТ 7.1 – 2003.

    \textbf{Способ библиографической группировки литературы в списке избирается автором
    работы (по согласованию с руководителем работы)} в зависимости от ее целевого назначения,
    характера, вида, в частности:
    \begin{itemize}
        \item  в порядке следования ссылок, согласно очередности упоминания документов в тексте
        работы;
        \item  в алфавитном порядке из перечня фамилий авторов, заглавий изданий. Описания
        работ, опубликованных на иностранных языках, приводятся в конце списка отдельным
        алфавитным рядом;
        \item по видам источников – законодательные и нормативные документы, опубликованные
        и неопубликованные документы, исследования по теме, специальная литература
        (нормативно-технические, патентные документы и т.п.) или монографии, учебники, статьи из
        журналов и сборников, специальная литература и т.п.;
        \item в порядке хронологии (прямой или обратной) опубликования документов.
    \end{itemize}

    \subsection{Приложения}
    4.8 Приложения
    В приложения рекомендуется включать связанные с выполненной ВКР материалы,
    которые не могут быть внесены в основную часть:
    \begin{itemize}
        \item  промежуточные математические доказательства, формулы и расчеты, таблицы вспомогательных цифровых данных;
        \item  протоколы испытаний;
        \item  описание аппаратуры и приборов, применяемых при проведении экспериментов,
        измерений и испытаний, заключение метрологической экспертизы;
        \item  инструкции, методики, разработанные в процессе выполнения, \textbf{авторские разработки
        (программа}, конспекты мероприятий, брошюры, рекомендации и т.п.);
        \item  иллюстрации вспомогательного характера;
        \item  инструкции и методики, описания алгоритмов и программ, задач, решаемых на
        компьютерах, разработанных в ходе выполнения
        \item  первичная информация (для экономических направлений/специальностей);
        \item  промежуточные формулы, расчеты, бухгалтерские балансы (для экономических
        направлений/ специальностей);
        \item  бланки методик, опросники, тесты, анкеты (в том числе авторские), вопросы для
        интервью и т.д.), результаты деятельности испытуемых (рисунки, анкеты и пр.).
    \end{itemize}

    На все приложения должны быть даны ссылки. Приложения располагают в порядке
    появления ссылок в тексте.

    Каждое приложение следует начинать с новой страницы. Слово «ПРИЛОЖЕНИЕ» и
    его порядковый номер (арабскими цифрами или заглавными буквами русского алфавита)
    располагаются в середине строки, без точки в конце, прописными буквами. Строкой ниже в
    скобках указывают «обязательное», «рекомендуемое» или «справочное».
    Приложение должно иметь заголовок, который записывают посередине симметрично
    относительно текста с прописной буквы отдельной строкой. Приложения, как правило, 
    оформляют на листах формата А4. Допускается выполнять на листах форматов АЗ, А4хЗ,
    А4х4, А2 и А1 (ГОСТ 2.301-68 ЕСКД. Форматы).

    Текст каждого приложения, при необходимости, может быть разделен на разделы,
    подразделы, пункты, подпункты, которые нумеруют в пределах каждого приложения. Перед
    номером ставится обозначение этого приложения. Приложения должны иметь общую с
    остальной частью документа сквозную нумерацию страниц.

    Приложение или несколько приложений могут быть оформлены в виде отдельного
    тома, при этом на титульном листе под номером тома следует писать слово «ПРИЛОЖЕНИЯ».
    При необходимости такое приложение может иметь раздел «СОДЕРЖАНИЕ». Все
    имеющиеся приложения должны быть перечислены в разделе «СОДЕРЖАНИЕ» с указанием
    их обозначений и заголовков.

    \subsection{Оформление текста}
    5 ТРЕБОВАНИЯ К ОФОРМЛЕНИЮ ТЕКСТА

    5.1 Общие требования к оформлению
    Компьютерная верстка текста выполняется в соответствии с таблицей 1. Страницы
    текста ВКР и включенные в текст иллюстрации и таблицы должны соответствовать формату
    А4 по ГОСТ 9327-60. Допускается применение формата A3 при наличии большого
    количества таблиц и иллюстраций данного формата. ВКР должна быть выполнена с
    использованием компьютера и принтера на одной стороне листа белой бумаги (210х297 мм,
    WHITE). Вне зависимости от способа выполнения ВКР качество напечатанного текста и
    оформления иллюстраций, таблиц, распечаток должно удовлетворять требованию их четкого
    воспроизведения. Необходимо переносить текст на новую строку целым словом, соблюдать
    равномерную плотность, контрастность и четкость изображения. Линии, буквы, цифры и
    знаки должны быть четкие, не расплывшиеся.

    Опечатки, описки и графические неточности, обнаруженные в процессе подготовки
    записки, допускается исправлять подчисткой или закрашиванием белой краской с
    последующим нанесением на том же месте исправленного текста (графика) машинописным
    способом или черными чернилами, пастой или тушью – рукописным способом.
    Повреждения листов, помарки и следы не полностью удаленного прежнего текста (графика)
    не допускаются.

    При выполнении ВКР с конструкторскими разработками обозначение изделий и
    документов устанавливается по ГОСТ 2.201-80. В соответствии с классификатором ЕСКД
    обозначение должно быть присвоено каждому изделию. Обозначение изделия является
    одновременно обозначением его основного документа (чертежа, детали или спецификации). 
    Обозначение должно быть указано на каждом листе конструкторского документа,
    выполненного на нескольких листах.

    \subsubsection{Параметры текста и документа}
    Заголовок главы
    \begin{itemize}
        \item Новая страница Да
        \item Шрифт Times New Romаn, пт 14        (ПРОПИСНЫМ)
        \item Интервал до, пт 0
        \item Интервал после, пт 14
        \item Выравнивание По центру
        \item Межстрочное расстояние 1,5 инт.
        
    \end{itemize}

    
    
    
    
    
    
    



\clearpage
