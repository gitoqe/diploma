\section{Анализ предметной области разработки Web-ресурса}
    \subsection{Обзор предметной области}
    \subsection{Обзор современных методов и технологий разработки web-ресурсов}
    \subsection{Обзор существующих решений для разработки web-ресурсов}
    \subsection{Проверка соответствия существующей системы современным требованиям к web-ресурсам}
    \subsection{Постановки задачи}
    \subsection{Постановка требований к разрабатываемому web-ресурсу}

\section{Проектирование web-ресурса}
    \subsection{Инструменты разработки web-ресурсов}
        \subsubsection{Системы управления контентом}
        \subsubsection{Инструменты разработки для front-end}
        \subsubsection{Инструменты разработки для back-end}
        \subsubsection{Выбор инструментов для разработки web-ресурса}

    \subsection{Построение структурной схемы существующей системы}
        \subsubsection{Использование редактора диаграмм Dia}
        \subsubsection{Компоненты существующей системы}
        \subsubsection{Агрегирование переходов между страницами существующей системы}
        \subsubsection{Анализ существующей системы сайта}

    \subsection{Построение структурной схемы web-ресурса}
        \subsubsection{Взаимодействие пользователя с web-ресурсом}
        \subsubsection{Компоненты web-ресурса}
        \subsubsection{Хранение данных для работы web-ресурса}

\section{Разработка web-ресурса}
    \subsection{Разработка общего шаблона}
    \subsection{Разработка формы обратной связи}
    \subsection{Разработка системы обработки запросов}
    \subsection{Разработка и подключение базы данных}
    \subsection{Разработка системы администрирования}

\section{Разработка методических материалов для администраторов и пользователей web-ресурса}
    \subsection{Руководства администратора}
    \subsection{Руководства пользователя}

\clearpage
