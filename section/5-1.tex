\section{Анализ предметной области разработки Web-ресурса}
    \subsection{Обзор предметной области}
    \subsection{Современные технологий и подходы в разработке web-ресурсов}
        \subsubsection{Системы управления контентом}
            https://www.nic.ru/info/blog/cms/
            \begin{itemize}
                \item wordpress
                \item 1С-Битрикс
                \item Joomla
                \item Drupal
            \end{itemize}

        \subsubsection{Технологии разработки для front-end}
        \subsubsection{Технологии разработки для back-end}
        \subsubsection{Операционные системы}
            \begin{itemize}
                \item windows
                \item mac os
                \item linux
            \end{itemize}
            
        \subsubsection{Среды разработки и редакторы кода}
            \begin{itemize}
                \item VS Code
                \item sublime
                \item atom
                \item VS
            \end{itemize}

    \subsection{Современные требования к web-ресурсам}
        \subsubsection{Определение требований к web-ресурсам}
        \subsubsection{Проверка соответствия существующей системы заявленным требованиям}
    
    \subsection{Постановка требований к разрабатываемому web-ресурсу}
    \subsection{Постановки задачи}
    
\section{Проектирование web-ресурса}
    \subsection{Выбор инструментов для разработки web-ресурса}
        \subsubsection{Выбор ОС для работы}

        \subsubsection{Выбор среды разработки}
            
        \subsubsection{Выбор редактора диаграмм}
            \begin{itemize}
                \item Dia   http://dia-installer.de/
                \item PlantUML  https://plantuml.com/ru/
            \end{itemize}
        
    \subsection{Построение структурной схемы существующей системы}
        \subsubsection{Компоненты существующей системы}
        \subsubsection{Агрегирование переходов между страницами существующей системы}
        \subsubsection{Анализ существующей системы сайта}

    \subsection{Построение структурной схемы web-ресурса}
        \subsubsection{Взаимодействие пользователя с web-ресурсом}
        \subsubsection{Компоненты web-ресурса}
        \subsubsection{Хранение данных для работы web-ресурса}

\section{Разработка web-ресурса}
    \subsection{Разработка общего шаблона}
    \subsection{Разработка формы обратной связи}
    \subsection{Разработка системы обработки запросов}
    \subsection{Разработка и подключение базы данных}
    \subsection{Разработка системы администрирования}

\section{Разработка методических материалов для администраторов и пользователей web-ресурса}
    \subsection{Руководства администратора}
    \subsection{Руководства пользователя}

\clearpage
