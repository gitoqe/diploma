\anonsection{Введение}

В последние десятилетия всё больше набирает темп всеобщая цифровизация, приводящая к возникновению у общества новых запросов к окружающей инфраструктуре.
Всё большая доступность и распространение персональных компьютеров, смартфонов и прочих устройств влечет за собой увеличение количества пользователей.
Повышение количества клиентов оказывает влияние на сферу обслуживания, представленную в виде самых разных профилей занятости: разработчиков, дизайнеров, архитекторов информационных решений, техников, инженеров и многих других специалистов.
Возникающий при этом спрос на получение подтверждения навыков в форме сертификата или диплома пытаются удовлетворить как частные, так и государственные организации.

Примером воздействия со стороны государства на повышение уровня знаний у технических специалистов можно назвать внедрение программ обучения информационным технологиям в образовательных учреждениях.
Сюда входят как занятия по изучению информатики в средней школе, так и различные специализированные предметы в высших и средних учебных заведениях.

Выбранная для выпускной квалификационной работы организация -- частное учреждение дополнительного профессионального образования <<Учебный центр <<Мезон>> \cite{meson-uc} (далее - Учреждение, Организация, Компания, Учебный центр) на базе предприятия ЗАО <<Мезон>> \cite{meson}.
Данная Организация как раз специализируется на предоставлении услуг дополнительного профессиональго образования.
Спектр предоставляемых программ захватывает как дошкольников и школьноков, так и взрослых.

Одной из ключевых задач для Учебного центра является взаимодействовие с клиентами.
Сюда входят различные стадии -- привлечение, консультирование, информирование, оформление для обучения, обучение и другие виды деятельности.
Одна из самых важных среди перечисленных -- интерактивное вовлечение и информирование, которые могут осуществляться как персоналом Учреждения, так и с помощью различных средств коммуникации.
Наиболее простым и распространённым вариантом интерактивного взаимодействия с клиентами является использование информационных ресурсов -- сайтов, приложений, email-рассылок.
Данные способы подразумевают технические возможности для их осуществления.

Учреждение обладает работающим сайтом, представляющим из себя статический многостраничный набор документов на языке гипертекстовой разметки HTML \cite{wiki-html} с использованием технологии каскадных таблиц стилей CSS \cite{wiki-css}, а также языка программирования JavaScript \cite{wiki-js}.
Выполняемая им задача включает в себя как информирование посетителей о предлагаемых Учебным центром услугах, внутренней организации и различной справочной информации.
Актуальность наполнения данного сайта и его внешнего вида являются критичными для успешного осуществления деятельности Учреждения.

Целью выпускной квалификационной работы является разработка веб-ресурса для <<Учебный центр <<Мезон>> г. Вологды.

Работа состоит из 5 разделов.

В первом разделе проводится аналитический обзор информационного обеспечения учебных центров.
Рассматриваются различные варианты проектирования веб-ресурсов по функциональности, дизайну, типу контента.
Проводится общий обзор средств разработки сайтов и систем управления контентом, затрагиваются проблемы их модернизации, а также применяемые паттерны и фреймворки.
Помимо этого, разбираются примеры имеющихся на рынке сайтов учебных центров с определением задействованных технологий.

Во втором разделе рассматривается предметная область разработки веб-ресурсов.
Здесь производится выбор средств для практического исполнения работы, определения качества внесенных доработок.
Проводится анализ исходного состояния сайта, на основании которого осуществляется постановка задачи и требований к разрабатываемому приложению.

В третьем разделе производится проектирование веб-ресурса.
Строятся диаграммы функциональной структуры ресурса, функциональной модели ресурса, вариантов использования, деятельности.
Проводится анализ структуры исходного веб-ресурса и на его основе формируется структура проектируемого сайта, а также выделяются основные сущности.
Стоитися полная атрибутивная модель данных, задействованная далее в создании и наполнении базы данных.
Проектируются и реализуются программные интерфейсы, настраивается их взаимодействие с клиентской частью.

В четвертом разделе представлены этапы разработки интерфейса системы.
Рассмотрены используемые технологии, формирование шаблона основной страницы.
Также проведена проверка внесенных изменений.

В пятом разделе содержится инструкция по работе с разработанным веб-ресурсом с позиции пользователя и администратора.

\clearpage
