\anonsection{Введение}

Цель преддипломной практики  – развитие у студентов целеустремленности, организованности, овладение методами для аналитической работы и научных исследований; формирование у студентов компетенций и навыков ведения самостоятельной работы. 

Задачи преддипломной практики: 

– закрепление полученных теоретических знаний по всему комплексу профильных и смежных с ними дисциплин; 

– закрепление практических навыков, приобретенных при выполнении лабораторных работ и практикумов; 

– приобретение практических навыков и компетенций в области информационных систем и технологий; 

– сбор материала для выполнения выпускной квалификационной работы.

\clearpage
\section{Выбор средств разработки}

После проведенного анализа в Главе 1 можно выделить основные требования к инструментам разработки и технологиям, задействованным в этом процессе.
Проходя по разобранным категориям дифференциации сайтов, можно выделить набор характеристик, которым должно соответствовать итоговый результат.

Функционально веб-ресурс должен будет соответствовать сайту-визитке, поскольку основной задачей будет являться привлечение новых и информирование имеющихся клиентов компании.

По отзывчивости дизайна лучшим вариантом будет выбор в сторону комбинации адаптивных макетов и отзывчивого способа работы шаблонов страниц сайта.

Относительно способа формирования контента предпочтение отдается динамическому.
В таком случае можно будет заранее предусмотреть возможность расширения функционала под новые задачи без необходимости перестроения ресурса с нуля.

По типу контента веб-ресурс будет привязан к корпоративной тематике.
Это обусловлено изначальной задачей, решаемой сайтом, которая не будет меняться в ходе выполнения данной работы.

Также на основе рассмотренных в Главе 1 средств разработки веб-ресурсов, необходимо выделить технологии, подходящие под описанные выше критерии.

Безусловно, современная веб-разработка завязана на использование как минимум языка разметки гипертекста HTML и каскадных таблиц стилей CSS.
Данная работа не станет исключением и также будет задействовать их как в прямом виде, так и опосредованно через использование прочего программного обеспечения.

Следующим ключевым решением будет выбор между использованием системы управления контентом (а также присущей ей экосистемы) и самостоятельным проектированием.
По работе Киямова Р.В., Хмелева Е.А. и Юнусова И.Ф. \cite{kiyamov-cms}, а также работе Иванищевой А.А., Комилова Х.И. и М.Д. Гехаева \cite{ivanisheva-cms} можно выделить преимущества и недостатки представленных на рынке CMS.
Исходя из приведенных доводов оптимальным вариантом может явиться WordPress.

Делая выбор между CMS и самостоятельно разработкой компонентов веб-ресурса, можно сделать предположение о большей полезности в выборе последнего.
Аргументировать такую позицию можно тем, что такой подход потребует задействование большего объема технологий и их возможностей для реализации готового веб-ресурса.
И, как следствие, количество шагов в выполнении данной работы увеличится, что приведет к большему количеству материалов, которые можно продемонстрировать, а также задействованию и изучению большего спектра технических средств.

При разработке веб-ресурса без использования систем управления контентом важным пунктом является выбор используемых технологий.
В Главе 1 приводились наборы технических средств -- стеков, рассмотренные в работе за авторством М.А. Давыдовского \cite{davidovsky-vibor}.
Среди указанных в статье, наиболее выделяются MEAN и MERN.
Первый предполагает использование Angular как средства разработки клиентской части, второй же задействует на этой роли React.
Остальные пункты остаются неизменными -- использование MongoDB в роли СУБД, Express и Node.js для организации серверной части приложения.

Ещё одним важным этапом является выбор способа взаимодействия между клиентской и серверной частью.
За данный пункт отвечают специальные интерфейсы прикладных программ - API (application programming interface), выполняющие роль посредника между частями приложения и определяющие сам способ передачи информации, её вид.

Наиболее популярным является архитектурный подход REST, основывающийся на протоколе HTTP для транспортировки данных.
Например, в приведённом выше стеке MERN/MEAN подход REST реализуется на базе технологии Express.js.
Альтернативным к REST подходом, является GraphQL, рассматривающий взаимодейтсвие и работу с данными через графы.

Подводя промежуточный итог можно выделить особенности проектируемого приложения.
Используется стек MEAN с возможной заменой MongoDB на другую СУБД SQLite.
Из данного выбора вытекают следующие технологии -- Angular и применяемый в нем язык программирования TypeScript, Node.js и Express.js для реализации интерфейсов взаимодействия REST и серверной логики приложения.
Для работы с шаблонами страниц будет использоваться HTML и CSS.

\clearpage
\section{Функциональная структура}

Исходя из поставленной задачи, а также текущего наполнения веб-ресурса компании, можно выделить основные разделы, которые необходимо отобразить.
Эти разделы: «Об организации», «Расписание», «Курсы», «Мероприятия», «Аренда», «Контакты», «Оплата», «Работы учеников».

Раздел «Об организации» содержит информацию о компании, руководстве, а также обширный набор вспомогательной документации, необходимой для размещения на сайте.

Раздел «Расписание» содержит непосредственное расписание занятий на учебный год и позволяет оперативно сориентировать клиентов касательно неучебных дней в организации.

Раздел «Курсы» представляет клиенту набор предоставляемых курсов для обучения с возможностью подробного рассмотрения особенностей каждого из них в отдельности.

Раздел «Мероприятия» содержит информацию о проводимых учебным центром мероприятий.

Раздел «Аренда» отображает важную информацию касательно аредны помещений и оборудования организации.

Раздел «Контакты» представляет различные способы связи с сотрудниками учреждения.

Раздел «Оплата» содержит важную информацию о способах оплаты услуг учебного центра.

Раздел «Работы учеников» представляет клиентам возможность ознакомиться с выставленными работами студентов разных направлений.

На основании приведенных разделов можно сформировать структурную схему веб-ресурса, отображающую получившуюся иерархию страниц.
Результат формирования такой структуры представлен на Рисунке \ref{struct-scheme}.

\addimghere{images/struct-scheme.png}{1}{Структурная схема проектируемого веб-ресурса}{struct-scheme}

\clearpage
\anonsection{Заключение}

В ходе выполнения данной работы был произведен поиск информации касательно актуальных технических средств и подходов, применяемых для выполнения задач, схожих с полученной темой ВКР.

Из полученной информации были выделены наиболее подходящие решения, пригодные для использования в проектировании веб-ресурса.
Также были выделены ключевые качества приложения, необходимые для реализации.

Также был проведен анализ информации исходной версии веб-ресурса, выделены ключевые разделы, необходимые для сохранения и реализации в новой версии.
На основании получившихся разделов была составлена структурная схема веб-сайта.
\clearpage
