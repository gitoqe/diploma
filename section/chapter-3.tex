\section{Разработка Web-ресурса для учебного центра «Мезон»}

На основании технологий и требований, выделенных в главах 1 и 2 можно составить краткий план выполнения практической части работы.
Необходимо обеспечить процесс разработки документацей и планами вносимых изменений.

В первую очередь необходимо подготовить схему планируемой функциональной структуры ресурса.
На основании данной схемы будут строится взаимосвязи между программными решениями, используемыми для разработки и дальнейшей работы сайта.

Следующий этап -- формирование структурной схемы веб-ресурса.
В её задачи будет входить упрощенное описание компонентов-страниц сайта, расположение и отношение между ними.
Этот шаг позволит с пониманием общей картины подходить к разработке отдельных частей ресурса.

Следующим важным решением будет выделение ключевых сущностей для хранения в базе данных, их взаимоотношения и зависимостей.
На основе этих данных будут строится отдельные компоненты страниц веб-ресурса с использованием шаблонов.


\subsection{Функциональная структура ресурса}

Исходя из определенных для использования в главах 1 и 2 программных средств, можно составить функциональную структуру ресурса.
Она призвана отобразить основные элементы системы, обеспечивающие работу элементов сайта, а также взаимосвязи между ними.

Выбранными элементами стека MEAN с внесёнными правками являются -- SQLite в роли СУБД, Express и Node.js для организации серверной части приложения, Angular для работы с клиентской частью.
Также стоит упомянуть CSS-библиотеку Bulma, которая также будет задействована в данной работе.
На основе приведенных программных средств составлена схема функциональных элементов на рисунке \ref{func-struct-scheme}.

\addimghere{images/функциональная-структура.png}{0.5}{Функциональная структура веб-ресурса}{func-struct-scheme}


\subsection{Анализ структуры исходного ресурса}\label{Анализ структуры исходного ресурса}

Главная страница исходного сайта содержит 4 основных раздела -- «Главная», «Сведения об организации», «Обучение», «Аренда».
Также здесь расположены различные вспомогательные ссылки, реализующие переходы на некоторые ключевые разделы ресурса.
Схема основных переходов внутри главной страницы отображена на рисунке \ref{meson-old-main}.

\addimghere{images/главная.png}{1}{Схема строения переходов исходного веб-ресурса}{meson-old-main}

В схеме на рисунке \ref{meson-old-main} отметкой <<(неактуально)>> помечены переходы на неактуальные страницы, которые по разным причиным либо давно не обновлялись, либо отсутствуют вовсе.
При этом основными функциональными элементами на главной странице являются ссылки «Расписание», «Стоимость» и «Брошюра».
Взаимодействие с данными, представленными на целевых страницах этих ссылок содержит самую полезную информацию, которая может потребоваться для клиентов учреждения.
Помимо этого, контент этих элементов частично дублируется и частично взаимно противоречит.

Раздел «Сведения об организации» содержит наибольшее количество вложенных страниц - «Основные сведения», «Документы», «Образование», «Образовательные стандарты», «Руководство и педагогический состав», «Структура и органы управления», «Материально-техническое обеспечение и оснащенность образовательного процесса», «Стипендии и иные виды материальной поддержки», «Платные образовательные услуги», «Финансово-хозяйственная деятельность», «Вакантные места для приема (перевода)».
Большая часть содержимого этих подстраниц либо дублируется в других частях ресурса, либо является вовсе неактуальной.

Раздел «Обучение» состоит из подразделов «Компьютерная школа», «Лаборатории инновационного творчества», «Школа личностного роста», «ИТ-курсы для взрослых».
Содержимое этих частей ресурса по большей части устарело и является неактуальным.

Раздел «Аренда» содержит информацию об аренде помещений учебного центра.
Часть информации данной страницы является неактуальной.
Подразделы здесь отсутствуют.

Помимо переходов на обозначенные разделы со страницы «Главная», также есть необозначенные там три -- «Независимая оценка качества», «Противодействие коррупции», «Инклюзивное образование».
Состав этих элементов также является частично неактуальным.

% // TODO sitemapmaker - карта переходов исходного сайта

Таким образом можно сделать вывод о заброшенности большинства разделов сайта.
Часть этих страниц можно актуализировать и восстановить в новой версии, а часть объединить с другими.


\subsection{Структура проектируемого ресурса}

На основе проведенного в подпункте \ref{Анализ структуры исходного ресурса} анализа можно выделить ключевые разделы проектируемого веб-ресурса с учетом объединения и перестановки исходных.
В результате получатся следующие пункты: «Поступление», «Курсы», «Расписание», «Работы учеников», «Об организации», «Оплата и аренда».

Раздел «Поступление» рассказывает пользователю о правилах записи на курсы учебного центра.

Раздел «Курсы» представляет клиенту набор имеющихся для проведения курсов обучения с возможностью подробного рассмотрения особенностей каждого из них в отдельности.

Раздел «Расписание» включает в себя расписание занятий групп на учебный год, позволяет оперативно сориентировать клиентов касательно неучебных дней в организации.

Раздел «Работы учеников» представляет клиентам возможность ознакомиться с выставленными работами студентов разных направлений.

Раздел «Об организации» содержит информацию о компании, руководстве, а также обширный набор вспомогательной документации, полезной для ознакомления.

Раздел «Оплата и аренда» содержит информацию о способах оплаты услуг учебного центра и предоставляемых в аредну помещениях.

Также стоит отметить точку входа на сайт -- раздел «Главная», встречающий пользователя.
Здесь описываются функциональные модули-страницы сайта и их содержимое, а также осовные данные об учебном центре.

На основании приведенных разделов можно сформировать структурную схему веб-ресурса, отображающую получившуюся иерархию страниц.
Результат формирования такой структуры представлен на Рисунке \ref{struct-scheme}.

\addimghere{images/структура-ресурса.png}{1}{Структурная схема проектируемого веб-ресурса}{struct-scheme}

Упрощение базовой структуры облегчит взаимодействие с информацией на сайте.
Также будет проще пользоваться ресурсом на мобильных устройствах.


\subsection{Выделение основных сущностей и построение полной атрибутивной модели данных}

В работе любой системы задействуются данные.
Их можно классифицировать по различным признакам, на основе которых можно выделить общий набор характеристик и хранить их в единообразном хранилище -- базе данных.

Для разрабатываемого ресурса будут задействованы возможности СУБД SQLite.
Важно учитывать особенность данной СУБД -- наличие ограниченного списка базовых типов данных.
В него входят NULL для пустых значений, INTEGER для целочисленных значений разного размера с учетом знака, REAL для храенения вещественных значений, TEXT для хранения строк, BLOB для хранения двоичных объектов.
Среди имеющегося на исходном сайте контента можно выделить несколько сущностей, доступных для переноса в формат базы данных.

Первая таблица -- <<Courses>>.
Предназначен для хранения информации о конкретном курсе.
В общие характеристики-поля можно выделить имя, описание, стоимость, тип оплаты -- за месяц или занятие, прямое продолжение курса, уровень курса -- дошкольный или школьный, класс или возраст в зависимости от уровня курса, длительность занятия, список тем.
Схема данной таблицы представлена на рисунке \ref{uml-courses}.
\addimghere{images/uml-courses.png}{0.35}{Схема таблицы <<Courses>>}{uml-courses}

Вторая таблица -- <<Levels>>
Это вспомогательная таблица содержит поле для выбора в таблице <<Courses>> -- наименование уровня.
Схема данной таблицы представлена на рисунке \ref{uml-levels}.
\addimghere{images/uml-levels.png}{0.35}{Схема таблицы <<Levels>>}{uml-levels}

Третья таблица -- <<Students\_work>>.
Эта таблица будет хранить данные для представления на странице итоговых работ студентов с разных курсов.
Для её работы потребуются следующие поля -- соответствующий курс, опциональное имя студента, год выполнения работы, ссылка на работу.
Схема данной таблицы представлена на рисунке \ref{uml-students_work}.
\addimghere{images/uml-students_work.png}{0.35}{Схема таблицы <<Students\_work>>}{uml-students_work}

Четвертая таблица -- <<Employees>>.
Сюда можно отнести данны о преподавательском составе.
Выделяемые поля -- фамилия, имя, отчество, должность, образование, год начала работы, педагогический стаж.
Схема данной таблицы представлена на рисунке \ref{uml-employees}.
\addimghere{images/uml-employees.png}{0.35}{Схема таблицы <<Employees>>}{uml-employees}

На данном этапе возникает необходимость в перечислении нескольких значений в составе одной ячейки.
Можно выполнить такую задачу через применение связующей таблицы, где будут перечислены соотношения между записями отдельных таблиц.

Пятая таблица -- <<Teaches>>.
Данная таблица -- связующая между таблицами <<Employees>> и <<Courses>>.
В её задачи входит перечисление соответствий сотрудников преподаваемым ими курсам.
Схема данной таблицы представлена на рисунке \ref{uml-teaches}.
\addimghere{images/uml-teaches.png}{0.35}{Схема таблицы <<Teaches>>}{uml-teaches}

Итоговый вид базы данных с таблицами, используемыми в формировании наполнения веб-ресурса представлен на рисунке \ref{uml-final}.
\addimghere{images/uml-final.png}{0.6}{Структура таблиц базы данных, участвующих в формировании содержимого веб-ресурса}{uml-final}


%\subsection{Выделение основных сущностей администрирования веб-ресурса}

Помимо использования контента, расположенного в базе данных, веб-ресурсу также требуется возможность редактирования исходной информации.
Для таких задач может подойти и обычные программные средства СУБД, так и решения встроенные непосредственно в сайт.

Осуществление данной задачи потребует контроля доступа к функционалу радактирования.
Вместе с этим возникает необходимость хранения данных для его активации.
Для этой задачи можно воспользоваться отдельной таблицей для записи пар логинов и паролей пользователей, обладающих правами редактора содержимого веб-ресурса.

В реализации подходящей таблицы должно быть несколько обязательных полей.
В них входят -- логин пользователя, пароль пользователя, уникальный идентификатор, а также вспомогательный уникальный токен для использования в процессе авторизации.
Схема данной таблицы представлена на рисунке \ref{uml-users}.
\addimghere{images/uml-users.png}{0.35}{Структура таблицы <<Users>>}{uml-users}

На основании представленных схем таблиц базы данных, а также описания полей в них содержащихся, можно сформировать общую картину приложения с точки зрения хранимой информации.
При этом важно учесть сущности, атрибуты и связи.
Указанные в формируемой схеме типы будут соответствовать применяемым в SQLite -- INTEGER, BLOB, TEXT и FLOAT соответственно.
Внешние ключи помечены как сокращение от <<foreign key>> -- <<FK>>, а первичный ключ от <<primary key>> -- <<PK>>.
За данную задачу отвечает полная атрибутивная модель данных, изображенная на рисунке \ref{uml-attr-model}.
\addimghere{images/uml-attr-model.png}{0.7}{Полная атрибутивная модель данных проектируемого веб-ресурса}{uml-attr-model}


\subsection{Разработка функциональной модели}

% // TODO Контекстная диаграмма IDEF0, описание

Контекстная диаграмма IDEF0 на рисунке \ref{idef0-drawio}.
\addimghere{images/idef0.drawio.png}{0.7}{Контекстная диаграмма  IDEF0}{idef0-drawio}

Описание стрелок контекстной диаграммы представлено в таблице \ref{table-context-diagram}.

\begin{longtable}[h]{|p{6.6cm}|p{6.6cm}|c|}
    \caption{Описание стрелок контекстной диаграммы}\label{table-context-diagram}\\
    \hline
    \centering Название стрелки&
    \centering Описание&
    Тип\\
    \hline
    \centering 1&
    \centering 2&
    3\\
    \hline\endfirsthead
    \multicolumn{1}{l}{Продолжение таблицы \ref{table-context-diagram}}\\
    \hline
    \centering Название стрелки&
    \centering Описание&
    Тип\\
    \hline
    \centering 1&
    \centering 2&
    3\\
    \endhead

    Пользовательское соглашение&
    Описание &
    Control\\

    \hline
    Администратор сайта&
    Описание&
    Mechanism\\

    \hline
    Информация о курсах&
    Описание&
    Input\\
    \hline
    Информация о поступлении&
    Описание &
    Input\\
    \hline
    Информация о расписании&
    Описание &
    Input\\
    \hline
    Информация о работах учеников&
    Описание&
    Input\\
    \hline
    Информация об организации&
    Описание&
    Input\\
    \hline
    Информация об оплате и аренде&
    Описание&
    Input\\

    \hline
    Компонент с информацией о курсах&
    Описание&
    Output\\
    \hline
    Компонент с информацией о поступлении&
    Описание&
    Output\\
    \hline
    Компонент с информацией о расписании&
    Описание&
    Output\\
    \hline
    Компонент с информацией о работах учеников&
    Описание&
    Output\\
    \hline
    Компонент с информацией об организации&
    Описание&
    Output\\
    \hline
    Компонент с информацией об организации&
    Описание&
    Output\\
    \hline
\end{longtable}

Декомпозиция первого уровня диаграммы IDEF0
% // TODO Декомпозиция первого уровня диаграммы IDEF0

Декомпозиция второго уровня диаграммы IDEF0
% // TODO Декомпозиция второго уровня диаграммы IDEF0

\subsection{Разработка диаграммы вариантов использования}
% // TODO Диаграмма вариантов использования 
Диаграмма вариантов использования 

\subsection{Разработка диаграммы деятельности}
% // TODO Диаграмма деятельности разрабатываемого портала
Диаграмма деятельности пользователей разрабатываемого портала

\clearpage
