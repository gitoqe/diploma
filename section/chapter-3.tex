\section{Разработка Web-ресурса для учебного центра «Мезон»}

% // TODO План и анализ текущего состояния ресурса

% // TODO Выделение основных страниц и сущностей веб-ресурса


\subsection{Функциональная структура ресурса}

Исходя из поставленной задачи, а также текущего наполнения веб-ресурса компании, можно выделить основные разделы, которые необходимо отобразить.
Эти разделы: «Об организации», «Расписание», «Курсы», «Мероприятия», «Аренда», «Контакты», «Оплата», «Работы учеников».

Раздел «Об организации» содержит информацию о компании, руководстве, а также обширный набор вспомогательной документации, необходимой для размещения на сайте.

Раздел «Расписание» содержит непосредственное расписание занятий на учебный год и позволяет оперативно сориентировать клиентов касательно неучебных дней в организации.

Раздел «Курсы» представляет клиенту набор предоставляемых курсов для обучения с возможностью подробного рассмотрения особенностей каждого из них в отдельности.

Раздел «Мероприятия» содержит информацию о проводимых учебным центром мероприятий.

Раздел «Аренда» отображает важную информацию касательно аредны помещений и оборудования организации.

Раздел «Контакты» представляет различные способы связи с сотрудниками учреждения.

Раздел «Оплата» содержит важную информацию о способах оплаты услуг учебного центра.

Раздел «Работы учеников» представляет клиентам возможность ознакомиться с выставленными работами студентов разных направлений.

На основании приведенных разделов можно сформировать структурную схему веб-ресурса, отображающую получившуюся иерархию страниц.
Результат формирования такой структуры представлен на Рисунке \ref{struct-scheme}.

\addimghere{images/struct-scheme.png}{1}{Структурная схема проектируемого веб-ресурса}{struct-scheme}

\clearpage
