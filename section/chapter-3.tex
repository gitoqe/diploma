\section{Разработка Web-ресурса для учебного центра «Мезон»}

% // TODO описать шаги выполнения разработки

\subsection{Функциональная структура ресурса}

% // TODO исходная схема

% // TODO анализ исходной схемы

% // TODO Выделение основных страниц и сущностей веб-ресурса

Исходя из поставленной задачи, а также текущего наполнения веб-ресурса компании, можно выделить основные разделы, которые необходимо отобразить.
Эти разделы: «Поступление», «Курсы», «Расписание», «Работы учеников», «Об организации», «Оплата и аренда».

Раздел «Поступление» рассказывает пользователю о правилах записи на курсы учебного центра

Раздел «Курсы» представляет клиенту набор имеющихся для проведения курсов обучения с возможностью подробного рассмотрения особенностей каждого из них в отдельности.

Раздел «Расписание» включает в себя расписание занятий групп на учебный год, позволяет оперативно сориентировать клиентов касательно неучебных дней в организации.

Раздел «Работы учеников» представляет клиентам возможность ознакомиться с выставленными работами студентов разных направлений.

Раздел «Об организации» содержит информацию о компании, руководстве, а также обширный набор вспомогательной документации, полезной для ознакомления.

Раздел «Оплата и аренда» содержит информацию о способах оплаты услуг учебного центра и предоставляемых в аредну помещениях.

Также стоит отметить точку входа на сайт - раздел «Главная», встречающий пользователя.
Здесь описываются функциональные модули-страницы сайта и их содержимое, а также некоторые факты об учебном центре.

На основании приведенных разделов можно сформировать структурную схему веб-ресурса, отображающую получившуюся иерархию страниц.
Результат формирования такой структуры представлен на Рисунке \ref{struct-scheme}.

% FIXME \addimghere{images/struct-scheme.png}{1}{Структурная схема проектируемого веб-ресурса}{struct-scheme}


\subsection{Проектирование }

% // TODO Выделение основных сущностей веб-ресурса

% // TODO Схема сущностей - выделение параметров, требуемых для работы в приложении

\clearpage
