\section{Разработка Web-ресурса для учебного центра «Мезон»}

На основании технологий и требований, выделенных в главах 1 и 2 можно составить краткий план выполнения практической части работы.
Необходимо обеспечить процесс разработки документацей и планами вносимых изменений.

В первую очередь необходимо подготовить схему планируемой функциональной структуры ресурса.
На основании данной схемы будут строится взаимосвязи между программными решениями, используемыми для разработки и дальнейшей работы сайта.
% // TODO схему планируемой функциональной структуры ресурса

Следующий этап - формирование структурной схемы веб-ресурса.
В её задачи будет входить упрощенное описание компонентов-страниц сайта, расположение и отношение между ними.
Этот шаг позволит с пониманием общей картины подходить к разработке отдельных частей ресурса.
% // TODO структурной схемы веб-ресурса.

Следующим важным решением будет выделение ключевых сущностей для хранения в базе данных, их взаимоотношения и зависимостей.
На основе этих данных будут строится отдельные компоненты страниц веб-ресурса с использованием шаблонов.
% // TODO  ключевых сущностей для хранения в базе данных

% // TODO Контекстная диаграмма  IDEF0 
% // TODO Декомпозиция первого уровня диаграммы IDEF0
% // TODO Декомпозиция второго уровня диаграммы IDEF0
% // TODO Диаграмма вариантов использования 
% // TODO Диаграмма деятельности разрабатываемого портала


\subsection{Функциональная структура ресурса}

Исходя из определенных для использования в главах 1 и 2 программных средств, можно составить функциональную структуру ресурса.
Она призвана отобразить основные элементы системы, обеспечивающие работу элементов сайта, а также взаимосвязи между ними.

Выбранными элементами стека MEAN с внесёнными правками являются - SQLite в роли СУБД, Express и Node.js для организации серверной части приложения, Angular для работы с клиентской частью.
Также стоит упомянуть CSS-библиотеку Bulma, которая также будет задействована в данной работе.
На основе приведенных программных средств составлена схема функциональных элементов на рисунке \ref{func-struct-scheme}.

\addimghere{images/функциональная-структура.png}{0.5}{Функциональная структура веб-ресурса}{func-struct-scheme}

\subsection{Структура ресурса}

% // TODO Выделение основных страниц и сущностей веб-ресурса
Главная страница исходного сайта содержит 4 основных раздела - Главная, Сведения об организации, Обучение, Аренда, а также различные вспомогательные ссылки.
Схема основных переходов внутри главной страницы отображена на рисунке \ref{meson-old-main}.

\addimghere{images/главная.png}{1}{Схема строения переходов исходного веб-ресурса}{meson-old-main}

На схеме отмечены переходны на неактуальные страницы, которые по разным причиным либо давно не обновлялись, либо отсутствуют вовсе.
Более глубокий анализ состава исходного сайта и переходов в нем  представить в формате схемы проблематично - присутствует большое количество тупиковых, дублирующих, неактуальных элементов.

При этом основными функциональными элементами на главной странице являются ссылки «Расписание», «Стоимость» и «Брошюра».
Взаимодействие с данными, представленными на целевых страницах этих ссылок содержит самую полезную информацию, которая может потребоваться для клиентов учреждения.
Помимо этого, контент этих элементов частично дублируется и частично взаимно противоречит.

Раздел «Сведения об организации» содержит наибольшее количество вложенных страниц - «Основные сведения», «Документы», «Образование», «Образовательные стандарты», «Руководство и педагогический состав», «Структура и органы управления», «Материально-техническое обеспечение и оснащенность образовательного процесса», «Стипендии и иные виды материальной поддержки», «Платные образовательные услуги», «Финансово-хозяйственная деятельность», «Вакантные места для приема (перевода)».
Большая часть содержимого этих подстраниц либо дублируется в других частях ресурса, либо является вовсе неактуальной.

Раздел «Обучение» состоит из подразделов «Компьютерная школа», «Лаборатории инновационного творчества», «Школа личностного роста», «ИТ-курсы для взрослых».
Содержимое этих частей ресурса по большей части устарело и является неактуальным.

Раздел «Аренда» содержит информацию об аренде помещений учебного центра.
Часть информации данной страницы является неактуальной.
Подразделы здесь отсутствуют.

Помимо переходов на обозначенные разделы со страницы «Главная», также есть необозначенные там три - «Независимая оценка качества», «Противодействие коррупции», «Инклюзивное образование».
Состав этих элементов также является частично неактуальным.

% // TODO sitemapmaker - карта переходов исходного сайта

Таким образом можно сделать вывод о заброшенности большинства разделов сайта.
Часть этих страниц можно актуализировать и восстановить в новой версии, а часть объединить с другими.

На основе проведенного анализа можно выделить ключевые разделы проектируемого веб-ресурса.
А именно: «Поступление», «Курсы», «Расписание», «Работы учеников», «Об организации», «Оплата и аренда».

Раздел «Поступление» рассказывает пользователю о правилах записи на курсы учебного центра.

Раздел «Курсы» представляет клиенту набор имеющихся для проведения курсов обучения с возможностью подробного рассмотрения особенностей каждого из них в отдельности.

Раздел «Расписание» включает в себя расписание занятий групп на учебный год, позволяет оперативно сориентировать клиентов касательно неучебных дней в организации.

Раздел «Работы учеников» представляет клиентам возможность ознакомиться с выставленными работами студентов разных направлений.

Раздел «Об организации» содержит информацию о компании, руководстве, а также обширный набор вспомогательной документации, полезной для ознакомления.

Раздел «Оплата и аренда» содержит информацию о способах оплаты услуг учебного центра и предоставляемых в аредну помещениях.

Также стоит отметить точку входа на сайт - раздел «Главная», встречающий пользователя.
Здесь описываются функциональные модули-страницы сайта и их содержимое, а также осовные данные об учебном центре.

На основании приведенных разделов можно сформировать структурную схему веб-ресурса, отображающую получившуюся иерархию страниц.
Результат формирования такой структуры представлен на Рисунке \ref{struct-scheme}.

\addimghere{images/структура-ресурса.png}{1}{Структурная схема проектируемого веб-ресурса}{struct-scheme}


\subsection{Проектирование }

% // TODO Выделение основных сущностей веб-ресурса

% // TODO Схема сущностей - выделение параметров, требуемых для работы в приложении

\clearpage
