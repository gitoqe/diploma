\section{ПРЕДМЕТНАЯ ОБЛАСТЬ И ПОСТАНОВКА ЗАДАЧИ}

Область разработки веб-ресурсов состоит из нескольких смежных областей информационных технологий.
Среди задействованных -- проектирование баз данных, разработка интерфейсов, разработка клиентской части, разработка серверной части, управление проектами, тестирование программного обеспечения и многие другие.
Наиболее близкими сферами, с которыми производится взаимодействие в данной работе, являются непосредственно разработка и проектирование.

\subsection{Выбор используемых программных средств}

Из проведенного в Главе 1 анализа можно выделить основные требования к инструментам разработки и технологиям, задействованным в этом процессе.
Проходя по разобранным категориям дифференциации сайтов, можно выделить набор характеристик, которым должно соответствовать итоговый результат.

Функционально веб-ресурс должен будет соответствовать смеси сайта-визитки и интернет-портала, поскольку основной задачей будет являться привлечение новых и информирование имеющихся клиентов компании.

По отзывчивости дизайна лучшим вариантом будет выбор в сторону комбинации адаптивных макетов и отзывчивого способа работы шаблонов страниц сайта.

Относительно способа формирования контента предпочтение отдается динамическому.
В таком случае можно будет заранее предусмотреть возможность расширения функционала под новые задачи без необходимости перестроения ресурса с нуля.

По типу контента веб-ресурс будет привязан к корпоративной тематике.
Это обусловлено изначальной задачей, решаемой сайтом, которая не будет меняться в ходе выполнения данной работы.

Также на основе рассмотренных в Главе 1 средств разработки веб-ресурсов, необходимо выделить технологии, подходящие под описанные выше критерии.

Безусловно, современная веб-разработка завязана на использование как минимум языка разметки гипертекста HTML и каскадных таблиц стилей CSS.
Данная работа не станет исключением и также будет задействовать их как в прямом виде, так и опосредованно через использование прочего программного обеспечения.

Следующим ключевым решением будет выбор между использованием системы управления контентом (а также присущей ей экосистемы) и самостоятельным проектированием.
По работе Киямова Р.В., Хмелева Е.А. и Юнусова И.Ф. \cite{kiyamov-cms}, а также работе Иванищевой А.А., Комилова Х.И. и М.Д. Гехаева \cite{ivanisheva-cms} можно выделить преимущества и недостатки представленных на рынке CMS.
Исходя из приведенных доводов оптимальным вариантом может явиться WordPress.

Делая выбор между CMS и самостоятельно разработкой компонентов веб-ресурса, можно сделать предположение о большей полезности в выборе последнего.
Аргументировать такую позицию можно тем, что такой подход потребует задействование большего объема технологий и их возможностей для реализации готового веб-ресурса.
И, как следствие, количество шагов в выполнении данной работы увеличится, что приведет к большему количеству материалов, которые можно продемонстрировать, а также задействованию и изучению большего спектра технических средств.

При разработке веб-ресурса без использования систем управления контентом важным пунктом является выбор используемых технологий.
В Главе 1 приводились наборы технических средств -- стеков, рассмотренные в работе за авторством М.А. Давыдовского \cite{davidovsky-vibor}.
Среди указанных в статье, наиболее выделяются MEAN и MERN.
Первый предполагает использование Angular как средства разработки клиентской части, второй же задействует на этой роли React.
Остальные пункты остаются неизменными -- использование MongoDB в роли СУБД, Express и Node.js для организации серверной части приложения.

Ещё одним важным этапом является выбор способа взаимодействия между клиентской и серверной частью.
За данный пункт отвечают специальные интерфейсы прикладных программ -- API (application programming interface), выполняющие роль посредника между частями приложения и определяющие сам способ передачи информации, её вид.

Наиболее популярным является архитектурный подход REST, основывающийся на протоколе HTTP для транспортировки данных.
Например, в приведённом выше стеке MERN/MEAN подход REST реализуется на базе технологии Express.js.
Альтернативным к REST подходом, является GraphQL, рассматривающий взаимодействие и работу с данными через графы.

Подводя промежуточный итог можно выделить особенности проектируемого приложения.
Использоваться будет стек MEAN с некоторыми изменениями.
Ввиду возникшей геополитической ситуации в 2022 году, большое количество компаний-разработчиков наиболее востребованных программных средств отказались от предоставления своих услуг и сервисов на территории Российской Федерации.
В связи с этим возникает необходимость замены недоступных решений на альтернативные.

Одним из сервисов, приостановивших свою деятельность на территории Российской Федерации, является MongoDB -- ключевой элемент MEAN стека, обеспечивающий хранение и взаимодействие с данными в приложении.
По этой причине возникает необходимость поиска другой СУБД.
Одним из подходящих вариантов является SQLite -- свободно распространяемая встраиваемая система для работы с базами данных.

Для остальных частей стека MEAN производить манипуляции поиска и замены на альтернативы не требуется.
В соответствии с этим будут задействованы следующие технологии -- Angular и применяемый в нем язык программирования TypeScript, Node.js и Express.js для реализации интерфейсов взаимодействия с базой данных на основе архитектурного подхода REST и серверной логики приложения.
Для работы с шаблонами страниц будет использоваться HTML и CSS.

Помимо составных элементов стека MEAN, преобразованного в SEAN ввиду замены СУБД Mongo DB на SQLite, также имеются различные дополнительные программные средства, использование которых широко распространено в профессиональной деятельности веб-разработки.
Например, для упрощения настройки параметров внешнего отображения шаблонов применяются готовые решения -- CSS библиотеки и фреймворки, которые содержат определения базовых стилей как собственных, так и наиболее распространенных компонентов, к которым можно отнести кнопки, поля ввода, абзацы, списки и прочие элементы HTML-разметки.

Одним из простых и бесплатных решений является Bulma \cite{bulma}.
Данный фреймворк, как и большинство представленных в сфере разработки веб-ресурсов, предоставляет функционал по быстрой настройке дизайна, его адаптивности под различные устройства, а также предлагает способы формирования некоторых ключевых элементов веб-ресурса.

Например, с помощью Bulma можно выстроить систему выпадающих списков, задать им цветовое оформление и размеры с помощью определенного набора классов, распределенных по отдельным тегам HTML.
Таким образом получается достичь универсальности подходов в формировании элементов веб-ресурса и упрощения процесса разработки с точки зрения дизайн-кода.
Также важно отметить, что вводимые фреймворком правила не являются абсолютными и подвергаются кастомизации в пределах допустимого сохранения работоспособности.

\subsection{Средства проверки}\label{средства-проверки}

Для контроля процесса разработки широко распространено использование различных вспомогательных программных решений.
Их классификация может сильно отличаться в зависимости от требуемого результата.
Контроль процесса разработки может потребоваться на разных этапах.
Основная задача проверок -- формирование уверенности в качествах итогового продукта, таких как удобство использования, корректное отрабатывание функционала, удобство чтения и доработки имеющейся кодовой базы, отсутствие явных ошибок при выполнении основной деятельности программы.

Одной из первых проверок, полезно влияющих на дальнейшие этапы разработки, является применение однотипного подхода в кодировании функционала приложения.
Четкие правила и требования к исходному коду позволят упростить процессы ознакомления и внесения доработок в веб-ресурс.
Для достижения таких  характеристик в веб-разработке обычно применяются средства форматирования -- форматтеры и подсветки кода в соответствии с заданными правилами -- линтеры.

Наиболее распространенным линтером для веб-разработки на JavaScript и TypeScript является ESlint \cite{eslint}.
С его помощью можно описать набор правил, необходимых для соблюдения в коде.
При нарушении обозначенных ограничений линтер будет выдавать предупреждения и ошибки в зависимости от установленных параметров.
Таким образом в процессе разработки появляется контроль за соблюдением обозначенных правил, что приведет к порядку в кодовой базе.

Для автоматизации соблюдения обозначенных линтером ограничений применяются средства форматирования -- форматтеры.
Их задача -- основываясь на заданном перечне правил внести изменения в код.
При этом процесс форматирования тоже может быть перенастроен,

Среди форматтеров, применяемых в веб-разработке JavaScript и TypeScript можно выделить Prettier \cite{prettier}.
Его функционал подходит описанию средств автоматического форматирования -- при обращении к нему можно указать целевые файлы и необходимые параметры, после чего предлагаемый код будет изменен в соответствии с заданными требованиями.

Применение форматтеров и линтеров может сильно повысить качество кодовой базы за счет повсеместного соблюдения общепринятых (в команде или организации) требований.

Проверки работоспособности взаимодействия с серверной частью через проектируемый API также необходимо иметь возможность в удобном формате отправлять и получать запросы.
Для таких целей могут применяться различные платформы по разработке программных интерфейсов приложения.
Они могут включать в себя большой функционал по созданию и проверке API различной степени сложности.

Пример такой платформы разработки программного интерфейса приложения -- Postman \cite{postman}.
Данный сервис представляет все описанные выше возможности средств такого рода.

\subsection{Анализ исходного состояния веб-ресурса}

Для проведения первичной оценки состояния веб-ресурса необходимо принимать во внимание множество различных факторов, в совокупности преобразующееся в общее впечатление.
Можно воспользоваться универсальными автоматизированными средствами проверки сайтов.

Одно из таких -- Lighthouse \cite{lighthouse}.
Это инструмент с открытым исходным кодом, разрабатываемый компанией Google \cite{google} и входящий в состав браузера Google Chrome \cite{chrome}.
В его функционал входит возможность проведения многоплановой оценки предлагаемого веб-ресурса как с точки зрения десктопной версии, так и мобильной.

На рисунке \ref{lighthouse-meson-mobile-1} результат проведения проверки исходного ресурса на соответствие основным требованиям современной веб-разработки на мобильном устройстве.

\addimghere{images/lighthouse-meson-mobile.png}{0.6}{Результат проверки главной страницы исходного веб-ресурса с помощью средств Lighthouse для мобильного устройства}{lighthouse-meson-mobile-1}

Также в Lighthouse есть возможность отдельной проверки работоспособности сайта на десктопной версии.
Результат такой проверки отображен на рисунке \ref{lighthouse-meson-desktop-1}.

\addimghere{images/lighthouse-meson-desktop.png}{0.6}{Результат проверки главной страницы исходного веб-ресурса с помощью средств Lighthouse для десктопного устройства}{lighthouse-meson-desktop-1}

В список оценок входят пять параметров, отвечающих за различные стороны взаимодействия с сайтом -- Perfomance, Accessibility, Best Practices, SEO, PWA.
Измерение проводится по шкале от 0 до 100, где 100 -- максимальное соответствие заданным требованиям категории.
Параметры проверки и выставляемые оценки не являются абсолютными -- некоторые особенности работы веб-ресурса можно проверить только вручную, а использование Lighthouse в данном случае несет вспомогательную функцию для определения успешности внесения изменений.
Далее эти пункты будут разобраны с указанием соответствующих областей проверки.

Параметр Perfomance отвечает за общую производительность сайта и производительность на слабых устройствах.
В рекомендации данного раздела входят предложения по уменьшению размера прилагаемых картинок, сжатия кода и прочие процедуры, призванные повысить скорость загрузки.

Параметр Accessibility соответствует доступности элементов ресурса для различных категорий пользователей.
В рекомендации данного раздела входят предложения по добавлении пояснительных подписей к большинству элементов сайта, позволяющих применять возможности экранного диктора.

Параметр Best Practices отвечает за применение в работе веб-ресурса современных подходов и технологий, а также отсутствие использования безопасных модулей и библиотек.
Рекомендации данного пункта очевидны -- при обнаружении устаревших подходов, либо скомпрометированных элементов кода, будет предложено это исправить.

Параметр SEO соответствует понятию search engine optimization -- оптимизация под поисковые системы.
Для выставления данного параметра, Lighthouse анализирует доступность мета-данных ресурса, их корректное наполнение.
Рекомендации завязаны на исправлении возможных проблем для работы поисковых систем.

Параметр PWA соответствует progressive web app или прогрессивному веб-приложению.
Суть данного понятия заключается в возможности работы сайта в формате приложения на устройстве.
Сам параметр, соответственно, проверяет возможность осуществления такой деятельности веб-ресурсом.

Как видно из результатов проверки, главная страница исходного веб-ресурса хорошо показывает себя по части производительности в показателе Perfomance -- 89 и 100 баллов для мобильной и десктопной версий соответственно.
Это можно объяснить тем, что сам состав страницы является сильно устаревшим, а потому содержит простые конструкции, одинаково успешно работающие как на новых, так и на старых устройствах.

Показатели Best Practices и SEO находятся в средней зоне 50 и 58 для мобильной версии, 58 и 70 для десктопа.
Данные показатели также можно объяснить использованием достаточно старых технологий, которые имеют современные аналоги, а также изменение требований современных поисковых систем к составу метаданных сайта.

Самый низкий показатель Accessibility остался в "красной" зоне -- 49 и 44 для мобильной и десктопной версии соответственно.
В данном случае можно сделать вывод о недостатке подписей для элементов сайта, что приводит к снижению удобства пользования сайтом в нестандартных форматах.

Показатель PWA в оценках вовсе отсутствует, что можно объяснить простым отсутствием поддержки данного параметра на сайте вне зависимости от формы проверки.

После разбора показателей можно подвести итоговую среднюю оценку веб-ресурса -- 61,5 для мобильной версии проверки и 68 для десктопной соответственно.
Данное измерение можно взять основу для проверки успешности вносимых изменений в сайт, где можно будет сравнить показатели до и после.

Помимо проверок возможностей веб-ресурса, важно заметить, что автоматизированные средства не имеют возможности удостовериться в корректности и логичности содержимого.
Например, исходный ресурс содержит большое количество неиспользуемых, устаревших материалов -- различные документы, описания, нерабочие ссылки и прочие элементы, утративших актуальность.

\subsection{Разработка функциональной модели веб-ресурса}

Для построения функциональной модели веб-ресурса можно воспользоваться методологией IDEF0 \cite{wiki-idef0}.
Для этого потребуется сначала оформить контекстную диаграмму проектируемого веб-ресурса, что в свою очередь означает необходимость выделить абстрактное общее описание системы.

Далее можно углубляться в указанные на схеме элементы.
Такой подход позволяет на разных уровнях информационной обеспеченности протекающими процессами поэтапно проанализировать каждый из них с желаемой степенью подробности.

В процессе составления элементов диаграммы могут возникать дополнительные вопросы к частям проектируемого веб-ресурса, что приведет к повышению проработанности как конкретных составляющих, так и системы в целом.

Построенная контекстная диаграмма IDEF0 отображена на рисунке \ref{idef0-drawio}.
\addimghere{images/diagrams/idef0.drawio.png}{0.9}{Контекстная диаграмма  IDEF0}{idef0-drawio}

В методологии IDEF0 организация модулей и их взаимодействие между собой и внешним миром обозначается с помощью стрелок, каждой сопоставляется своя роль.

Входящая информация, используемая блоком получается от стрелки входа, она присоединяется к блоку слева.
Управляющие указания поступают сверху от модуля.
Механизмом или исполнителем является стрелка, примыкающая снизу от блока.
Выход отображается как исходящая из правой кромки модуля стрелка.

Описание стрелок, их типы и характеристики применительно к построенной контекстной диаграмме представлено в таблице \ref{table-context-diagram}.

\begin{small}
\begin{longtable}[h]{| p{5.8cm} | p{7.8cm} | c |}
    \caption{Описание элементов контекстной диаграммы}\label{table-context-diagram}\\
    \hline
    \centering Название стрелки&
    \centering Описание&
    Тип\\
    \hline
    \centering 1&
    \centering 2&
    3\\
    \hline\endfirsthead
    \multicolumn{3}{@{}l}{Продолжение таблицы <<Описание элементов контекстной диаграммы>> \ref{table-context-diagram}}\\
    \hline
    \centering 1&
    \centering 2&
    3\\
    \endhead

    Пользовательское соглашение&
    Пользовательское соглашение, содержащее условия использования функционала сайта, между клиентом-пользователем портала и администратором-владельцем веб-ресурса&
    Control\\

    \hline
    Администратор сайта&
    Производит сбор, проверку и редактуру входящей информацию в соответствии с требованиями учебного центра&
    Mechanism\\

    \hline
    Информация о курсах&
    Информация о курсах, предлагаемых учебным центром, их описание и план занятий&
    Input\\
    \hline
    Информация о поступлении&
    Информация о необходимых документах, стоимости обучения и предлагаемых скидках&
    Input\\
    \hline
    Информация о расписании&
    Информация о расписании активных групп учебного центра&
    Input\\
    \hline
    Информация о работах учеников&
    Информация о примерах работ учеников различных курсов&
    Input\\
    \hline
    Информация об организации&
    Информация о коллективе, документация и сведениях об учебном центре&
    Input\\
    \hline
    Информация об оплате и аренде&
    Информация о способах оплаты обучения и предлагаемых услугах аренды помещений&
    Input\\

    \hline
    Компонент с информацией о курсах&
    Информационный блок, содержащий отредактированные и отобранные данные о проводимых курсах, их описании и плане занятий&
    Output\\
    \hline
    Компонент с информацией о поступлении&
    Информационный блок, содержащий отредактированные и отобранные данные о необходимых документах, стоимости обучения и предлагаемых скидках&
    Output\\
    \hline
    Компонент с информацией о расписании&
    Информационный блок, содержащий отредактированные и отобранные данные о расписании активных групп учебного центра&
    Output\\
    \hline
    Компонент с информацией о работах учеников&
    Информационный блок, содержащий отредактированные и отобранные данные о примерах работ учеников различных курсов&
    Output\\
    \hline
    Компонент с информацией об организации&
    Информационный блок, содержащий отредактированные и отобранные данные о коллективе, документации и сведениях об учебном центре&
    Output\\
    \hline
    Компонент с информацией об оплате и аренде&
    Информационный блок, содержащий отредактированные и отобранные данные о способах оплаты обучения и предлагаемых услугах аренды помещений&
    Output\\
    \hline
\end{longtable}
\end{small}

Следующим этапом формирования функциональной модели деятельности веб-ресурса является более глубокий разбор полученной схемы.
Для этого необходимо провести декомпозицию первого уровня готовой диаграммы IDEF0.
В данном случае необходимо разобрать процесс <<Работа веб-ресурса учебного центра>> на составляющие.

Можно выделить следующие этапы: <<Взаимодействие с преподавательским составом и руководством УЦ>>, <<Внесение изменений в базу данных веб-ресурса УЦ>>, <<Проверка и формирование готовых компонентов веб-ресурса>>.

Результат проведения декомпозиции первого уровня над полученной диаграммой IDEF0 отображен на рисунке \ref{idef0-decompose-1}.
\addimghere{images/diagrams/idef0-decompose-1.drawio.png}{1.0}{Декомпозиция первого уровня исходной диаграммы IDEF0}{idef0-decompose-1}

В полученной схеме можно провести еще один этап декомпозиции -- блок <<Проверка и формирование готовых компонентов веб-ресурса>> состоит из частей <<Проверка связей между базой данных и компонентами>>, <<Внесение изменений в компоненты>> и <<Проверка корректности работы компонента>>.
Результат проведения декомпозиции второго уровня над этим блоком отображен на рисунке \ref{idef0-decompose-2}.
\addimghere{images/diagrams/idef0-decompose-2.drawio.png}{1.0}{Декомпозиция второго уровня исходной диаграммы IDEF0}{idef0-decompose-2}

\subsection{Разработка диаграммы вариантов использования}

Диаграмма вариантов использования поможет отобразить отношения между активными участниками процесса использования веб-ресурса и различными типами взаимодействия.
На основе этих связей можно будет выявить способы решения задач пользователей системы.
На рисунке \ref{variants-both} отображены диаграммы вариантов использования для разрабатываемого веб-ресурса применительно к роли администратора и пользователя.

\addimghere{images/diagrams/variants-both.drawio.png}{0.6}{Диаграмма вариантов использования для администратора и пользователя}{variants-both}

Возможности администратора ограничены взаимодействием с отображаемым контентом сайта -- редактированием, добавлением и удалением информации в базе данных, а также внесением правок в состав компонентов.
Варианты взаимодействия пользователя описываются стандартным набором вариантов взаимодействия на сайтах.
В этот набор входим просмотр информации, переключение между разделами, использование формы связи и переход на прочие ресурсы учебного центра.

\subsection{Разработка диаграммы деятельности}

Диаграмма деятельности пользователей разрабатываемого портала отображена на рисунке \ref{occupation}.

\addimghere{images/diagrams/occupation.drawio.png}{0.7}{Диаграмма деятельности пользователей разрабатываемого портала}{occupation}

В задачи данной диаграммы входит отображение в формате, схожем с блок-схемой алгоритма, различных путей прохождения этапов взаимодействия с веб-ресурсом.

Прямое взаимодействие начинается с загрузки частей веб-ресурса, в которые входят файлы скриптов, гипертекстовой разметки, каскадных таблиц стилей, а также используемые медиа-файлы.
Далее производится отображение загруженного контента в виде главной страницы, которая будет содержать в себе остальные компоненты сайта.
Первоначально в этот контейнер помещается домашняя страница, с которой уже можно перейти на побочные.

Последующие варианты взаимодействия зависят от потребностей пользователя -- можно как ознакомится с содержимым прочих разделов, а можно воспользоваться адресной строкой браузера для ввода адреса скрытого компонента авторизации.

При переходе на страницу входа администратора отобразится форма для ввода логина и пароля.
После ввода заранее полученных данных производится процесс проверки.
При успешном результате проведения процесса авторизации будет открыт доступ к странице редактирования содержимого компонентов, где можно рассмотреть и внести правки в отображаемый контент.
После завершения взаимодействия с позиции администратора, можно произвести операцию выхода из этого режима.
В случае ввода некорректных данных в форму логина будет выдано сообщение об ошибке с возможностью повторного ввода данных.


\subsection{Постановка задачи}\label{Постановка задачи}

Основной целью ВКР является разработка веб-ресурса учебного центра, соответствующего современным представлениям о сайте организации, занимающейся обучением в сфере информационных технологий.
Анализ, проведенный в первой и второй главах привел к формированию основных пунктов, требуемых для воплощения, также способах их реализации.

В клиентской части для отображения ключевых особенностей работы организации, информации о курсах и учебной деятельности подойдет формат информационного портала.

Для удобства взаимодействия с веб-ресурсом требуется разработать и скомпоновать навигационные интерактивные элементы, позволяющие быстро и интуитивно пользоваться разделами сайта.

Необходимо реализовать структуру хранения информации и взаимодействия с ней.
Использование базы данных подойдет для однотипной информации, частично повторяющейся.
Под такое описание подойдут описания курсов, информация об аудиториях, преподавателях, документах.
Для удобства изменения такой информации, а соответственно и отображаемого контента на сайте, можно использовать встроенный непосредственно в веб-ресурс способ взаимодействия с хранилищем.
При применении такого способа открывается возможность внесения правок, добавления или удаления содержимого.

Интерфейсы взаимодействия с серверной с использованием клиентской зачастую требуют особого внимания к контролю доступа к таким возможностям.
В связи с этим необходимо также добавить функционал редактирования контента страниц, формируемых из хранилища.

С точки зрения серверной части необходимо проработать способы взаимодействия клиентской стороны и базы данных для получения, хранения и редактирования информации.
При этом необходимо оставить возможность изменения информации в хранилище с помощью подручных средств -- систем управления базами данных, либо других с схожим функционалом.

К общим требованиям можно отнести повышение оценок проверки, полученных в подразделе \ref{средства-проверки}.
Таким образом средние показатели переработанной версии ресурса должны превышать  61,5 для мобильной версии и 68 для десктопной соответственно.
В данное требование входят требования к оптимизация кода запросов, используемых стилей и задействованных графических элементов.

\clearpage
