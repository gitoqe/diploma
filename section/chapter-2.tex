\section{(2) Анализ предметной области разработки Web-ресурса и постановка задачи}

% // TODO Анализ предметной области разработки Web-ресурса

\subsection{Средства разработки}

Из проведенного в Главе 1 анализа можно выделить основные требования к инструментам разработки и технологиям, задействованным в этом процессе.
Проходя по разобранным категориям дифференциации сайтов, можно выделить набор характеристик, которым должно соответствовать итоговый результат.

Функционально веб-ресурс должен будет соответствовать сайту-визитке, поскольку основной задачей будет являться привлечение новых и информирование имеющихся клиентов компании.

По отзывчивости дизайна лучшим вариантом будет выбор в сторону комбинации адаптивных макетов и отзывчивого способа работы шаблонов страниц сайта.

Относительно способа формирования контента предпочтение отдается динамическому.
В таком случае можно будет заранее предусмотреть возможность расширения функционала под новые задачи без необходимости перестроения ресурса с нуля.

По типу контента веб-ресурс будет привязан к корпоративной тематике.
Это обусловлено изначальной задачей, решаемой сайтом, которая не будет меняться в ходе выполнения данной работы.

Также на основе рассмотренных в Главе 1 средств разработки веб-ресурсов, необходимо выделить технологии, подходящие под описанные выше критерии.

Безусловно, современная веб-разработка завязана на использование как минимум языка разметки гипертекста HTML и каскадных таблиц стилей CSS.
Данная работа не станет исключением и также будет задействовать их как в прямом виде, так и опосредованно через использование прочего программного обеспечения.

Следующим ключевым решением будет выбор между использованием системы управления контентом (а также присущей ей экосистемы) и самостоятельным проектированием.
По работе Киямова Р.В., Хмелева Е.А. и Юнусова И.Ф. \cite{kiyamov-cms}, а также работе Иванищевой А.А., Комилова Х.И. и М.Д. Гехаева \cite{ivanisheva-cms} можно выделить преимущества и недостатки представленных на рынке CMS.
Исходя из приведенных доводов оптимальным вариантом может явиться WordPress.

Делая выбор между CMS и самостоятельно разработкой компонентов веб-ресурса, можно сделать предположение о большей полезности в выборе последнего.
Аргументировать такую позицию можно тем, что такой подход потребует задействование большего объема технологий и их возможностей для реализации готового веб-ресурса.
И, как следствие, количество шагов в выполнении данной работы увеличится, что приведет к большему количеству материалов, которые можно продемонстрировать, а также задействованию и изучению большего спектра технических средств.

При разработке веб-ресурса без использования систем управления контентом важным пунктом является выбор используемых технологий.
В Главе 1 приводились наборы технических средств -- стеков, рассмотренные в работе за авторством М.А. Давыдовского \cite{davidovsky-vibor}.
Среди указанных в статье, наиболее выделяются MEAN и MERN.
Первый предполагает использование Angular как средства разработки клиентской части, второй же задействует на этой роли React.
Остальные пункты остаются неизменными -- использование MongoDB в роли СУБД, Express и Node.js для организации серверной части приложения.

Ещё одним важным этапом является выбор способа взаимодействия между клиентской и серверной частью.
За данный пункт отвечают специальные интерфейсы прикладных программ - API (application programming interface), выполняющие роль посредника между частями приложения и определяющие сам способ передачи информации, её вид.

Наиболее популярным является архитектурный подход REST, основывающийся на протоколе HTTP для транспортировки данных.
Например, в приведённом выше стеке MERN/MEAN подход REST реализуется на базе технологии Express.js.
Альтернативным к REST подходом, является GraphQL, рассматривающий взаимодейтсвие и работу с данными через графы.

Подводя промежуточный итог можно выделить особенности проектируемого приложения.
Использоваться будет стек MEAN с некоторыми изменениями.
Ввиду возникшей геополитической ситуации в 2022 году, большое количество компаний-разработчиков наиболее востребованных программных средств отказались от предоставления своих услуг и сервисов на территории Российской Федерации.
В связи с этим возникает необходимость замены недоступных решений на альтернативные.

Одним из сервисов, приостановивших свою деятельность на территории Российской Федерации, является MongoDB -- ключевой элемент MEAN стека, обеспечивающий хранение и взаимодействие с данными в приложении.
По этой причине возникает необходимость поиска другой СУБД.
Одним из подходящих вариантов является SQLite -- свободно распространяемая встраиваемая система для работы с базами данных.

Для остальных частей стека MEAN производить манипуляции поиска и замены на альтернативы не требуется.
В соответствии с этим будут задействованы следующие технологии -- Angular и применяемый в нем язык программирования TypeScript, Node.js и Express.js для реализации интерфейсов взаимодействия с базой данных на основе архитектурного подхода REST и серверной логики приложения.
Для работы с шаблонами страниц будет использоваться HTML и CSS.

% // TODO ESLINT
% // TODO POSTMAN
% // TODO NODEMON
% // TODO BULMA

\subsection{Постановка задачи}
% // TODO Постановка задачи



\clearpage
