\section{(2) Анализ предметной области разработки Web-ресурса и постановка задачи}

% // TODO Анализ предметной области разработки Web-ресурса

\subsection{Выбор используемых программных средств}

Из проведенного в Главе 1 анализа можно выделить основные требования к инструментам разработки и технологиям, задействованным в этом процессе.
Проходя по разобранным категориям дифференциации сайтов, можно выделить набор характеристик, которым должно соответствовать итоговый результат.

Функционально веб-ресурс должен будет соответствовать сайту-визитке, поскольку основной задачей будет являться привлечение новых и информирование имеющихся клиентов компании.

По отзывчивости дизайна лучшим вариантом будет выбор в сторону комбинации адаптивных макетов и отзывчивого способа работы шаблонов страниц сайта.

Относительно способа формирования контента предпочтение отдается динамическому.
В таком случае можно будет заранее предусмотреть возможность расширения функционала под новые задачи без необходимости перестроения ресурса с нуля.

По типу контента веб-ресурс будет привязан к корпоративной тематике.
Это обусловлено изначальной задачей, решаемой сайтом, которая не будет меняться в ходе выполнения данной работы.

Также на основе рассмотренных в Главе 1 средств разработки веб-ресурсов, необходимо выделить технологии, подходящие под описанные выше критерии.

Безусловно, современная веб-разработка завязана на использование как минимум языка разметки гипертекста HTML и каскадных таблиц стилей CSS.
Данная работа не станет исключением и также будет задействовать их как в прямом виде, так и опосредованно через использование прочего программного обеспечения.

Следующим ключевым решением будет выбор между использованием системы управления контентом (а также присущей ей экосистемы) и самостоятельным проектированием.
По работе Киямова Р.В., Хмелева Е.А. и Юнусова И.Ф. \cite{kiyamov-cms}, а также работе Иванищевой А.А., Комилова Х.И. и М.Д. Гехаева \cite{ivanisheva-cms} можно выделить преимущества и недостатки представленных на рынке CMS.
Исходя из приведенных доводов оптимальным вариантом может явиться WordPress.

Делая выбор между CMS и самостоятельно разработкой компонентов веб-ресурса, можно сделать предположение о большей полезности в выборе последнего.
Аргументировать такую позицию можно тем, что такой подход потребует задействование большего объема технологий и их возможностей для реализации готового веб-ресурса.
И, как следствие, количество шагов в выполнении данной работы увеличится, что приведет к большему количеству материалов, которые можно продемонстрировать, а также задействованию и изучению большего спектра технических средств.

При разработке веб-ресурса без использования систем управления контентом важным пунктом является выбор используемых технологий.
В Главе 1 приводились наборы технических средств -- стеков, рассмотренные в работе за авторством М.А. Давыдовского \cite{davidovsky-vibor}.
Среди указанных в статье, наиболее выделяются MEAN и MERN.
Первый предполагает использование Angular как средства разработки клиентской части, второй же задействует на этой роли React.
Остальные пункты остаются неизменными -- использование MongoDB в роли СУБД, Express и Node.js для организации серверной части приложения.

Ещё одним важным этапом является выбор способа взаимодействия между клиентской и серверной частью.
За данный пункт отвечают специальные интерфейсы прикладных программ - API (application programming interface), выполняющие роль посредника между частями приложения и определяющие сам способ передачи информации, её вид.

Наиболее популярным является архитектурный подход REST, основывающийся на протоколе HTTP для транспортировки данных.
Например, в приведённом выше стеке MERN/MEAN подход REST реализуется на базе технологии Express.js.
Альтернативным к REST подходом, является GraphQL, рассматривающий взаимодейтсвие и работу с данными через графы.

Подводя промежуточный итог можно выделить особенности проектируемого приложения.
Использоваться будет стек MEAN с некоторыми изменениями.
Ввиду возникшей геополитической ситуации в 2022 году, большое количество компаний-разработчиков наиболее востребованных программных средств отказались от предоставления своих услуг и сервисов на территории Российской Федерации.
В связи с этим возникает необходимость замены недоступных решений на альтернативные.

Одним из сервисов, приостановивших свою деятельность на территории Российской Федерации, является MongoDB -- ключевой элемент MEAN стека, обеспечивающий хранение и взаимодействие с данными в приложении.
По этой причине возникает необходимость поиска другой СУБД.
Одним из подходящих вариантов является SQLite -- свободно распространяемая встраиваемая система для работы с базами данных.

Для остальных частей стека MEAN производить манипуляции поиска и замены на альтернативы не требуется.
В соответствии с этим будут задействованы следующие технологии -- Angular и применяемый в нем язык программирования TypeScript, Node.js и Express.js для реализации интерфейсов взаимодействия с базой данных на основе архитектурного подхода REST и серверной логики приложения.
Для работы с шаблонами страниц будет использоваться HTML и CSS.

Помимо составных элементов стека MEAN, преобразованного в SEAN ввиду замены СУБД Mongo DB на SQLite, также имеются различные дополнительные программные средства, использование которых широко распространено в профессиональной деятельности веб-разработки.
Например, для упрощения настройки параметров внешнего отображения шаблонов применяются готовые решения -- CSS библиотеки и фреймворки, которые содержат определения базовых стилей как собственных, так и наиболее распространенных компонентов, к которым можно отнести кнопки, поля ввода, абзацы, списки и прочие элементы HTML-разметки.

Одним из простых и бесплатных решений является Bulma \cite{bulma}.
Данный фреймворк, как и большинство представленных в сфере разработки веб-ресурсов, предоставлят функционал по быстрой настройке дизайна, его адаптивности под различные устройства, а также предлагает способы формирования некоторых ключевых элементов веб-ресурса.

Например, с помощью Bulma можно выстроить систему выпадающих списков, задать им цветовое оформление и размеры с помощью определенного набора классов, распределенных по отдельным тегам HTML.
Таким образом получается достичь универсальности подходов в формировании элементов веб-ресурса и упрощения процесса разработки с точки зрения дизайн-кода.
Также важно отметить, что вводимые фреймворком правила не являются абсолютными и подвергаются кастомизации в пределах допустимого сохранения работоспособности.

\subsection{Средства проверки}

Для контроля процесса разработки широко распространено использование различных вспомогательных программных решений.
Их классификация может сильно отличаться в зависимости от требуемого результата.
Контроль процесса разработки может потребоваться на разных этапах.
Основная задача проверок - формирование уверенности в качествах итогового продукта, таких как удобство использования, корректное отрабатывание функционала, удобство чтения и доработки имеющейся кодовой базы, отсутствие явных ошибок при выполнении основной деятельности программы.

Одной из первых проверок, полезно влияющих на дальнейшие этапы разработки, является применение однотипного подхода в кодировании функционала приложения.
Четкие правила и требования к исходному коду позволят упростить процессы ознакомления и внесения доработок в веб-ресурс.
Для достижения таких  характеристик в веб-разработке обычно применяются средства форматирования - форматтеры и подсветки кода в соответствии с заданными правилами - линтеры.

Наиболее распространенным линтером для веб-разработки на JavaScript и TypeScript является ESlint \cite{eslint}.
С его помощью можно описать набор правил, необходимых для соблюдения в коде.
При нарушении обозначенных ограничений линтер будет выдавать предупреждения и ошибки в зависимости от установленных параметров.
Таким образом в процессе разработки появляется контроль за соблюдением обозначенных правил, что приведет к порядку в кодовой базе.

Для автоматизации соблюдения обозначенных линтером ограничений применяются средства форматирования - форматтеры.
Их задача - основываясь на заданном перечне правил внести изменения в код.
При этом процесс форматирования тоже может быть кастомизирован,

Среди форматтеров, применяемых в веб-разработке JavaScript и TypeScript можно выделить Prettier \cite{prettier}.
Его функционал подходит описанию средств автоматического форматирования - при обращении к нему можно указать целевые файлы и необходимые параметры, после чего предлагаемый код будет изменен в соответствии с заданными требованиями.

Применение форматтеров и линтеров может сильно повысить качество кодовой базы за счет повсеместного соблюдения общепринятых (в команде или организации) требований.

Проверки работоспособности взаимодействия с серверной частью через проектируемый API также необходимо иметь возможность в удобном формате отправлять и получать запросы.
Для таких целей могут применяться различные платформы по разработке программных интерфейсов приложения.
Они могут включать в себя большой функционал по созданию и проверке API различной степени сложности.

Пример такой платформы разработки программного интерфеса приложения - Postman \cite{postman}.
Данный сервис представляет все описанные выше возможности средств такого рода.

% // TODO Lighthouse

% // TODO Тестирование черезе angular-karma, Nx, Jest

\subsection{Анализ исходного состояния веб-ресурса}

Для проведения первичной оценки состояния веб-ресурса необходимо принимать во внимание монжество различных факторов, в совокупности преобразующееся в общее впечатление.
Можно воспользоваться универсальными автоматизированными средствами проверки сайтов.

Одно из таких - Lighthouse \cite{lighthouse}.
Это инструмент с открытым исходным кодом, разрабатываемый компанией Google \cite{google} и входящий в сотав браузера Google Chrome \cite{chrome}.
В его функционал входит возможность проедения многоплановой оценки предлагаемого веб-ресурса как с точки зрения десктопной версии, так и мобильной.

% // TODO рисунок декстопа
% // TODO рисунок мобилки
% // TODO анализ результатов, выводы 

% // TODO указать про неиспользуемые страницы, старые документы, нерабочие ссылки, неактуальные элементы

% // TODO Выявление требований к проектируемому ресурсу

% // TODO Админ-панель и система логина - почему не нужна

\subsection{Постановка задачи}
% // TODO Постановка задачи



\clearpage
