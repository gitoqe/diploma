\section{Разработка интерфейса системы}

Для разработки интерфейса веб-ресурса потребуется определится с несколькими ключевыми пунктами.
В них входят: цветовая схема, набор используемых шрифтов, выбор используемых библиотек и фреймворков.

\subsection{Используемые технологии для разработки интерфейса}

Принимать сложные решения касательно цветовой схемы не придется -- базовые цвета, на основе которых можно построить всю систему оформления уже присутствуют в существующем логотипе компании.
На рисунке \ref{logo_uc} показан текущий логотип компании.

\addimghere{images/logo_uc}{0.4}{Логотип учебного центра <<Мезон>>}{logo_uc}

В логотипе используются базовые цвета, задействованные также и в дополнительной документации, буклетах.
В коде приложения данные цвета можно закрепить в основном стилевом файле приложения.
На рисунке \ref{code-front-scss-colors} отображена часть кода основного стилевого файла с определениями базовых цветов на основе логотипа учебного центра.

\addimghere{images/code/code-front-scss-colors.png}{0.4}{Определения основных переменных для работы с цветовым оформлением}{code-front-scss-colors}

Следующим важным решением является выбор используемых шрифтов.
Хорошо подобранный набор используемых символов облегчит ориентацию по сайту и позволит удобно пользоваться сайтом на разных устройствах.
Среди свободно распространяемых шрифтов особое место занимают предлагаемые компанией Google Lato \cite{google-lato} и Nunito \cite{google-nunito}.
На рисунке \ref{code-front-scss-fonts} показан код подключения данных шрифтов в основной стилевой файл.

\addimghere{images/code/code-front-scss-fonts.png}{0.7}{Подключение шрифтов Lato и Nunito в основном стилевом файле}{code-front-scss-fonts}

Данные шрифты поддерживают кириллицу и латиницу, а также содержат основной набор специальных символов.

Как было указано в главе 2, для упрощения задачи оформления базовых элементов будет задействована CSS-библиотека Bulma.
Она позволит настраивать компоненты HTML-страниц простым добавлением классов.
На рисунке \ref{bulma-button-style} пример оформления элементов-кнопок с официального сайта Bulma \cite{bulma}.

\addimghere{images/examples/bulma-button-style.png}{1}{Настройка стилей кнопок с использованием Bulma}{bulma-button-style}

Также полезным может оказаться применение наборов иконок для обозначения некоторых элементов интерфейса.
Для таких задач можно снова воспользоваться свободно распространяемой библиотекой Font Awesome \cite{fontawesome}.
При подключении данной библиотеки производится путем внедрения в основной HTML-документ скрипта.
На рисунке \ref{code-front-fontawesome} отображен участок кода, отвечающий за подключение библиотеки Font Awesome.

\addimghere{images/code/code-front-fontawesome.png}{0.8}{Подключение библиотеки Font Awesome в главном HTML-файле веб-ресурса}{code-front-fontawesome}

Остальные решения, касающиеся оформления сайта будут задействовать функционал, предлагаемый Angular.

\subsection{Формирование шаблона основной страницы}
% // TODO Шаблон основной страницы

\subsection{Формирование шаблонов побочных страниц}
% // TODO Шаблоны страниц

% // TODO Схемы шаблонов страниц

\subsection{Проверка внесенных изменений с использованием Lighthouse}
% // TODO Проверка в Lighthouse

% // TODO тесты средствами ангуляра

\clearpage
