\section{РАЗРАБОТКА ИНТЕРФЕЙСА СИСТЕМЫ}

Для разработки интерфейса веб-ресурса потребуется определится с несколькими ключевыми пунктами.
В них входят: цветовая схема, набор используемых шрифтов, выбор используемых библиотек и фреймворков.


\subsection{Используемые технологии для разработки интерфейса}

Принимать сложные решения касательно цветовой схемы не придется -- базовые цвета, на основе которых можно построить всю систему оформления уже присутствуют в существующем логотипе компании.
На рисунке \ref{logo_uc} показан текущий логотип компании.

\addimghere{images/logo_uc}{0.35}{Логотип учебного центра <<Мезон>>}{logo_uc}

В логотипе используются базовые цвета, задействованные также и в дополнительной документации, буклетах.
В коде приложения данные цвета можно закрепить в основном стилевом файле приложения.
В шестнадцатеричном формате указан выбранный цвет, ему сопоставляется символьная переменная.

На рисунке \ref{code-front-scss-colors} отображена часть кода основного стилевого файла с определениями базовых цветов на основе логотипа учебного центра.

\addimghere{images/code/code-front-scss-colors.png}{0.35}{Определения основных переменных для работы с цветовым оформлением}{code-front-scss-colors}

Следующим важным решением является выбор используемых шрифтов.
Хорошо подобранный набор используемых символов облегчит ориентацию по сайту и позволит удобно пользоваться сайтом на разных устройствах.
Среди свободно распространяемых шрифтов особое место занимают предлагаемые компанией Google Lato \cite{google-lato} и Nunito \cite{google-nunito}.
На рисунке \ref{code-front-scss-fonts} показан код подключения данных шрифтов в основной стилевой файл.

\addimghere{images/code/code-front-scss-fonts.png}{0.7}{Подключение шрифтов Lato и Nunito в основном стилевом файле}{code-front-scss-fonts}

Данные шрифты поддерживают кириллицу и латиницу, а также содержат основной набор специальных символов.

Как было указано в главе 2, для упрощения задачи оформления базовых элементов будет задействована CSS-библиотека Bulma.
Она позволит настраивать компоненты HTML-страниц простым добавлением классов.
На рисунке \ref{bulma-button-style} пример оформления элементов-кнопок с официального сайта Bulma \cite{bulma}.

\addimghere{images/examples/bulma-button-style.png}{1}{Настройка стилей кнопок с использованием Bulma}{bulma-button-style}

Также полезным может оказаться применение наборов иконок для обозначения некоторых элементов интерфейса.
Для таких задач можно снова воспользоваться свободно распространяемой библиотекой Font Awesome \cite{fontawesome}.
При подключении данной библиотеки производится путем внедрения в основной HTML-документ скрипта.
На рисунке \ref{code-front-fontawesome} отображен участок кода, отвечающий за подключение библиотеки Font Awesome.

\addimghere{images/code/code-front-fontawesome.png}{0.8}{Подключение библиотеки Font Awesome в главном HTML-файле веб-ресурса}{code-front-fontawesome}

Остальные решения, касающиеся оформления сайта будут задействовать функционал, предлагаемый Angular.


\subsection{Формирование шаблона основной страницы}

Основываясь на исходном внешнем облике сайта, а также современной практике оформления веб-ресурсов, можно предложить частичное сохранение структуры главной страницы.
В верхней части располагается <<шапка>> -- навигационный элемент, содержащий ссылки на основные используемые разделы.
Важно также учесть, что помимо разделов веб-ресурса, есть еще и страница основной организации.
Ссылку на нее, аналогично исходному сайту, предлагается оставить в формате логотипа в навигационной панели слева.
Второй же логотип будет принадлежать веб-ресурсу учебного центра и вести на его главную страницу.
Общий шаблон представления данного элемента представлен на рисунке \ref{design-main-header}.

\addimghere{images/diagrams/design-main-header.png}{0.8}{Макет навигационной панели страницы в десктопном формате}{design-main-header}

В нижней части находится <<подвал>> -- вспомогательный элемент, содержащий, как правило, справочную и контактную информацию.
Сюда можно разместить логотип учебного центра, адрес, основные номера телефонов, ссылки на социальные сети.
Общий шаблон представления данного элемента представлен на рисунке \ref{design-main-footer}.

\addimghere{images/diagrams/design-main-footer.png}{0.9}{Макет <<подвала>> стандратной страницы в десктопном формате}{design-main-footer}

Общий шаблон представления интерфейса для десктопной версии сайта представлен на рисунке \ref{design-main}.

\addimghere{images/diagrams/design-main.png}{0.8}{Макет стандратной страницы в десктопном формате}{design-main}

Как видно по макету на рисунке \ref{design-main}, центральную часть сайта занимает основной блок контента, при этом производится его центрирование и ограничение по ширине в 1200 пикселов. 
Данное ограничение определено тем, что большинство устройств обладают размером экрана сильно превышающим 1200 пикселов в ширину, из-за чего содержимое страницы либо будет растянуто слева-направо, либо будет сосредоточено у левой границы экрана.
Чтобы избежать этого широко применяется подход ограничения размера зоны контента.

Для мобильной версии сайта также необходимо сформировать свой макет.
Общий шаблон представления интерфейса для мобильной версии сайта представлен на рисунке \ref{design-main-mobile}.

\addimghere{images/diagrams/design-main-mobile.png}{0.4}{Макет стандратной страницы в десктопном формате}{design-main-mobile}

На шаблонах, представленных на рисунках \ref{design-main} и \ref{design-main-mobile} отображаются элементы, помеченные как <<Main header>>, <<Content>> и <<Text about main header and content of this page>>.
Содержимое этих частей макетов заполняется содержимым, соответствующим активному разделу.
При этом <<Content>> может принимать индивидуальные варианты контента -- таблицы, изображения, текстовые блоки и прочие HTML-элементы.

Например, страница <<Курсы>>, формируемая из таблицы <<Courses>> содержит повторяющийся по количеству доступных записей шаблон.
Внешний вид данной страницы отображен на рисунке \ref{result-courses}.

\addimghere{images/examples/result-courses.png}{0.8}{Страница <<Курсы>>, формируемая на основе содержимого таблицы <<Courses>>}{result-courses}

Данная страница динамически изменяет свое наполнение в зависимости от содержимого базы.
Такой подход избавляет администратора веб-ресурса вручную редактировать страницу каждого курса для достижения желаемого результата.

Как альтернативный подход можно привести в пример страницу <<Поступление>>, являющуюся статической.
Её содержимое жестко задано внутри HTML-файла и контролируется только им.
Необходимость замены части информации потребует редактирования исходного кода.

Внешний вид данной страницы отображен на рисунке \ref{result-enrollment}.

\addimghere{images/examples/result-enrollment.png}{0.8}{Страница <<Поступление>>}{result-enrollment}


\subsection{Проверка внесенных изменений с использованием Lighthouse}

После внесения изменений, перечисленных в главах 3 и 4 данной работы, а также исправления отмеченных первых отчетах недостатков сайта, были получены готовые исходные файлы веб-ресурса.
Одной из задач, среди поставленных в разделе <<Постановка задачи>> в главе 2, было улучшение результатов проверки с использованием Lighthouse.
Можно приступить к повторному анализу для проведения сравнения результатов.

На рисунке \ref{lighthouse-meson-mobile-result} результат проведения проверки разработанного ресурса на соответствие основным требованиям современной веб-разработки на мобильном устройстве.

\addimghere{images/lighthouse-meson-mobile-result.png}{0.6}{Результат проверки главной страницы разработанного веб-ресурса с помощью средств Lighthouse для мобильного устройства}{lighthouse-meson-mobile-result}

На рисунке \ref{lighthouse-meson-desktop-result} отображен результат проверки для десктопного устройства.

\addimghere{images/lighthouse-meson-desktop-result.png}{0.6}{Результат проверки главной страницы разработанного веб-ресурса с помощью средств Lighthouse для десктопного устройства}{lighthouse-meson-desktop-result}

Итоговые средние оценки исходного веб-ресурса -- 61,5 для мобильной версии проверки и 68 для десктопной соответственно.
Даже без проведения подсчетов заметно, что результаты всех показателей, кроме производительности, однозначно выросли.

Проведение подсчетов для разработанного веб-ресурса дает средние показатели 91,25 для мобильной версии и 93 для десктопной.
Можно сделать вывод о положительном влиянии переработанной версии на показатели проверки в Lighthouse.
Баллы, потерянные в категории производительности частично оправданы переходом на другой способ формирования самого состава веб-ресурса -- статичные HTML файлы против динамически заменяемых шаблонов страниц.

\clearpage
