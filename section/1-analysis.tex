% Глава 1 должна содержать: 
% выбор направления исследований или направления разработки; 
% описание, способы (методы) совершенствования технологического(их) процесса(ов); 
% аналитический обзор источников информации и нормативных данных
\section{Анализ предметной области разработки Web-ресурса}
\subsection{Обзор предметной области}
    Разработка веб-ресурса как предметная область - весьма современная и востребованная тема.
    На данный момент сложно представить компанию без веб-ресурса.
    При этом даже можно не указывать область деятельности компании -- все виды коммерческого взаимодействия подразумевают контакт с внешним миром через средства связи и одно из наиболее распространенных и востребованых -- веб-ресурс.
    
    Само понятие веб-ресурса не конретизирует форму представления, а лишь заключает в себе способ передачи информации целевой аудитории.
    Таким образом под такое определение может попасть как полноценное приложение, так и, например, одностраничный сайт с картинкой-заглушкой.

    % // TODO Обзор предметной области

\subsection{Современные технологий и подходы в разработке web-ресурсов}
    % // TODO Современные технологий и подходы в разработке web-ресурсов
    \subsubsection{Системы управления контентом}
        Широкое распространение в последние годы получает направление упрощенного формирования веб-ресурсов.
        Простота подхода заключается в используемых средствах для достижения финального результата.
        Одним из таких средств явлются системы управления контентом (Content Management System), заключающие в себе программное обеспечение, строящее процесс взаимодействия с пользователем по типу работы с конструктором. % ссылка на вики цмс
        Основные выгоды, получаемые при выборе данного подхода сосредоточены вокруг использования уже готовых решений и их кастомизации под конкретные нужды заказчика.
        Процесс доработки при этом зачастую является менее требовательным к уровню подготовки исполнителя, а потому становится проще и доступнее даже для людей, не владеющих навыками взаимодействия с кодом.\\
        За доступность в использовании приходится расплачиваться ограничениями, заложенными в саму платформу-конструктор.
        Зачастую эти ограничения выливаются в виде невозможности тонкой настройки отдельных компонентов системы под конкретные нужды пользователя.
        Эта проблема решается, обычно, несколькими путями - либо переходом на более кастомизируемый и подходящий под задачи конструктор, либо взаимодействием с разработчиком (зачастую на платной основе) с целью получения желаемых изменений в продукте.
        Второй вариант может проявляться в разных видах взаимодействия - единоразовое приобретение прав на самостоятельное внесение изменений, долгосрочный контракт с вендором.
        Еще одним из минусов такого подхода является вынужденная зависимость от предоставляемого ПО, а следовательно и от его разработчика.

        Но даже с учетом приведённых недостатков, данный вид формирования веб-ресурса развивается.
        Вот лишь некоторые из самых популярных представителей современных систем управления контентом по версии портала CMS magazine \cite{}:
        % // TODO  информация с вики про примеры + скрины
        \begin{itemize}
            \item Wordpress
            \item 1С-Битрикс
            \item Joomla
            \item Drupal
            \item Tilda Publishing
        \end{itemize}

    \subsubsection{Применяемые подходы к организации веб-ресурса}
        Альтернативным подходом к использованию систем управления контентом, является изначальная разработка всей системы.
        Данный подход вовсе не отрицает возможности использовать наемный труд со стороны, а лишь указывается, что программные средства будут разработаны под индивидуальные требования заказчика.

        С точки зрения подходов в работе веб-ресурсов можно выделить два направления:
        \begin{itemize}
            \item Многостраничное приложение или MPA - multi page application.
            Заключается в построении схемы обмена данными между клиентом и сервером, основанной на передаче цельных HTML-файлов.
            Данный способ приводит к перезагрузке страницы при каждом запросе;
            \item Одностраничное приложение или SPA - single page application.
            Заключается в ином подходе - подгрузке основной части приложения при инициализирующем запросе и дальнейшее наполнение данными при последующих запросах к серверу.
        \end{itemize}
        % // TODO схемы отправки-получения запросов для этих методов
        % // TODO + и - подходов

        Данные подходы уже хорошо себя зарекомендовали и применяются повсеместно.
        Но поверх них существуют и другие уровени реализации приложения - непосредственное воплощение в технологиях.

        Подходы в использовании определенного набора технологий в профессиональной среде называются "стек" (от английского stack - куча).
        Само название стека определяется его содержимым - набором инструментов, обычно отвечающим следующим задачам:
        \begin{itemize}
            \item Формирование клиентской части приложения - то что видит и с чем взаимодействует пользователь;
            \item Взаимодействие сторон клиента и сервера - построение каналов обработки сообщений между частями приложения;
            \item Хранение и взаимодействие с базой - собственно организация накопления, обработки и выдачи данных для использования приложения.
        \end{itemize}

        Например стек MEAN берёт своё название от имен применяемых технологий, решающих приведенные выше задачи:
        % // TODO можно подробнее по каждому пункту с ссылками
        \begin{itemize}
            \item MongoDb - СУБД;
            \item Express.js - построение маршрутизации в приложении;
            \item Angular.js - клиентская (front-end) часть приложения;
            \item Node.js - серверная (back-end) часть приложения.
        \end{itemize}
        
        И простым изменением одного из компонентов можно получить новый стек - MERN.
        В данном случае за клиентскую часть будет отвечать другая технология - React.js.

        Ещё одним важным этапом является выбор способа взаимодействия между клиентской и серверной частью.
        За данный пункт отвечают специальные интерфейсы прикладных программ - API (application programming interface), выполняющие роль посредника между частями приложения и определяющие сам способ передачи информации, её вид.
        % // TODO цитата на рест
        Наиболее популярным является архитектурный подход REST, основывающийся на протоколе HTTP для транспортировки данных.
        Например, в приведённом выше стеке MERN/MEAN подход REST реализуется на базе технологии Express.js.
        Более подробно о строении подхода REST будет рассказано в % вставить ссылку на часть диплома с разбором rest-a

        Альтернативным к REST подходом, является GraphQL, рассматривающий взаимодейтсвие и работу с данными через графы.
        Данный подход также будет подробнее рассмотрен позже.


    \subsubsection{Применяемые языки программирования}
        Разработка веб-ресурсов вот уже несколько десятилетий вбирает в себя самые разные технологии и подходы.
        В данном разделе будут приведены наиболее востребованные и распространенные языки программирования и их сочетания, применяемые в индустрии на 2021-2022 год.
        % // TODO подробности
        \begin{itemize}
            \item HTML
            \item CSS
            \item PHP
            \item Ruby
            \item Python
            \item Java
            \item JavaScript
        \end{itemize}
        
        В целом приведенный список можно разделить на две важные части - языки для front-end части и back-end.
        Здесь важно указать, что подразумевают два этих понятия.

        Front-end - часть веб-приложения, отвечающая в большей степени за визуальную составляющую - то, с чем взаимодействует пользователь.
        К чисто фронтовым технологиям можно отнести HTML и CSS - они отвечают именно за часть отображения и оформления контента.
        
        Back-end - часть веб-приложения, отвечающая за обработку системной и бизнес-логики для дальнейшей её передачи в некотором виде пользователю через фронтовую часть.
        К данной категории можно отнести языки, сосредотачивающие контроль над серверной стороной - Ruby, Java.

        Хоть приведенные примеры и несколько категоричны, важно указать, что границы между четким front и back приложения и относящимся к ним языкам программирования имеют тенденцию размываться, что приводит к неоднозначности, например, такого утверждения: "JavaScript - чисто front-end язык программирования", которое может быть верным и неверными в зависимости от множества технических факторов.

        Исходя из описанных выше данных, будут выбраны в предпочтение для реализации технологии, касающиеся языка JavaScript.

    \subsubsection{Технологии разработки для front-end}
        В данном пункте приведены различные реализации надстроек над основным языком, называемые фреймворками.
        Фреймворк — программная платформа, определяющая структуру программной системы; программное обеспечение, облегчающее разработку и объединение разных компонентов большого программного проекта [] % TODO ссылка на определение фреймврока с вики
        
        Сам фреймворк играет ключевую роль в способе представления информации и взаимодействия с ней.
        Поскольку фреймворк является лишь расширением, ему требуется основание для работы.
        Таким основанием может являться один из языков программирования, который ложится в основу проработки внутренней логики и строения приложения.
        Наиболее популярные фреймворки, расширяющие возможности JavaScript:
        % // TODO Технологии разработки для front-end
        \begin{itemize}
            \item react
            \item angular
            \item vue
        \end{itemize}
        Дополнительно стоит упомянуть возможность работы с "чистым" JavaScript - без применения фреймворков.
        Для такой ситуации можно применить подход описания систем управления контентом, представленные ранее в данной работе.
        Минусы и плюсы использовани фреймворков также заключаются в ограничениях свобод разработчика, его зависимости от используемой технологии.
        Есть компании, определяющие подход своей разработки с приложением через применение чистого JavaScript без дополнений, но, как правило вместе с таким подходом повышается уровень требований к программистам ввиду необходимости реализации множества компонентов приложения.

    \subsubsection{Технологии разработки для back-end}
        Некоторая уникальность JavaScript заключается в возможности его применения как для front-end части, так и для back-end.
        Данная возможность появилась вместе с созданием Райаном Далом технологии Node.js в 2009 году.

        Node.js - программная платформа, основанная на движке V8 (транслирующем JavaScript в машинный код), превращающая JavaScript из узкоспециализированного языка в язык общего назначения.
        Node.js добавляет возможность JavaScript взаимодействовать с устройствами ввода-вывода через свой API, написанный на C++, подключать другие внешние библиотеки, написанные на разных языках, обеспечивая вызовы к ним из JavaScript-кода.
        Node.js применяется преимущественно на сервере, выполняя роль веб-сервера, но есть возможность разрабатывать на Node.js и десктопные оконные приложения (при помощи NW.js, AppJS или Electron для Linux, Windows и macOS) и даже программировать микроконтроллеры (например, tessel, low.js и espruino).
        В основе Node.js лежит событийно-ориентированное и асинхронное (или реактивное) программирование с неблокирующим вводом/выводом. % ссылка на ноду \cite{wiki-nodejs}
        
        Сами технологии, основанные на Node.js многогранны, но наиболее востребованными для разработки веб-приложения являются:
        % // TODO Технологии разработки для back-end
        \begin{itemize}
            \item electron.js
            \item express.js
        \end{itemize}

    
    \subsubsection{Операционные системы}
        Пункт, связанный с операционными системами, на мой взгляд, также является достойным обсуждения.
        С точки зрения веб-разработки нет большой разницы в применяемой платформе - основной набор технологий будет работать идентично на любой ОС.
        % // TODO Операционные системы
        \begin{itemize}
            \item windows
            \item mac os
            \item linux
        \end{itemize}
    
        
    \subsubsection{Среды разработки и редакторы кода}
        Одной из важных частей разработки является непосредственное взаимодействие с кодом - его редактирование, написание, удаление, чтение.
        Эти процессы являются критическими для разработчика, поскольку являются 
        % // TODO Среды разработки и редакторы кода
        \begin{itemize}
            \item VS Code
            \item sublime
            \item atom
            \item VS
        \end{itemize}
    
    \subsubsection{Редакторы диаграмм}
        % // TODO Выбор редактора диаграмм
        Необходимы для составления диаграм взаимодействия
        Одним из наиболее популярных средств является универсальный язык моделирования UML \cite{wiki-UML}.
        Видов программного обеспечения, использующего данный язык моделирования много:
        \begin{itemize}
            \item Rational Rose
            \item Dia   http://dia-installer.de/
            \item PlantUML  https://plantuml.com/ru/
            Можно использовать для отображения объектов и сущностей формата JSON (т.к. исопльзуется ЯП javascript) - https://plantuml.com/ru/json

            Использование с латехом
            https://plantuml.com/ru/latex

        \end{itemize}

\subsection{Определение требований к web-ресурсам}
    % // TODO Определение требований к web-ресурсам
% TODO добавить ссылку, переработать текст

    Консо́рциум Всеми́рной паути́ны (англ. World Wide Web Consortium, W3C) — организация, разрабатывающая и внедряющая технологические стандарты для Всемирной паутины. Основателем и главой Консорциума является сэр Тимоти Джон Бернерс-Ли, автор множества разработок в области информационных технологий. По состоянию на 29 мая 2019 года Консорциум насчитывает 444 члена.

    W3C разрабатывает для Интернета единые принципы и стандарты (называемые «рекомендациями», англ. W3C Recommendations), которые затем внедряются производителями программ и оборудования. Таким образом достигается совместимость между программными продуктами и аппаратурой различных компаний, что делает Всемирную сеть более совершенной, универсальной и удобной.

    Миссия W3C: «Полностью раскрыть потенциал Всемирной паутины путём создания протоколов и принципов, гарантирующих долгосрочное развитие Сети».
    Более конкретная цель W3C — помочь компьютерным программам достичь способности к взаимодействию в Сети (т. н. «сетева́я интеропера́бельность», англ. Web interoperability). Применение единых стандартов в Сети — это ключевой шаг для достижения такого взаимодействия.

    Две другие важнейшие задачи Консорциума — обеспечить полную «интернационализа́цию Сети́» и сделать Сеть доступной для людей с ограниченными возможностями. Для решения первой задачи Консорциум активно сотрудничает с организацией «Юнико́д» (англ. Unicode) и рядом других рабочих групп, занимающихся международным сотрудничеством в Интернете и языковыми технологиями. Для решения второй задачи Консорциум не только сотрудничает с организациями соответствующего профиля, но и разработал свои собственные рекомендации, которые сейчас активно набирают популярность. Существует "Руководство по обеспечению доступности веб-контента (WCAG)", разработанное Консорциумом всемирной паутины (W3C). В нем четко расписаны все требования по контенту сайтов и его форматированию, чтобы максимум людей могли комфортно пользоваться информацией.
    
\subsection{Проверка соответствия существующей системы современным требованиям и технологиям}
    % // TODO Проверка соответствия существующей системы заявленным требованиям
    \subsubsection{Обзор версий оформления страниц сайта и их развития}
        % // TODO 
        https://web.archive.org/web/20120219134344/http://www.meson.ru/

\subsection{Постановка требований к разрабатываемому web-ресурсу}
    % // TODO Постановка требований к разрабатываемому web-ресурсу

\subsection{Постановки задачи}
    % // TODO Постановки задачи
    Разработка веб-приложения в соответствии с представленными ранее подходами и технологиями.

\clearpage
