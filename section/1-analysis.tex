\section{Анализ предметной области разработки Web-ресурса}
\subsection{Обзор предметной области}                                                           % // TODO Обзор предметной области

\subsection{Современные технологий и подходы в разработке web-ресурсов}                         % // TODO Современные технологий и подходы в разработке web-ресурсов
    \subsubsection{Системы управления контентом}                                                % // TODO Системы управления контентом
        https://www.nic.ru/info/blog/cms/
        \begin{itemize}
            \item wordpress
            \item 1С-Битрикс
            \item Joomla
            \item Drupal
        \end{itemize}

    \subsubsection{Технологии разработки для front-end}                                         % // TODO Технологии разработки для front-end
        \begin{itemize}
            \item javascript
            \item react
            \item angular
            \item vue
        \end{itemize}

    \subsubsection{Технологии разработки для back-end}                                          % // TODO Технологии разработки для back-end
        \begin{itemize}
            \item node.js
            \item electron.js
            \item express.js
            \item 
        \end{itemize}

    \subsubsection{Операционные системы}                                                        % // TODO Операционные системы
        \begin{itemize}
            \item windows
            \item mac os
            \item linux
        \end{itemize}
        
    \subsubsection{Среды разработки и редакторы кода}                                             % // TODO Среды разработки и редакторы кода
        \begin{itemize}
            \item VS Code
            \item sublime
            \item atom
            \item VS
        \end{itemize}
    
    \subsubsection{Редакторы диаграмм}                                                % // TODO Выбор редактора диаграмм
        Необходимы для составления диаграм взаимодействия
        Одним из наиболее популярных средств является универсальный язык моделирования UML \cite{wiki-UML}.
        Видов программного обеспечения, использующего данный язык моделирования много:
        \begin{itemize}
            \item Rational Rose
            \item Dia   http://dia-installer.de/
            \item PlantUML  https://plantuml.com/ru/
            Можно использовать для отображения объектов и сущностей формата JSON (т.к. исопльзуется ЯП javascript) - https://plantuml.com/ru/json

            Использование с латехом
            https://plantuml.com/ru/latex

        \end{itemize}

\subsection{Определение требований к web-ресурсам}
    % // TODO Определение требований к web-ресурсам
    
\subsection{Проверка соответствия существующей системы современным требованиям и технологиям}   % // TODO Проверка соответствия существующей системы заявленным требованиям
    \subsubsection{Обзор версиq оформления страниц сайта и их развития}                          % // TODO 
        https://web.archive.org/web/20120219134344/http://www.meson.ru/

\subsection{Постановка требований к разрабатываемому web-ресурсу}                               % // TODO Постановка требований к разрабатываемому web-ресурсу

\subsection{Постановки задачи}                                                                  % // TODO Постановки задачи

\clearpage
