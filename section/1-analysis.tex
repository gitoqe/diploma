% Глава 1 должна содержать: 
% выбор направления исследований или направления разработки; 
% описание, способы (методы) совершенствования технологического(их) процесса(ов); 
% аналитический обзор источников информации и нормативных данных
\section{Анализ предметной области разработки Web-ресурса}
\subsection{Обзор предметной области}
    Разработка веб-ресурса как предметная область - весьма современная и востребованная тема.
    На данный момент сложно представить компанию без веб-ресурса.
    При этом даже можно не указывать область деятельности компании -- все виды коммерческого взаимодействия подразумевают контакт с внешним миром через средства связи и одно из наиболее распространенных и востребованых -- веб-ресурс.
    
    Само понятие веб-ресурса не конретизирует форму представления, а лишь заключает в себе способ передачи информации целевой аудитории.
    Таким образом под такое определение может попасть как полноценное приложение, так и, например, одностраничный сайт с картинкой-заглушкой.

    % // TODO Обзор предметной области

\subsection{Современные технологий и подходы в разработке web-ресурсов}
    % // TODO Современные технологий и подходы в разработке web-ресурсов
    \subsubsection{Системы управления контентом}
        Широкое распространение в последние годы получает направление упрощенного формирования веб-ресурсов.
        Простота подхода заключается в используемых средствах для достижения финального результата.
        Одним из таких средств явлются системы управления контентом (Content Management System \cite{wiki-CMS}), заключающие в себе программное обеспечение, строящее процесс взаимодействия с пользователем по типу работы с конструктором.
        Основные выгоды, получаемые при выборе данного подхода сосредоточены вокруг использования уже готовых решений и их кастомизации под конкретные нужды заказчика.
        Процесс доработки при этом зачастую является менее требовательным к уровню подготовки исполнителя, а потому становится проще и доступнее даже для людей, не владеющих навыками взаимодействия с кодом.

        За доступность в использовании приходится расплачиваться ограничениями, заложенными в саму платформу-конструктор.
        Зачастую эти ограничения выливаются в виде невозможности тонкой настройки отдельных компонентов системы под конкретные нужды пользователя.
        Эта проблема решается, обычно, несколькими путями - либо переходом на более кастомизируемый и подходящий под задачи конструктор, либо взаимодействием с разработчиком (зачастую на платной основе) с целью получения желаемых изменений в продукте.
        Второй вариант может проявляться в разных видах взаимодействия - единоразовое приобретение прав на самостоятельное внесение изменений, долгосрочный контракт с вендором.
        Еще одним из минусов такого подхода является вынужденная зависимость от предоставляемого ПО, а следовательно и от его разработчика.

        Но даже с учетом приведённых недостатков, данный вид формирования веб-ресурса развивается.
        Вот лишь некоторые из самых популярных представителей современных систем управления контентом по версии портала CMS magazine \cite{}:
        % // TODO  информация с вики про примеры + скрины
        \begin{itemize}
            \item Wordpress
            \item 1С-Битрикс
            \item Joomla
            \item Drupal
            \item Tilda Publishing
        \end{itemize}

    \subsubsection{Применяемые подходы к организации веб-ресурса}
        Альтернативным подходом к использованию систем управления контентом, является изначальная разработка всей системы.
        Данный подход вовсе не отрицает возможности использовать наемный труд со стороны, а лишь указывается, что программные средства будут разработаны под индивидуальные требования заказчика.

        С точки зрения подходов в работе веб-ресурсов можно выделить два направления:
        \begin{itemize}
            \item Многостраничное приложение или MPA - multi page application.
            Заключается в построении схемы обмена данными между клиентом и сервером, основанной на передаче цельных HTML-файлов.
            Данный способ приводит к перезагрузке страницы при каждом запросе;
            \item Одностраничное приложение или SPA - single page application.
            Заключается в ином подходе - подгрузке основной части приложения при инициализирующем запросе и дальнейшее наполнение данными при последующих запросах к серверу.
        \end{itemize}
        % // TODO схемы отправки-получения запросов для этих методов
        % // TODO + и - подходов

        Данные подходы уже хорошо себя зарекомендовали и пприменяются повсеместно.
        Но поверх них существует следующий уровень реализации приложения - непосредственное воплощение в технологиях.
        И одним из важных выборов при разработке веб-ресурса является определение используемого набора языков программирования.

    \subsubsection{Применяемые языки программирования}
        Разработка веб-ресурсов вот уже несколько десятилетий вбирает в себя самые разные технологии и подходы.
        В данном разделе будут приведены наиболее востребованные и распространенные языки программирования и их сочетания, применяемые в индустрии на 2021-2022 год.
        % // TODO подробности
        \begin{itemize}
            \item HTML
            \item CSS
            \item PHP
            \item Ruby
            \item Python
            \item Java
            \item JavaScript
        \end{itemize}
        
        В целом приведенный список можно разделить на две важные части - языки для front-end части и back-end.
        Здесь важно указать, что подразумевают два этих понятия.

        Front-end - часть веб-приложения, отвечающая в большей степени за визуальную составляющую - то, с чем взаимодействует пользователь.
        К чисто фронтовым технологиям можно отнести HTML и CSS - они отвечают именно за часть отображения и оформления контента.
        
        Back-end - часть веб-приложения, отвечающая за обработку системной и бизнес-логики для дальнейшей её передачи в некотором виде пользователю через фронтовую часть.
        К данной категории можно отнести языки, сосредотачивающие контроль над серверной стороной - Ruby, Java.

        Хоть приведенные примеры и несколько категоричны, важно указать, что границы между четким front и back приложения и относящимся к ним языкам программирования имеют тенденцию размываться, что приводит к неоднозначности, например, такого утверждения: "JavaScript - чисто front-end язык программирования", которое может быть верным и неверными в зависимости от множества технических факторов.


    \subsubsection{Технологии разработки для front-end}
        % // TODO Технологии разработки для front-end
        \begin{itemize}
            \item javascript
            \item react
            \item angular
            \item vue
        \end{itemize}

    \subsubsection{Технологии разработки для back-end}
        % // TODO Технологии разработки для back-end
        \begin{itemize}
            \item node.js
            \item electron.js
            \item express.js
            \item 
        \end{itemize}

    \subsubsection{Операционные системы}
        % // TODO Операционные системы
        \begin{itemize}
            \item windows
            \item mac os
            \item linux
        \end{itemize}
        
    \subsubsection{Среды разработки и редакторы кода}
        % // TODO Среды разработки и редакторы кода
        \begin{itemize}
            \item VS Code
            \item sublime
            \item atom
            \item VS
        \end{itemize}
    
    \subsubsection{Редакторы диаграмм}
        % // TODO Выбор редактора диаграмм
        Необходимы для составления диаграм взаимодействия
        Одним из наиболее популярных средств является универсальный язык моделирования UML \cite{wiki-UML}.
        Видов программного обеспечения, использующего данный язык моделирования много:
        \begin{itemize}
            \item Rational Rose
            \item Dia   http://dia-installer.de/
            \item PlantUML  https://plantuml.com/ru/
            Можно использовать для отображения объектов и сущностей формата JSON (т.к. исопльзуется ЯП javascript) - https://plantuml.com/ru/json

            Использование с латехом
            https://plantuml.com/ru/latex

        \end{itemize}

\subsection{Определение требований к web-ресурсам}
    % // TODO Определение требований к web-ресурсам
    
\subsection{Проверка соответствия существующей системы современным требованиям и технологиям}
    % // TODO Проверка соответствия существующей системы заявленным требованиям
    \subsubsection{Обзор версиq оформления страниц сайта и их развития}
        % // TODO 
        https://web.archive.org/web/20120219134344/http://www.meson.ru/

\subsection{Постановка требований к разрабатываемому web-ресурсу}
    % // TODO Постановка требований к разрабатываемому web-ресурсу

\subsection{Постановки задачи}
    % // TODO Постановки задачи

\clearpage
