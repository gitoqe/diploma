\section{Aналитический обзор информационного обеспечения учебных центров}
\subsection{ИТ-управление учебным центром: цели, задачи и функции}
По части вопроса управления учебным центром можно выделить несколько пунктов, задействующих в себе полностью или частично отрасль информационных технологий и представляющий интерес в данной работе.
Итак, соответствующие темам вопросы:
\begin{itemize}
    \item Управленческие - от поддержания внутренней базы кадров до бухгалтерского учета;
    \item Технической поддержки - хранение информации об используемой технике, ПО и его версиях, параметрах настройки, а также распределении по кабинетам/аудиториям;
    \item Информационно-оповещательные - распространение информации о деятельности учебного центра и способах взаимодействия с ним;
    \item Внутреннего учета - обеспечение возможности получения, обработки и хранения данных о клиентах учебного центра (студенты, ответственные за них лица);
    \item Проверки выполнения заданий - предоставление функционала по сбору, выдаче и контролю за правильностью выполнения поставленных студентам задач;
    \item Контрольная - проверка доступов и способов взаимодейтсвия внутри и между систем.
\end{itemize}

Данный список содержит пункты, которые могут пересекаться, быть вложенными друг в друга или взаимодействовать иным образом в зависимости от конкретных задач, поставленных перед учебным центром, специфике его целевой аудитории и прочих факторов.

Исходя из приведенных выше вопросов уже можно сделать вывод об индивидуальности подхода в проектировании и использовании ИТ-инфраструктуры.
Теперь можно перейти к рассмотрению основной задачи данного раздела - определению ключевых требований к ИТ-управлению учебным центром.

Цели

Наверное, самым простым вопросом относительно применения ИТ в управлении учебным центром является постановка цели.
В данном случае лучшей формулировкой будет - оптимизация процессов взаимодействия сотрудников и технического обеспечения учреждения для получения наиболее продуктивного подхода в организации учебного процесса и, как результат, максимизация прибыли.

Задачи

В этом пункте надо понимать постановку вопроса и конкретные требования от заказчика - непосредственно организации.
Но, тем не менее, можно сформулировать общие пункты-задачи для ИТ-инфраструктуры учебного центра:
\begin{itemize}
    \item Оптимизация способов взаимодействия между действующими лицами относительно услуг учебного центра (клиент - преподаватель, преподаватель - студент, студент - учебная система и т.д.);
    \item Предоставление информации о деятельности учебного центра (состав учебных программ, наличие свободных мест в группах, данные об учебном процессе и прочее);
    \item Формирование технической возможности автоматизированной работы студента с системой учреждения (проверка, выдача задач, ведение статистики);
    \item Обеспечение документооборота внутри организации (бухгалтерия, данные сотрудников и учеников);
    \item Ведение учета посещаемости и оценок за выполнение задач.
\end{itemize}

И снова считаю важным указать, что данный набор пунктов не является абсолютным и напрямую зависит от требований заказчика ИТ-системы.

Функции

С данным пунктом проще всего - он должен содержать выжимку из приведенных выше, оформленную в четко сформулированные понятия.
Итак, используя разобранные ранее вопросы можно выделить следующие функции ИТ-управления учебным центром:
\begin{itemize}
    \item Цифровизационная - частичный или полный перенос бумажного документооборота в цифровой формат;
    \item Оптимизационная - модернизация имеющихся процессов, исключение лишних, преобразование неоптимальных;
    \item Социальная - расширение охвата целевой аудитории учреждения, привлечение новых клиентов>;
    \item Управленческая - предоставление контроля над элементами системы учебного центра;
    \item Статистическая - сбор и обработка статистики о процессах работы учреждения;
    \item Автоматизационная - перенос части вручную выполняемых задач на автоматически исполняемые.
\end{itemize}

Частичное или полное соответствие разрабатываемой ИТ-системы приведенным функциям может свидетельствовать о ее пригодности применения в деятельности учебного центра.

\subsection{Типы сайтов и обзор средств разработки сайтов}

Относительно разделения сайтов на категории существует множество подходов, использующих разные способы выделения ключевых качеств для определенного типа.

Наиболее распространенными способами диффиренциации сайтов являются следующие:
\begin{itemize}
	\item По контенту - какую информацию предоставляет сайт;
	\item По способу предоставления контента - каким образом данные передаются конечному пользователю;
	\item По универсальности отображения контента - адаптирование дизайна элементов сайта под изменямые параметры устройства пользователя;
	\item По функциональности - изначально решаемой сайтом задачей;
\end{itemize}

Исходя из таких категорий можно более подробно рассмотреть каждую из них.

Распределение сайтов по типу контента.

Весьма очевидно, что в зависимости от сферы деятельности организации, будут меняться и данные, которые сайт должен предоставлять пользователю-клиенту.
Например, профессиональное агентство "Студия Вячеслава Денисова" \cite{denisov} выделяет следующие категории:
\begin{itemize}
	\item Блоги
    
    Блоги - регулярно обновляемые страницы, привязанные к определенной личности, либо компании.
	Тематика и способ подачи информации блога могут быть совершенно разнообразными.
    \addimghere{LJ.png}{0.6}{Внешний вид сервиса LiveJournal}{LiveJournal}
	Например, некогда очень популярный сайт livejournal.com \cite{livejournal} (Рисунок \ref{LiveJournal}) является платформой для создания и ведения блогов.

	\item Корпоративные сайты

	Все больше компаний осознают, что у них должен быть хотя бы базовый сайт, чтобы они выглядели заслуживающими доверия и профессиональными. Компании могут не продавать напрямую через корпоративные веб-сайты, но они могут использовать их с целью предоставления информации о себе и для поддержки клиентов.

	\item Краудфандинговые платформы

	В прошлом финансирование нового делового предприятия или проекта предусматривало поиск крупных сумм денег у ограниченного круга состоятельных граждан — спонсоров.

	Краудфандинг — новая практика финансирования проектов или предприятий путем привлечения небольших сумм у множества людей. Она включает создание привлекательного видео и описание преимуществ проекта, постановку целей в надежде достичь ее к установленному сроку (сумма необходимых средств).

	Заинтересованные люди пожертвуют небольшие суммы, чтобы стать частью проекта и воспользоваться определенными бонусами. Краудфандинговые сайты становятся очень популярным ресурсом для финансирования новых стартапов.

    Пример такого сайта - kickstarter.com \cite{kickstarter} (Рисунок \ref{kickstarter}).
    \addimghere{kickstarter.png}{0.6}{Внешний вид сервиса kickstarter}{kickstarter}

	\item Сайты e-commerce

	Сайт электронной коммерции может пересекаться с блогом или корпоративным сайтом, однако в конечном итоге его целью является продажа продуктов и услуг через Интернет.

	Сайт, который является исключительно корпоративным и не имеет функций продажи, все еще побуждает пользователей покупать продукт или услугу. Вся разница в том, что они не могут совершить сделку через сам сайт.

    Пример такого сайта - ozon.ru \cite{ozon} (Рисунок \ref{ozon}).
    \addimghere{ozon.png}{0.6}{Внешний вид сервиса ozon}{ozon}

	\item Образовательные ресурсы

	«Какие бывают виды веб-сайтов?», «Как сварить яйцо?»… очень вероятно, что при вводе в поисковую систему таких запросов вы натолкнетесь на информационные (образовательные) сайты. Их цель — предоставить пользователю сведения, которые он ищет.

    Пример такого сайта - stepik.org \cite{stepik} (Рисунок \ref{stepik}).
    \addimghere{stepik.png}{0.6}{Внешний вид сервиса stepik}{stepik}

	\item Новостные сайты

	Сайты новостей и журналов нуждаются в небольшом объяснении.

	Основная цель новостного сайта — держать читателей в курсе текущих событий. То же самое можно сказать о сайте онлайн-журнала, где в большей степени внимание уделяется развлечениям. Если вы хотите открыть газету или онлайн-журнал, взгляните на функциональные возможности CMS «1С-Битрикс: Управление сайтом».

	\item Социальные медиа

	Сайты социальных сетей уникальны как по функциональности, так и по контенту.

	Социальные медиа созданы как место для обмена мыслями, изображениями и видео, и все чаще становятся популярным местом для чтения свежих новостей и даже для обучения.

	\item Сайты ТВ или потокового видео

	В последние годы популярность сайтов потокового видео резко пошла вверх.

	Netflix и подобные сайты произвели революцию в самом способе просмотра телевизионных программ по всему миру. Сотни тысяч часов кино и телеэфира, высокое качество просмотра и относительно небольшая плата — этот вид сайтов расширяет горизонты ТВ.
\end{itemize}

Разновидности сайтов по способу предоставления контента согласно статьи того же профессионального агентства "Студия Вячеслава Денисова" \cite{denisov}:
\begin{itemize}
    \item Статические сайты

    Статические веб-сайты — самые простые, они появились в Интернете первыми.
    Их содержимое не зависит от действий пользователя и обновляется относительно редко, только при участии контент-менеджера.
    Статические веб-сайты создаются с использованием простого HTML-кода и выполняют только информативную функцию.
    
    \item Динамические сайты
    
    Динамический сайт или веб-страница позволяет отображать разный контент при каждом посещении.
    В эту категорию входят блоги, сайты электронной коммерции и вообще любой сайт, который регулярно обновляется с участием пользователей.
    
    Динамические веб-сайты также могут быть настроены для демонстрации разного контента разным пользователям, с учетом определенного времени дня и др.
    Динамические веб-сайты обеспечивают более личный и интерактивный интерфейс для пользователя, хотя их разработка сложней и дороже, чем статических сайтов.
\end{itemize}

Различия в сложности разработки и поддержки статических и динамических сайтов лежит также в используемых для этого технологиях - как правило их набор, требования к разработчикам и количество единиц используемого ПО сильно вырастает при уходе от статических сайтов.
Более подробное рассмотрение технических средств, применяемых для работы с динамическими веб-страницами будет приведет в последующих главах данной работы.

Тематика контента, а также способ его предоставления играют важную роль, но за последние десятилетия вслед за развитием техники, обретала большую значимость и возможность пользоваться знакомыми ресурсами в привычном их виде, но с применением разных устройств.
Естественным спутником распространения смартфонов явилась и необходимость позволить пользователям взаимодействовать со знакомыми приложениями и веб-ресурсами с экрана небольшого устройства с диагональю экрана в несколько дюймов.

Данная задача была решена с использованием разных технологий и подходов, а общее название ей "отзывчивость дизайна".
Разделение веб-сайтов по данному критерию также приведено в статье "Студия Вячеслава Денисова" \cite{denisov}:
\begin{itemize}
    \item Статические макеты

    Статические, или фиксированные макеты сайта плохо оптимизированы для экрана, отличающегося от стандартного десктопного ПК. Они имеют фиксированную ширину в пикселях, потому при открытии подобной веб-страницы на мобильном телефоне пользователям придется увеличивать каждый фрагмент для прочтения.
    
    С одной стороны, статические сайты могут загружаться немного быстрее из-за их простоты. Однако они категорически не рекомендуются в эпоху мобильного Интернета из-за ужасного пользовательского опыта для владельцев мобильных телефонов и планшетов.
    
    Более 50\% всех поисковых запросов в настоящее время осуществляется на мобильных устройствах, поэтому mobile friendly – важнейший фактор при разработке сайта.
    
    \item Жидкие или подвижные макеты
    
    Веб-сайт, созданный с использованием жидкого дизайна, гарантирует, что страница будет отображаться одинаково с точки зрения пропорций, независимо от размера вашего экрана. Каждый элемент сайта должен занимать одинаковое место на экране благодаря использованию процентов вместо пикселей. Например, 5\% от ширины.
    
    \item Адаптивные макеты
    
    При так называемом адаптивном дизайне макет определяет ширину экрана и соответствующим образом приспосабливается к ней с помощью медиа-запросов.
    
    Изначально в адаптивный макет закладываются блоки фиксированного размера, но количество таких фиксированных размеров достаточно для любых экранов. Медиа-запрос CSS определяет ширину браузера и автоматически выбирает оптимальный для нее размер.
    
    Такое решение идеально подходит для устаревших статических сайтов, которые требуется дешево и сердито модернизировать для поддержки мобильных устройств. Проблема в том, что отсутствие определенных размеров блоков может приводить к неправильному отображению веб-страниц на новых устройствах.
    
    \item Отзывчивые макеты
    
    Преимущество отзывчивого сайта в том, что здесь одновременно используются медиа-запросы CSS о ширине браузера и относительные величины (как в подвижном макете).
    
    В результате становится возможным плавное изменение и корректное отображение информации. Это наиболее совершенный подход в эпоху мобильного Интернета.
\end{itemize}

Исходя из приведенных пунктов можно выделить наименее подходящий для текущего времени подход - использование статических макетов.
Он также актуален для небольших проектов - сайтов-визиток, либо простых ознакомительных веб-сервисов, но его серьезное применение отходит на второй план при необходимости использовать все возможности современных подходов в ИТ.

Из способов разделения сайтов на категории остался подход на основе функциональности.
В данном случае категории перекликаются с распределением по типу контента, так как являются тесно связанными и влияющими друг на друга.


\begin{itemize}
    \item Сайт-визитка

    Считается, что визитка — это самый доступный и технически простой веб-сайт.
    Подобный проект имеет несколько страниц и может использоваться малыми предприятиями, которым нужно просто утвердить свое присутствие в Интернете.
    
    Например, небольшая компания по ремонту сантехники нуждается в визитке с домашней страницей с контактной информацией, страницей «О нас» и, возможно, парой фотографий примеров работы.
    Отсюда и название — проект очень похож на рекламный буклет, высокотехнологичную онлайн-визитку для потенциальных клиентов.
    
    \item Сайт электронной коммерции
    
    Веб-сайт электронной коммерции — это более сложный инструмент, с помощью которого пользователи могут оплачивать товары и услуги в Интернете.
    Сюда относят всевозможные интернет-магазины, аукционы и тому подобное.
    
    Обычно сайт e-commerce предполагает реализацию товаров или услуг одной компании нескольким пользователям, но коммерческий сайт может работать сразу с несколькими поставщиками, приобретая форму торговой площадки.
    
    Торговые площадки позволяют нескольким поставщикам искать покупателей и реализовывать товары через один сайт.
    
    \item Интернет-портал
    
    Особенность интернет-порталов заключается в том, что они объединяют информацию из множества разных источников.
    
    Закрытые порталы могут быть предназначены для внутреннего использования в школе, университете или компании, что позволяет учащимся и сотрудникам получать доступ к своим электронным письмам, оповещениям и файлам на одном сайте.
    
    \item Вики-сайт
    
    Вики-сайт — это особый проект, который позволяет людям совместно работать онлайн и писать контент вместе. 
    Самым популярным примером является сама Википедия \cite{wiki} (Рисунок \ref{wiki}), которая позволяет любому изменять, добавлять и оценивать содержание статей.
    .
    \addimghere{wiki.png}{0.6}{Внешний вид сервиса Википедия}{wiki}
    
    \item Социальная платформа
    
    Сайты социальных сетей — это масштабные, дорогостоящие и сложные проекты, которые позволяют обмениваться изображениями, видео или идеями. 
    Они поощряют интерактивное взаимодействие и обмен информацией между рядовыми пользователями.
    
    На территории РФ наиболее выразительным примером является социальная сеть Вконтакте \cite{vk} (Рисунок \ref{vk}).
    \addimghere{vk.png}{0.6}{Внешний вид сервиса ВКонтакте}{vk}
\end{itemize}

Также весьма важным разделением веб-сайтов на категории можно указать сам способ их создания.
Можно выделить два больших направления:
\begin{itemize}
    \item Классический подход - формирование функционального, интерфейсного и контентного наполнения веб-сайта производится вручную с использованием средств автоматизации.
    Данный вариант характеризуется большей трудоемкостью и повышенными требованиями к навыкам программирования, администрирования и дизайна ПО;
    \item Подход с использование систем управления контентом - применение ПО, предназначенного для упрощения процесса создания, администрирования и оформления веб-приложения.
    В данном случае сам процесс менее требователен к возможностям заказчика, но при этом зачастую теряется возможность производить тонкую доработку исходного функционала продукта.
\end{itemize}
Данные категории будут более подробно рассмотрены в дальнейших частях данной работы.

На этом можно обзор подходов к выделению особенностей веб-сайтов можно завершить, упомянув, что в зависимости от конкретных требований количество категорий может сильно изменяться.

\subsection{Анализ систем управления контентом}

    Широкое распространение в последние годы получает направление упрощенного формирования веб-ресурсов.
    Простота подхода заключается в функциональных возможностях используемых технологий для достижения финального результата.

    Одним из таких средств явлются системы управления контентом (Content Management System), заключающие в себе программное обеспечение, строящее процесс взаимодействия с пользователем по типу работы с конструктором \cite{wiki-CMS}.

    Основные выгоды, получаемые при выборе данного подхода сосредоточены вокруг использования уже готовых решений и их кастомизации под конкретные нужды заказчика.
    Процесс доработки при этом зачастую является менее требовательным к уровню подготовки исполнителя, а потому становится проще и доступнее даже для людей, не владеющих навыками взаимодействия с кодом.
    
    За доступность в использовании приходится расплачиваться ограничениями, заложенными в саму платформу-конструктор.
    Зачастую эти ограничения выливаются в виде невозможности тонкой настройки отдельных компонентов системы под конкретные нужды пользователя.
    Эта проблема решается, обычно, несколькими путями - либо переходом на более кастомизируемый и подходящий под задачи конструктор, либо взаимодействием с разработчиком (зачастую на платной основе) с целью получения желаемых изменений в продукте.

    Второй вариант решения может проявляться в разных видах взаимодействия - единоразовое приобретение прав на самостоятельное внесение изменений, либо долгосрочный контракт с представителем-разработчиком исходной CMS.
    Как следствие, еще одним из минусов такого подхода является вынужденная зависимость от предоставляемого ПО, а следовательно и от его разработчика.

    Но даже с учетом приведённых недостатков, данный вид формирования веб-ресурса развивается.
    Вот лишь некоторые из самых популярных представителей современных систем управления контентом по версии портала CMS magazine \cite{cmsmagazine}:
    \begin{itemize}
        \item Wordpress
        Wordpress - свободно распространяемая система управления содержимым сайта с открытым исходным кодом; написана на PHP; сервер базы данных — MySQL; выпущена под лицензией GNU GPL версии 2.
        Сфера применения — от блогов до достаточно сложных новостных ресурсов.
        Встроенная система «тем» и «плагинов» вместе с удачной архитектурой позволяет конструировать проекты широкой функциональной сложности \cite{wiki-WordPress}.

        \item 1С-Битрикс
        1С-Битрикс: Управление сайтов — система управления контентом веб-проекта (CMS) от российской компании «1С-Битрикс» \cite{wiki-1cb}.

        \item Joomla
        Joomla! — система управления содержимым (CMS), написанная на языках PHP и JavaScript, использующая в качестве хранилища базы данных СУБД MySQL или другие стандартные промышленные реляционные СУБД.
        Является свободным программным обеспечением, распространяемым под лицензией GNU GPL \cite{wiki-joomla}.

        \item Drupal
        Drupal — система управления содержимым (CMS), используемая также как каркас для веб-приложений (CMF), написанная на языке PHP и использующая в качестве хранилища данных реляционную базу данных.
        Drupal является свободным программным обеспечением, защищённым лицензией GPL, и развивается усилиями энтузиастов со всего мира \cite{wiki-drupal}.

        \item Tilda Publishing
        Tilda Publishing (сокр. Tilda) — блочный конструктор сайтов, не требующий навыков программирования.
        Позволяет создавать сайты, интернет-магазины, посадочные страницы, блоги и email-рассылки \cite{wiki-tilda}.
    \end{itemize}

    В целом CMS могут обладать совершенно разными характеристиками и возможностями в зависимости от компании-разработчика. Например одним из самых простых интерфейсов обладает CMS от Tilda Publishing (интерфейс настроек отображен на Рисунке \ref{tilda}).

    \addimghere{tilda.png}{0.6}{Пример интерфейса настроек сайта в Tilda Publishing}{tilda}

    В данном сервисе заложен большой объем функционала - начиная от выбора заготовленных шаблонов (отображен на рисунке \ref{tilda-choose}) и до настройки конкретных компонентов страницы (отображено на Рисунке \ref{tilda-sett}).

    \addimghere{tilda-choose.png}{0.6}{Страница выбора шаблона сайта в Tilda Publishing}{tilda-choose}

    \addimghere{tilda-sett.png}{0.6}{Страница выбора и настройки компонента сайта в Tilda Publishing}{tilda-sett}

    Применение CMS может сильно ускорить процесс перехода проекта со стадии идеи к первоначальной реализации.
    В этом, наверное заключается наибольшая выгода для компаний.
    Что же касается частных пользователей CMS - они получают простой инструмент, позволяющий без подробного изучения программирования и посвящения этому большого количества времени, дать необходимый результат - собственный сайт, пусть и на платформе CMS.

\subsection{Паттерны и фреймворки}
    Альтернативным подходом к использованию систем управления контентом, является изначальная поэтапная разработка всей системы.
    Данный подход вовсе не отрицает возможности использовать труд сторонних разработчиков или результатов их деятельности, а лишь возводит индивидуальные требования заказчика как первоначальные.

    С точки зрения подходов в работе веб-ресурсов можно выделить два направления:
    \begin{itemize}
        \item Многостраничное приложение или MPA - multi page application.
        Заключается в построении схемы обмена данными между клиентом и сервером, основанной на передаче цельных HTML-файлов.
        Данный способ приводит к перезагрузке страницы при каждом запросе;
        \item Одностраничное приложение или SPA - single page application.
        Заключается в ином подходе - подгрузке основной части приложения при инициализирующем запросе и дальнейшее наполнение данными при последующих запросах к серверу.
    \end{itemize}

    Сравнение данных подходов представлено на Рисунке \ref{spa-mpa}.
    Как видно на изображении, передача цельных HTML-документов в MPA-подходе приводит к полной перезагрузке документа, в то время как SPA-подход задействует получение информации c помощью более легковесных форматов (например, JSON).
    \addimghere{spa-mpa.png}{0.6}{Схема сравнения жизненных циклов в подходе SPA и MPA}{spa-mpa}

    Данные подходы уже хорошо себя зарекомендовали и применяются повсеместно.
    Но поверх них существуют и другие уровени реализации приложения - непосредственное воплощение в технологиях.

    Подходы в использовании определенного набора технологий в профессиональной среде называются "стек" (от английского stack - куча, кипа).
    Само название стека определяется его содержимым - набором инструментов, обычно отвечающим следующим задачам:
    \begin{itemize}
        \item Формирование клиентской части приложения - то что видит и с чем взаимодействует пользователь;
        \item Взаимодействие сторон клиента и сервера - построение каналов обработки сообщений между частями приложения;
        \item Хранение и взаимодействие с базой - собственно организация накопления, обработки и выдачи данных для использования приложения.
    \end{itemize}

    Например стек MEAN берёт своё название от имен применяемых технологий, решающих приведенные выше задачи:
    \begin{itemize}
        \item MongoDb - СУБД
        MongoDB — документоориентированная система управления базами данных, не требующая описания схемы таблиц.
        Считается одним из классических примеров NoSQL-систем, использует JSON-подобные документы и схему базы данных.
        Написана на языке C++.
        Применяется в веб-разработке, в частности, в рамках JavaScript-ориентированного стека MEAN \cite{wiki-MongoDB}.
        
        \item Express.js - построение маршрутизации в приложении
        Express.js, или просто Express, фреймворк web-приложений для Node.js, реализованный как свободное и открытое программное обеспечение под лицензией MIT.
        Он спроектирован для создания веб-приложений и API.
        Де-факто является стандартным каркасом для Node.js \cite{wiki-Express.js}.

        \item Angular - клиентская (front-end) часть приложения
        Angular (версия 2 и выше) — открытая и свободная платформа для разработки веб-приложений, написанная на языке TypeScript, разрабатываемая командой из компании Google, а также сообществом разработчиков из различных компаний.
        Angular — полностью переписанный фреймворк от той же команды, которая написала AngularJS \cite{wiki-angular.js}.

        \item Node.js - серверная (back-end) часть приложения
        Node или Node.js — программная платформа, основанная на движке V8 (транслирующем JavaScript в машинный код), превращающая JavaScript из узкоспециализированного языка в язык общего назначения.
        Node.js добавляет возможность JavaScript взаимодействовать с устройствами ввода-вывода через свой API, написанный на C++, подключать другие внешние библиотеки, написанные на разных языках, обеспечивая вызовы к ним из JavaScript-кода \cite{wiki-nodejs}.
    \end{itemize}
    
    Простым изменением одного из компонентов можно получить новый стек - MERN.
    В данном случае за клиентскую часть будет отвечать другая технология - React.js.
    React (иногда React.js или ReactJS) — JavaScript-библиотека с открытым исходным кодом для разработки пользовательских интерфейсов \cite{wiki-react}.

    Ещё одним важным этапом является выбор способа взаимодействия между клиентской и серверной частью.
    За данный пункт отвечают специальные интерфейсы прикладных программ - API (application programming interface), выполняющие роль посредника между частями приложения и определяющие сам способ передачи информации, её вид.

    Наиболее популярным является архитектурный подход REST, основывающийся на протоколе HTTP для транспортировки данных.
    Например, в приведённом выше стеке MERN/MEAN подход REST реализуется на базе технологии Express.js.
    
    REST — архитектурный стиль взаимодействия компонентов распределённого приложения в сети.
    Другими словами, REST — это набор правил того, как программисту организовать написание кода серверного приложения, чтобы все системы легко обменивались данными и приложение можно было масштабировать.
    REST представляет собой согласованный набор ограничений, учитываемых при проектировании распределённой гипермедиа-системы \cite{wiki-rest}.

\subsection{Обзор Web-ресурсов для учебных центров}
    Для выполнения работы по созданию веб-ресурса учебного центра можно опираться на уже представленные сайты других организаций.

    Для определения технической стороны можно воспользоваться сервисами-анализаторами, например приложением компании iTrack \cite{iTrack}. 
    При использовании данного сервиса можно определить, что сайт вологодской компании А-Элита \cite{aelita} (Главная страница сайта отображена на Рисунке \ref{aelita}) использует WordPress.

    \addimghere{aelita.png}{0.6}{Главная страница сайта компании А-Элита}{aelita}

    Результат проверки отображен на Рисунке \ref{aelita-tr}.

    \addimghere{aelita-tr.png}{0.6}{Результат проверки сайта компании А-Элита}{aelita-tr}

    В целом анализ содержимого сайта можно свести к двум выводам.
    Первый - используется CMS WordPress.
    Второй - главная страница представляет собой страницу-лэндинг с размещенными на ней ссылками, перенаправляющими либо на сторониие ресурсы, либо на разделы этой же страницы.

    Пример части главной страницы расположен на Рисунке \ref{aelita2}.

    \addimghere{aelita2.png}{0.6}{Часть главной страницы компании А-Элита}{aelita2}

    Данный подхо прост в реализации и позволяет реализовать все необходимые требования не прибегая к созданию множества страниц и переходов между ними.

    Еще одним примером учебного центра будет Учебный центр "Энергетик" \cite{energy}.
    Результат (отображен на Рисунке \ref{energy-tr} его проверки в сервисе компании iTrack - использование CMS Drupal.

    \addimghere{energy-tr.png}{0.6}{Результат проверки сайта Учебный центр "Энергетик"}{energy-tr}

    При анализе внешнего вида веб-ресурса можно выявить, что здесь реализуется уже многостраничная система сайта, обрабатываемая собственно CMS Drupal.
    На Рисунке \ref{energy} отображен внешний вид главной страницы с раскрытым меню переходов.

    \addimghere{energy.png}{0.6}{Главная страница сайта Учебный центр "Энергетик"}{energy}.
