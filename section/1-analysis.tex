\section{Aналитический обзор информационного обеспечения учебных центров}
\subsection{ИТ-управление учебным центром: цели, задачи и функции}
По части вопроса управления учебным центром можно выделить несколько пунктов, задействующих в себе полностью или частично отрасль информационных технологий и представляющий интерес в данной работе.
Итак, соответствующие темам вопросы:
\begin{itemize}
    \item Управленческие - от поддержания внутренней базы кадров до бухгалтерского учета;
    \item Технической поддержки - хранение информации об используемой технике, ПО и его версиях, параметрах настройки, а также распределении по кабинетам/аудиториям;
    \item Информационно-оповещательные - распространение информации о деятельности учебного центра и способах взаимодействия с ним;
    \item Внутреннего учета - обеспечение возможности получения, обработки и хранения данных о клиентах учебного центра (студенты, ответственные за них лица);
    \item Проверки выполнения заданий - предоставление функционала по сбору, выдаче и контролю за правильностью выполнения поставленных студентам задач;
    \item Контрольная - проверка доступов и способов взаимодейтсвия внутри и между систем.
\end{itemize}

Данный список содержит пункты, которые могут пересекаться, быть вложенными друг в друга или взаимодействовать иным образом в зависимости от конкретных задач, поставленных перед учебным центром, специфике его целевой аудитории и прочих факторов.

Исходя из приведенных выше вопросов уже можно сделать вывод об индивидуальности подхода в проектировании и использовании ИТ-инфраструктуры.
Теперь можно перейти к рассмотрению основной задачи данного раздела - определению ключевых требований к ИТ-управлению учебным центром.

Цели

Наверное, самым простым вопросом относительно применения ИТ в управлении учебным центром является постановка цели.
В данном случае лучшей формулировкой будет - оптимизация процессов взаимодействия сотрудников и технического обеспечения учреждения для получения наиболее продуктивного подхода в организации учебного процесса и, как результат, максимизация прибыли.

Задачи

В этом пункте надо понимать постановку вопроса и конкретные требования от заказчика - непосредственно организации.
Но, тем не менее, можно сформулировать общие пункты-задачи для ИТ-инфраструктуры учебного центра:
\begin{itemize}
    \item Оптимизация способов взаимодействия между действующими лицами относительно услуг учебного центра (клиент - преподаватель, преподаватель - студент, студент - учебная система и т.д.);
    \item Предоставление информации о деятельности учебного центра (состав учебных программ, наличие свободных мест в группах, данные об учебном процессе и прочее);
    \item Формирование технической возможности автоматизированной работы студента с системой учреждения (проверка, выдача задач, ведение статистики);
    \item Обеспечение документооборота внутри организации (бухгалтерия, данные сотрудников и учеников);
    \item Ведение учета посещаемости и оценок за выполнение задач.
\end{itemize}

И снова считаю важным указать, что данный набор пунктов не является абсолютным и напрямую зависит от требований заказчика ИТ-системы.

Функции

С данным пунктом проще всего - он должен содержать выжимку из приведенных выше, оформленную в четко сформулированные понятия.
Итак, используя разобранные ранее вопросы можно выделить следующие функции ИТ-управления учебным центром:
\begin{itemize}
    \item Цифровизационная - частичный или полный перенос бумажного документооборота в цифровой формат;
    \item Оптимизационная - модернизация имеющихся процессов, исключение лишних, преобразование неоптимальных;
    \item Социальная - расширение охвата целевой аудитории учреждения, привлечение новых клиентов>;
    \item Управленческая - предоставление контроля над элементами системы учебного центра;
    \item Статистическая - сбор и обработка статистики о процессах работы учреждения;
    \item Автоматизационная - перенос части вручную выполняемых задач на автоматически исполняемые.
\end{itemize}

Частичное или полное соответствие разрабатываемой ИТ-системы приведенным функциям может свидетельствовать о ее пригодности применения в деятельности учебного центра.

\subsection{Типы сайтов и обзор средств разработки сайтов}

Относительно разделения сайтов на категории существует множество подходов, использующих разные способы выделения ключевых качеств для определенного типа.

Наиболее распространенными способами диффиренциации сайтов являются следующие:
\begin{itemize}
	\item По контенту - какую информацию предоставляет сайт;
	\item По способу предоставления контента - каким образом данные передаются конечному пользователю;
	\item По универсальности отображения контента - адаптирование дизайна элементов сайта под изменямые параметры устройства пользователя;
	\item По функциональности - изначально решаемой сайтом задачей;
\end{itemize}

Исходя из таких категорий можно более подробно рассмотреть каждую из них.

Распределение сайтов по типу контента.

Весьма очевидно, что в зависимости от сферы деятельности организации, будут меняться и данные, которые сайт должен предоставлять пользователю-клиенту.
Например, профессиональное агентство "Студия Вячеслава Денисова" \cite{denisov} выделяет следующие категории:
\begin{itemize}
	\item Блоги
    
    Блоги - регулярно обновляемые страницы, привязанные к определенной личности, либо компании.
	Тематика и способ подачи информации блога могут быть совершенно разнообразными.

	Например, некогда очень популярный сайт livejournal.com \cite{livejournal} является платформой для создания и ведения блогов.
    
	\item Корпоративные сайты

	Все больше компаний осознают, что у них должен быть хотя бы базовый сайт, чтобы они выглядели заслуживающими доверия и профессиональными. Компании могут не продавать напрямую через корпоративные веб-сайты, но они могут использовать их с целью предоставления информации о себе и для поддержки клиентов.

	\item Краудфандинговые платформы

	В прошлом финансирование нового делового предприятия или проекта предусматривало поиск крупных сумм денег у ограниченного круга состоятельных граждан — спонсоров.

	Краудфандинг — новая практика финансирования проектов или предприятий путем привлечения небольших сумм у множества людей. Она включает создание привлекательного видео и описание преимуществ проекта, постановку целей в надежде достичь ее к установленному сроку (сумма необходимых средств).

	Заинтересованные люди пожертвуют небольшие суммы, чтобы стать частью проекта и воспользоваться определенными бонусами. Краудфандинговые сайты становятся очень популярным ресурсом для финансирования новых стартапов.

    Пример такого сайта - kickstarter.com \cite{kickstarter}.

	\item Сайты e-commerce

	Сайт электронной коммерции может пересекаться с блогом или корпоративным сайтом, однако в конечном итоге его целью является продажа продуктов и услуг через Интернет.

	Сайт, который является исключительно корпоративным и не имеет функций продажи, все еще побуждает пользователей покупать продукт или услугу. Вся разница в том, что они не могут совершить сделку через сам сайт.

    Пример такого сайта - ozon.ru \cite{ozon}.

	\item Образовательные ресурсы

	«Какие бывают виды веб-сайтов?», «Как сварить яйцо?»… очень вероятно, что при вводе в поисковую систему таких запросов вы натолкнетесь на информационные (образовательные) сайты. Их цель — предоставить пользователю сведения, которые он ищет.

    Пример такого сайта - stepik.org \cite{stepik}.

	\item Новостные сайты

	Сайты новостей и журналов нуждаются в небольшом объяснении.

	Основная цель новостного сайта — держать читателей в курсе текущих событий. То же самое можно сказать о сайте онлайн-журнала, где в большей степени внимание уделяется развлечениям. Если вы хотите открыть газету или онлайн-журнал, взгляните на функциональные возможности CMS «1С-Битрикс: Управление сайтом».

	\item Социальные медиа

	Сайты социальных сетей уникальны как по функциональности, так и по контенту.

	Социальные медиа созданы как место для обмена мыслями, изображениями и видео, и все чаще становятся популярным местом для чтения свежих новостей и даже для обучения.

	\item Сайты ТВ или потокового видео

	В последние годы популярность сайтов потокового видео резко пошла вверх.

	Netflix и подобные сайты произвели революцию в самом способе просмотра телевизионных программ по всему миру. Сотни тысяч часов кино и телеэфира, высокое качество просмотра и относительно небольшая плата — этот вид сайтов расширяет горизонты ТВ.
\end{itemize}

Разновидности сайтов по способу предоставления контента согласно статьи того же профессионального агентства "Студия Вячеслава Денисова" \cite{denisov}:
\begin{itemize}
    \item Статические сайты

    Статические веб-сайты — самые простые, они появились в Интернете первыми.
    Их содержимое не зависит от действий пользователя и обновляется относительно редко, только при участии контент-менеджера.
    Статические веб-сайты создаются с использованием простого HTML-кода и выполняют только информативную функцию.
    
    \item Динамические сайты
    
    Динамический сайт или веб-страница позволяет отображать разный контент при каждом посещении.
    В эту категорию входят блоги, сайты электронной коммерции и вообще любой сайт, который регулярно обновляется с участием пользователей.
    
    Динамические веб-сайты также могут быть настроены для демонстрации разного контента разным пользователям, с учетом определенного времени дня и др.
    Динамические веб-сайты обеспечивают более личный и интерактивный интерфейс для пользователя, хотя их разработка сложней и дороже, чем статических сайтов.
\end{itemize}

Различия в сложности разработки и поддержки статических и динамических сайтов лежит также в используемых для этого технологиях - как правило их набор, требования к разработчикам и количество единиц используемого ПО сильно вырастает при уходе от статических сайтов.
Более подробное рассмотрение технических средств, применяемых для работы с динамическими веб-страницами будет приведет в последующих главах данной работы.

Тематика контента, а также способ его предоставления играют важную роль, но за последние десятилетия вслед за развитием техники, обретала большую значимость и возможность пользоваться знакомыми ресурсами в привычном их виде, но с применением разных устройств.
Естественным спутником распространения смартфонов явилась и необходимость позволить пользователям взаимодействовать со знакомыми приложениями и веб-ресурсами с экрана небольшого устройства с диагональю экрана в несколько дюймов.

Данная задача была решена с использованием разных технологий и подходов, а общее название ей "отзывчивость дизайна".
Разделение веб-сайтов по данному критерию также приведено в статье "Студия Вячеслава Денисова" \cite{denisov}:
\begin{itemize}
    \item Статические макеты

    Статические, или фиксированные макеты сайта плохо оптимизированы для экрана, отличающегося от стандартного десктопного ПК. Они имеют фиксированную ширину в пикселях, потому при открытии подобной веб-страницы на мобильном телефоне пользователям придется увеличивать каждый фрагмент для прочтения.
    
    С одной стороны, статические сайты могут загружаться немного быстрее из-за их простоты. Однако они категорически не рекомендуются в эпоху мобильного Интернета из-за ужасного пользовательского опыта для владельцев мобильных телефонов и планшетов.
    
    Более 50\% всех поисковых запросов в настоящее время осуществляется на мобильных устройствах, поэтому mobile friendly – важнейший фактор при разработке сайта.
    
    Жидкие или подвижные макеты
    
    Веб-сайт, созданный с использованием жидкого дизайна, гарантирует, что страница будет отображаться одинаково с точки зрения пропорций, независимо от размера вашего экрана. Каждый элемент сайта должен занимать одинаковое место на экране благодаря использованию процентов вместо пикселей. Например, 5% от ширины.
    
    Адаптивные макеты
    
    При так называемом адаптивном дизайне макет определяет ширину экрана и соответствующим образом приспосабливается к ней с помощью медиа-запросов.
    
    Изначально в адаптивный макет закладываются блоки фиксированного размера, но количество таких фиксированных размеров достаточно для любых экранов. Медиа-запрос CSS определяет ширину браузера и автоматически выбирает оптимальный для нее размер.
    
    Такое решение идеально подходит для устаревших статических сайтов, которые требуется дешево и сердито модернизировать для поддержки мобильных устройств. Проблема в том, что отсутствие определенных размеров блоков может приводить к неправильному отображению веб-страниц на новых устройствах.
    
    Отзывчивые макеты
    
    Преимущество отзывчивого сайта в том, что здесь одновременно используются медиа-запросы CSS о ширине браузера и относительные величины (как в подвижном макете).
    
    В результате становится возможным плавное изменение и корректное отображение информации. Это наиболее совершенный подход в эпоху мобильного Интернета.
\end{itemize}

\subsection{Анализ систем управления контентом}
e
\subsection{Паттерны и фреймворки}
r
\subsection{Обзор Web-ресурсов для учебных центров}
t

% Глава 1 должна содержать: 
% выбор направления исследований или направления разработки; 
% описание, способы (методы) совершенствования технологического(их) процесса(ов); 
% аналитический обзор источников информации и нормативных данных
\section{Анализ предметной области разработки Web-ресурса}
\subsection{Обзор предметной области}
    Разработка веб-ресурса как предметная область - весьма современная и востребованная тема.
    На данный момент сложно представить компанию без веб-ресурса.
    При этом даже можно не указывать область деятельности компании -- все виды коммерческого взаимодействия подразумевают контакт с внешним миром через средства связи и одно из наиболее распространенных и востребованых -- веб-ресурс.
    
    Само понятие веб-ресурса не конретизирует форму представления, а лишь заключает в себе способ передачи информации целевой аудитории.
    Таким образом под такое определение может попасть как полноценное приложение, так и, например, одностраничный сайт с картинкой-заглушкой.

    % // TODO Обзор предметной области

\subsection{Современные технологий и подходы в разработке web-ресурсов}
    % // TODO Современные технологий и подходы в разработке web-ресурсов
    \subsubsection{Системы управления контентом}
        Широкое распространение в последние годы получает направление упрощенного формирования веб-ресурсов.
        Простота подхода заключается в используемых средствах для достижения финального результата.
        Одним из таких средств явлются системы управления контентом (Content Management System), заключающие в себе программное обеспечение, строящее процесс взаимодействия с пользователем по типу работы с конструктором. % ссылка на вики цмс
        Основные выгоды, получаемые при выборе данного подхода сосредоточены вокруг использования уже готовых решений и их кастомизации под конкретные нужды заказчика.
        Процесс доработки при этом зачастую является менее требовательным к уровню подготовки исполнителя, а потому становится проще и доступнее даже для людей, не владеющих навыками взаимодействия с кодом.\\
        За доступность в использовании приходится расплачиваться ограничениями, заложенными в саму платформу-конструктор.
        Зачастую эти ограничения выливаются в виде невозможности тонкой настройки отдельных компонентов системы под конкретные нужды пользователя.
        Эта проблема решается, обычно, несколькими путями - либо переходом на более кастомизируемый и подходящий под задачи конструктор, либо взаимодействием с разработчиком (зачастую на платной основе) с целью получения желаемых изменений в продукте.
        Второй вариант может проявляться в разных видах взаимодействия - единоразовое приобретение прав на самостоятельное внесение изменений, долгосрочный контракт с вендором.
        Еще одним из минусов такого подхода является вынужденная зависимость от предоставляемого ПО, а следовательно и от его разработчика.

        Но даже с учетом приведённых недостатков, данный вид формирования веб-ресурса развивается.
        Вот лишь некоторые из самых популярных представителей современных систем управления контентом по версии портала CMS magazine \cite{}:
        % // TODO  информация с вики про примеры + скрины
        \begin{itemize}
            \item Wordpress
            \item 1С-Битрикс
            \item Joomla
            \item Drupal
            \item Tilda Publishing
        \end{itemize}

    \subsubsection{Применяемые подходы к организации веб-ресурса}
        Альтернативным подходом к использованию систем управления контентом, является изначальная разработка всей системы.
        Данный подход вовсе не отрицает возможности использовать наемный труд со стороны, а лишь указывается, что программные средства будут разработаны под индивидуальные требования заказчика.

        С точки зрения подходов в работе веб-ресурсов можно выделить два направления:
        \begin{itemize}
            \item Многостраничное приложение или MPA - multi page application.
            Заключается в построении схемы обмена данными между клиентом и сервером, основанной на передаче цельных HTML-файлов.
            Данный способ приводит к перезагрузке страницы при каждом запросе;
            \item Одностраничное приложение или SPA - single page application.
            Заключается в ином подходе - подгрузке основной части приложения при инициализирующем запросе и дальнейшее наполнение данными при последующих запросах к серверу.
        \end{itemize}
        % // TODO схемы отправки-получения запросов для этих методов
        % // TODO + и - подходов

        Данные подходы уже хорошо себя зарекомендовали и применяются повсеместно.
        Но поверх них существуют и другие уровени реализации приложения - непосредственное воплощение в технологиях.

        Подходы в использовании определенного набора технологий в профессиональной среде называются "стек" (от английского stack - куча).
        Само название стека определяется его содержимым - набором инструментов, обычно отвечающим следующим задачам:
        \begin{itemize}
            \item Формирование клиентской части приложения - то что видит и с чем взаимодействует пользователь;
            \item Взаимодействие сторон клиента и сервера - построение каналов обработки сообщений между частями приложения;
            \item Хранение и взаимодействие с базой - собственно организация накопления, обработки и выдачи данных для использования приложения.
        \end{itemize}

        Например стек MEAN берёт своё название от имен применяемых технологий, решающих приведенные выше задачи:
        % // TODO можно подробнее по каждому пункту с ссылками
        \begin{itemize}
            \item MongoDb - СУБД;
            \item Express.js - построение маршрутизации в приложении;
            \item Angular.js - клиентская (front-end) часть приложения;
            \item Node.js - серверная (back-end) часть приложения.
        \end{itemize}
        
        И простым изменением одного из компонентов можно получить новый стек - MERN.
        В данном случае за клиентскую часть будет отвечать другая технология - React.js.

        Ещё одним важным этапом является выбор способа взаимодействия между клиентской и серверной частью.
        За данный пункт отвечают специальные интерфейсы прикладных программ - API (application programming interface), выполняющие роль посредника между частями приложения и определяющие сам способ передачи информации, её вид.
        % // TODO цитата на рест
        Наиболее популярным является архитектурный подход REST, основывающийся на протоколе HTTP для транспортировки данных.
        Например, в приведённом выше стеке MERN/MEAN подход REST реализуется на базе технологии Express.js.
        Более подробно о строении подхода REST будет рассказано в % вставить ссылку на часть диплома с разбором rest-a

        Альтернативным к REST подходом, является GraphQL, рассматривающий взаимодейтсвие и работу с данными через графы.
        Данный подход также будет подробнее рассмотрен позже.


    \subsubsection{Применяемые языки программирования}
        Разработка веб-ресурсов вот уже несколько десятилетий вбирает в себя самые разные технологии и подходы.
        В данном разделе будут приведены наиболее востребованные и распространенные языки программирования и их сочетания, применяемые в индустрии на 2021-2022 год.
        % // TODO подробности
        \begin{itemize}
            \item HTML
            \item CSS
            \item PHP
            \item Ruby
            \item Python
            \item Java
            \item JavaScript
        \end{itemize}
        
        В целом приведенный список можно разделить на две важные части - языки для front-end части и back-end.
        Здесь важно указать, что подразумевают два этих понятия.

        Front-end - часть веб-приложения, отвечающая в большей степени за визуальную составляющую - то, с чем взаимодействует пользователь.
        К чисто фронтовым технологиям можно отнести HTML и CSS - они отвечают именно за часть отображения и оформления контента.
        
        Back-end - часть веб-приложения, отвечающая за обработку системной и бизнес-логики для дальнейшей её передачи в некотором виде пользователю через фронтовую часть.
        К данной категории можно отнести языки, сосредотачивающие контроль над серверной стороной - Ruby, Java.

        Хоть приведенные примеры и несколько категоричны, важно указать, что границы между четким front и back приложения и относящимся к ним языкам программирования имеют тенденцию размываться, что приводит к неоднозначности, например, такого утверждения: "JavaScript - чисто front-end язык программирования", которое может быть верным и неверными в зависимости от множества технических факторов.

        Исходя из описанных выше данных, будут выбраны в предпочтение для реализации технологии, касающиеся языка JavaScript.

    \subsubsection{Технологии разработки для front-end}
        В данном пункте приведены различные реализации надстроек над основным языком, называемые фреймворками.
        Фреймворк — программная платформа, определяющая структуру программной системы; программное обеспечение, облегчающее разработку и объединение разных компонентов большого программного проекта [] % TODO ссылка на определение фреймврока с вики
        
        Сам фреймворк играет ключевую роль в способе представления информации и взаимодействия с ней.
        Поскольку фреймворк является лишь расширением, ему требуется основание для работы.
        Таким основанием может являться один из языков программирования, который ложится в основу проработки внутренней логики и строения приложения.
        Наиболее популярные фреймворки, расширяющие возможности JavaScript:
        % // TODO Технологии разработки для front-end
        \begin{itemize}
            \item react
            \item angular
            \item vue
        \end{itemize}
        Дополнительно стоит упомянуть возможность работы с "чистым" JavaScript - без применения фреймворков.
        Для такой ситуации можно применить подход описания систем управления контентом, представленные ранее в данной работе.
        Минусы и плюсы использовани фреймворков также заключаются в ограничениях свобод разработчика, его зависимости от используемой технологии.
        Есть компании, определяющие подход своей разработки с приложением через применение чистого JavaScript без дополнений, но, как правило вместе с таким подходом повышается уровень требований к программистам ввиду необходимости реализации множества компонентов приложения.

    \subsubsection{Технологии разработки для back-end}
        Некоторая уникальность JavaScript заключается в возможности его применения как для front-end части, так и для back-end.
        Данная возможность появилась вместе с созданием Райаном Далом технологии Node.js в 2009 году.

        Node.js - программная платформа, основанная на движке V8 (транслирующем JavaScript в машинный код), превращающая JavaScript из узкоспециализированного языка в язык общего назначения.
        Node.js добавляет возможность JavaScript взаимодействовать с устройствами ввода-вывода через свой API, написанный на C++, подключать другие внешние библиотеки, написанные на разных языках, обеспечивая вызовы к ним из JavaScript-кода.
        Node.js применяется преимущественно на сервере, выполняя роль веб-сервера, но есть возможность разрабатывать на Node.js и десктопные оконные приложения (при помощи NW.js, AppJS или Electron для Linux, Windows и macOS) и даже программировать микроконтроллеры (например, tessel, low.js и espruino).
        В основе Node.js лежит событийно-ориентированное и асинхронное (или реактивное) программирование с неблокирующим вводом/выводом. % ссылка на ноду \cite{wiki-nodejs}
        
        Сами технологии, основанные на Node.js многогранны, но наиболее востребованными для разработки веб-приложения являются:
        % // TODO Технологии разработки для back-end
        \begin{itemize}
            \item electron.js
            \item express.js
        \end{itemize}

    
    \subsubsection{Операционные системы}
        Пункт, связанный с операционными системами, на мой взгляд, также является достойным обсуждения.
        С точки зрения веб-разработки нет большой разницы в применяемой платформе - основной набор технологий будет работать идентично на любой ОС.
        % // TODO Операционные системы
        \begin{itemize}
            \item windows
            \item mac os
            \item linux
        \end{itemize}
    
        
    \subsubsection{Среды разработки и редакторы кода}
        Одной из важных частей разработки является непосредственное взаимодействие с кодом - его редактирование, написание, удаление, чтение.
        Эти процессы являются критическими для разработчика, поскольку являются 
        % // TODO Среды разработки и редакторы кода
        \begin{itemize}
            \item VS Code
            \item sublime
            \item atom
            \item VS
        \end{itemize}
    
    \subsubsection{Редакторы диаграмм}
        % // TODO Выбор редактора диаграмм
        Необходимы для составления диаграм взаимодействия
        Одним из наиболее популярных средств является универсальный язык моделирования UML \cite{wiki-UML}.
        Видов программного обеспечения, использующего данный язык моделирования много:
        \begin{itemize}
            \item Rational Rose
            \item Dia   http://dia-installer.de/
            \item PlantUML  https://plantuml.com/ru/
            Можно использовать для отображения объектов и сущностей формата JSON (т.к. исопльзуется ЯП javascript) - https://plantuml.com/ru/json

            Использование с латехом
            https://plantuml.com/ru/latex

        \end{itemize}

\subsection{Определение требований к web-ресурсам}
    % // TODO Определение требований к web-ресурсам
% TODO добавить ссылку, переработать текст

    Консо́рциум Всеми́рной паути́ны (англ. World Wide Web Consortium, W3C) — организация, разрабатывающая и внедряющая технологические стандарты для Всемирной паутины. Основателем и главой Консорциума является сэр Тимоти Джон Бернерс-Ли, автор множества разработок в области информационных технологий. По состоянию на 29 мая 2019 года Консорциум насчитывает 444 члена.

    W3C разрабатывает для Интернета единые принципы и стандарты (называемые «рекомендациями», англ. W3C Recommendations), которые затем внедряются производителями программ и оборудования. Таким образом достигается совместимость между программными продуктами и аппаратурой различных компаний, что делает Всемирную сеть более совершенной, универсальной и удобной.

    Миссия W3C: «Полностью раскрыть потенциал Всемирной паутины путём создания протоколов и принципов, гарантирующих долгосрочное развитие Сети».
    Более конкретная цель W3C — помочь компьютерным программам достичь способности к взаимодействию в Сети (т. н. «сетева́я интеропера́бельность», англ. Web interoperability). Применение единых стандартов в Сети — это ключевой шаг для достижения такого взаимодействия.

    Две другие важнейшие задачи Консорциума — обеспечить полную «интернационализа́цию Сети́» и сделать Сеть доступной для людей с ограниченными возможностями. Для решения первой задачи Консорциум активно сотрудничает с организацией «Юнико́д» (англ. Unicode) и рядом других рабочих групп, занимающихся международным сотрудничеством в Интернете и языковыми технологиями. Для решения второй задачи Консорциум не только сотрудничает с организациями соответствующего профиля, но и разработал свои собственные рекомендации, которые сейчас активно набирают популярность. Существует "Руководство по обеспечению доступности веб-контента (WCAG)", разработанное Консорциумом всемирной паутины (W3C). В нем четко расписаны все требования по контенту сайтов и его форматированию, чтобы максимум людей могли комфортно пользоваться информацией.
    
\subsection{Проверка соответствия существующей системы современным требованиям и технологиям}
    % // TODO Проверка соответствия существующей системы заявленным требованиям
    \subsubsection{Обзор версий оформления страниц сайта и их развития}
        % // TODO 
        https://web.archive.org/web/20120219134344/http://www.meson.ru/

\subsection{Постановка требований к разрабатываемому web-ресурсу}
    % // TODO Постановка требований к разрабатываемому web-ресурсу

\subsection{Постановки задачи}
    % // TODO Постановки задачи
    Разработка веб-приложения в соответствии с представленными ранее подходами и технологиями.

\clearpage
