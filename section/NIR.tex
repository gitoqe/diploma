\anonsection{Введение}

\textbf{Цель практики «Научно-исследовательская работа»} – развитие у студентов целеустремленности, организованности, овладение методами для    аналитической работы и научных исследований; формирование у студентов компетенций и навыков ведения самостоятельной исследовательской работы.

\textbf{Задачи практики «Научно-исследовательская работа»}:

-- закрепление полученных теоретических знаний по всему комплексу профильных и смежных с ними дисциплин;

-- закрепление практических навыков, приобретенных при выполнении лабораторных работ и практикумов;

-- приобретение практических навыков и компетенций в области информационных систем и технологий;

-- сбор материала для выполнения научно-исследовательской работы.
\clearpage
\section{Типы сайтов и обзор средств разработки сайтов}

В публикации \cite{ualiev-jebegenov-issledovanie} журнала "Вестник науки" за авторством Уалиева Н.С. и Жебегенова Ә.М. проводится следующий обзор технологий для разработки сайтов.

Выбор соответствующего технологического стека особенно сложен образовательных учреждений, поскольку они обычно имеют ограниченные бюджеты и, таким образом, нуждаются в технологическом стеке, который обеспечивает наименьшую финансовую трату, чтобы получить свои проекты бесплатно.
Правильный стек технологий в значительной степени является ключом к успеху проекта, в то время как неправильный выбор технологий веб-разработки может быть причиной неудачи.
Есть несколько критерии чтобы выявить наиболее подходящего стека технологий для веб-приложения.
Прежде чем переходить к критериям выбора современного стека веб-технологий, необходимо четко понимать, что включает в себя процесс веб-разработки.
Не вдаваясь слишком глубоко в детали, есть две стороны веб-разработки: на стороне клиента и на стороне сервера.
Клиентская сторона также называется передней частью.
Программирование на стороне сервера включает в себя приложение (и внутренний язык программирования, который питает его), базу данных и сам сервер.

Программирование на стороне клиента.
Веб-разработка на стороне клиента (т. е. интерфейс) включает в себя все, что пользователи видят на своих экранах. Вот основные компоненты стека интерфейсных технологий:

-- Язык разметки гипертекста (HTML) и Каскадные таблицы стилей (CSS).
HTML сообщает браузеру, как отображать содержимое веб-страниц, в то время как CSS-стили этого содержимого.
Bootstrap, Wordpress, Joomla – это полезная платформа для управления HTML и CSS.

-- JavaScript (JS). JS делает веб-страницы интерактивными.
Существует множество библиотек JavaScript (таких как jQuery, React.js, и Zepto.JS) и фреймворки (такие как Angular, Vue, Backbone и Ember) для более быстрой и легкой веб-разработки.

Программирование на стороне сервера.
Серверная сторона не видна пользователям, но она питает клиентскую сторону, так же, как электростанция генерирует электричество для дома.
Основная проблема заключается в выборе серверных технологий для разработки веб-приложения.
Что касается серверных языков программирования, то они используются для создания логики сайтов и приложений.
Фреймворки для языков программирования предлагают множество инструментов для более простого и быстрого кодирования.
Отметим некоторые популярные языки программирования и их основные фреймворки (в скобках): Ruby (Ruby on Rails), Python (Django, Flask, Pylons), PHP (Laravel), Java (Spring), Scala (Play).
Node.JS, Среда выполнения JavaScript, также используется для внутреннего программирования.

Веб-приложению требуется место для хранения данных, для чего и используется база данных.
Существует два типа баз данных: реляционные и нереляционные (последний подразделяется на несколько категорий), каждая из которых имеет свои плюсы и минусы.
Вот наиболее распространенные базы данных для веб-разработки: MySQL (relational), PostgreSQL (relational), MongoDB (non-relational, document).

Веб-приложению требуется система кэширования для снижения нагрузки на базу данных и обработки больших объемов трафика.
Memcached и Redis являются наиболее распространенными системами кэширования.
Наконец, веб-приложению требуется сервер для обработки запросов от компьютеров клиентов.
В этой области есть два основных домена: Apache и Nginx.

Для разработки веб-приложения необходимо выбрать сервер, базу данных, язык программирования, фреймворк и инструменты интерфейса, которые будут использованы.
Эти технологии веб-разработки основаны друг на друге и, по сути, в совокупности называются стеком \cite{ualiev-jebegenov-issledovanie-1}.
\clearpage
\section{Анализ систем управления контентом}

Более конкретный анализ систем управления контентом проведен в работе \cite{ivanisheva-cms} А.А. Иванищевой Х.И. Комилова и М.Д. Гехаева.

Система управления контентом (CMS) -- является системой, управляющей информацией или компьютерной программой, обеспечивающей процесс совместного создания, управление и редактирования содержимого ресурса.
В современном профессиональном и непрофессиональном CMS называют "движком" сайта.
Они подразделяются на платные и бесплатные.

В работе рассматриваются популярные бесплатные системы управления контентом -- WordРress, Joomla
и Drupal.

WordРress является одной из популярных бесплатных простых платформ по созданию сайтов.
Создание собственного сайта займет всего полчаса.
В платформу встроено множество простых шаблонов, позволяющих даже без навыков программирования получить ресурс с требуемой тематикой, обладающий нужным функционалом.
Из существенных минусов этого движка выделяются следующие: требования к объему информации, большая нагрузка на сервер, риск дублирования страниц, блокировка многих антивирусов к доступу на сайт, малый набор базовых функций.

Joomla также является одной из популярных систем управления контентом.
Она бесплатна, как и WordРress, но обладает некоторыми отличиями.
Joomla содержит платные шаблоны, улучшенные по дизайну, которые можно модифицировать с помощью встроенных компонентов, модулей и плагинов.
Данная система часто обновляется, что обеспечивает безопасность от возможного взлома.
Из минусов использования Joomla можно выделить:: появление копий при редактировании материала страницы, ограниченность SEO, сложность в установке, редактировании и изучении.

Drupal -- еще одна CMS, нацеленная на больший охват аудитории.
Она подходит для создания крупных интернет-магазинов.
Может использоваться в виде каркаса для web-приложений.
Важным плюсом является имеет открытый исходный код и поддерживаемый уровень безопасности системы.
Из важных минусов Drupal: высокая сложность овладения, сложности в работе с БД, высокие требования к хостингу.

Исходя из перечисленных бесплатных популярных CMS можно сделать вывод, что каждая платформа универсальна в своей сфере применения.
Например, WordРress лучше подойдет новичкам.
Платформа Joomla рассчитана на более опытных специалистов в сфере IT.
Drupal ориентирован больше на профессиональных разработчиков web-приложений. 
\clearpage
\section{Паттерны и фреймворки}

Подходы в использовании определенного набора технологий в профессиональной среде называются "стек" (от английского stack - куча, кипа).
Само название стека зачастую определяется его содержимым - набором инструментов, технологий или операционных систем, обычно отвечающим  задачам формирования клиентской части приложения, взаимодействия сторон клиента и сервера, хранения и взаимодействия с базой данных другим, в зависимости от области и специфики применения.


Подробный разбор веб-стеков приведен в статье Давыдовского М.А. "Выбор веб-стека для реализации цифровой среды предоставления транспортных услуг" \cite{davidovsky-vibor}.

%LAMP

LAMP стек \cite{davidovsky-vibor-1} использует операционную систему Linux (L), веб-сервер Apache (A), базу данных MySQL (M) и язык PHP (P) \cite{davidovsky-vibor-php}.
В настоящее время это самый популярный стек для разработки динамических сайтов и веб-приложений.

Имеются следующие модификации этого стека: 

-- LEMP стек вместо Apache использует nginх;

-- LLMP стек вместо Apache использует lighttpd. Lighttpd -- это быстрый защищенный веб-сервер, который работает как в операционной системе Linux, так и Microsoft Windows;

-- LAPP стек вместо MySQL использует PostgreSQL;

-- WAMP стек вместо Linux использует операционную систему Microsoft Windows;

-- XAMPP стек может работать как в Linux, так и в Windows и использует язык PHP или Perl.

% WISA

WISA стек использует только продукты фирмы Microsoft: операционную систему Windows, веб-сервер Internet Information Services (IIS), систему управления базами данных Microsoft SQL Server, платформу для разработки веб-приложений ASP.NET.

Платформа ASP.NET позволяет писать код приложения на разных языках, входящих в комплект .NET Framework.
Это языки: C\#, VisualBasic.NET, J\#, JScript.NET и другие.
Эта платформа использует технологию AJAX и шаблон проектирования MVC (Model-View-Controller), который позволяет разделить модель данных, внешнее представление и управление данными.

% MAMP

MAMP стек – это стек, работающий в операционной системе macOS фирмы Apple.
Этот стек включает веб-сервера Apache и nginх, систему управления базами данных MySQL, языки программирования PHP, Perl и Python.
Данный стек позволяет работать с системой управления контентом WordPress.

Серверная платформа Node \cite{davidovsky-vibor-NODE} использует скриптовый язык программирования JavaScript.
Она не требует использования Apache или другого веб-сервера, т.к. выполняет его функции.
Поэтому в названии стеков с этой платформой не указывается веб-сервер, а на первом месте стоит система управления базами данных.

% MEAN

MEAN стек \cite{davidovsky-vibor-mean} использует следующие программы:

-- MongoDB (M) -- NoSQL-система управления базами данных, хранящая данные в формате
«ключ-значение»;

-- Express (E) -- фреймворк для разработки серверной части веб-приложений на платформе Node. Он является стандартным каркасом для разработки приложений на Node и содержит ряд методов, упрощающих код на Node;

-- AngularJS (A) -- фреймворк для разработки клиентской части одностраничных веб-приложений на платформе Node.
Он включает функции обработки данных JSON-формата, использует MVC-шаблон проектирования веб-приложения, позволяет разработать веб-приложение, которое переносит часть нагрузки по обработке данных со стороны сервера на сторону клиента;

-- Node (N) -- платформа разработки серверных приложений.

Модификациями стека MEAN являются:

-- MERN стек, в котором AngularJS заменен на React.
React -- библиотека на JavaScript для разработки пользовательских интерфейсов одностраничных и мобильных приложений;

-- MEEN стек, в котором AngularJS заменен на Ember.js.
Ember.js -- это фреймворк, предоставляющий каркас для разработки веб-приложений с использованием MVC-шаблона.
\clearpage
