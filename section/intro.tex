\anonsection{Введение}

// TODO ссылки на главы

Для выполнения данной дипломной работы было установлено требование практического применения (внедрения) результата деятельности на практике в некотором учреждении. В ходе поисков подходящих предприятий выбор пал на присутствующее в городе Частное учреждение дополнительного профессиональго образования "Учебный центр "Мезон" \cite{meson-uc} (далее - Учреждение, Организация) на базе предприятия ЗАО "Мезон" \cite{meson}. Среди востребованных со стороны Организации, а также предложенных со стороны университета тем, была наиболее интересная исполнителю тема -- разработка web-ресурса. Итоговая тема данной выпускной квалификационной работы на основе выбранной темы и предприятия -- Разработка Web-ресурса для учебного центра «Мезон».

Рассмотрение предметной области разработки и применения web-ресурсов будет приведено в основной части данной работы. Здесь же будет сконцентрировано внимание на самой теме информатики, автоматизации и обучения, на которой специализируется целевая Организация.

В последние десятилетия всё больше набирает темп всеобщая цифровизация, приводящая к возникновению у общества новых запросов к окружающей инфраструктуре.
Всё большая доступность и распространение персональных компьютеров и смартфонов влечет за собой увеличение количества пользователей, что в свою очередь влияет на сферу обслуживания этих пользователей, представленную в виде самых разных профилей занятости: разработчиков, дизайнеров, архитекторов информационных решений, техников, инженеров и многих других специалистов.
Возникающий при этом спрос пытаются удовлетворить как частные, так и государственные организации.

Примером государственного влияния на удовлетворение спроса в технических специалистах можно назвать внедрение программ обучения информационным технологиям в государтвенных образовательных учреждениях -- школах, СУЗах, ВУЗах.
Сюда входят как занятия по общей информатике в средней школе, так и различные специализированные предметы в Университетах и Институтах.

В процессе внедрения информатики в среду образования производится формирование и утверждение программ обучения.
Данный процесс занимает достаточно большое время в том числе и из-за большого количества сопроводительной документации, требующей тщательного и подробного составления, контроля и наблюдения.
Ввиду этих затрат временных ресурсов возникает эффект наложения сроков на разных уровнях учреждений, ответственных за принятие и утверждение программ.
По итогу получается весьма неповоротливая система, которая для внесения правок в существующие документы или оформления новых, требует больших затрат как человеческих, так и временных.
Результат прослеживается следующий -- общее оставание программ обучения информатике разного уровня от современных технологических стандартов.

Проблема временных затрат на работу с документацией не является единственной.
Можно также выделить ещё одну причину -- нагрузка на преподавательский состав разных уровней образования.
Преподаватель в настоящее время является не только специалистом в сфере своей компетенции, но также берёт на себя функции оформления и контроля поступающей документации -- программ обучения, нормативных актов, результатов учебной деятельности студентов (отчеты, контрольные, курсовые, дипломные, научно-исследовательские работы и многие другие), а также различные элементы внутреннего документооборота организации.
Возникающее дополнительное давление негативно сказывается на времени, которое можно было бы потратить на саморазвитие в сфере собственной компетенции или улучшении и актуализации образовательных программ.
Вместе с нагрузкой своё влияние оказывает также незаинтересованность некоторых преподавателей в изучении нового материала, используемого в сфере информационных технологий -- языков программирования, технологических подходов, программных средств, целых направлений деятельности.

Проблемы, приведённые выше не являются повсеместными, но встречаются достаточно часто, чтобы у общественности мог сформироваться запрос на альтернативные пути получения образовательных услуг.
Среди таких альтернатив можно выделить огромное множество способов и вариаций взаимодействия -- форумы, митапы, марафоны, конференции, выставки, курсы, тренинги, менторские программы, дополнительное образование и многие другие.
Данный список можно продолжать дальше в зависимости от сферы профессиональных интересов.
Суть же данного явления в том, что общество видит как темпы развития информационных технологий, так и уровень их внедрения в системы государственного образования.
Неактуальность или, вернее, несоответствие этих пунктов и порождает спрос.

Выбранная для выпускной квалификационной работы Организация как раз специализируется на предоставлении услуг дополнительного профессиональго образования.
Внутреннее устройство Учреждения не лишено проблем образовательной системы в сфере информационных технологий, приведённых ранее, но за счёт меньших масштабов производства, а также иной формы внутренней организации, обладает более адаптируемыми механизмами влияния на образовательный процесс.
Это приводит к возможности упрощенной и ускоренной актуализации программ обучения, предоставляемых Учреждением.

Для выполнения своих функций ЧУ ДПО "УЦ "Мезон" требуется взаимодействовать с клиентами.
Сюда входят различные стадии -- привлечение, консультирование, информирование, оформление для обучения, обучение и пр.
Одна из самых востребованных -- интерактивное вовлечение и информирование, которые могут осуществляться как персоналом Учреждения, так и с помощью различного программного обеспечения.
Наиболее простым и распространённым вариантом интерактивного взаимодействия с клиентами является использование web-ресурсов -- сайтов, приложений, email-рассылок.
Данные способы подразумевают технические возможности для их осуществления.

Учреждение обладает работающим сайтом \cite{meson-uc}, представляющим из себя многостраничный статический набор документов на языке гипертекстовой разметки HTML \cite{wiki-html} с использованием технологии каскадных таблиц стилей CSS \cite{wiki-css}, а также языка программирования JavaScript \cite{wiki-js}.
Данный сайт уступает по нескольким показателям:
\begin{itemize}
    \item технический -- исполнение содержимого сайта является неактуальным с точки зрения используемых технологий;
    \item функциональный -- возможности, предоставляемые сайтом не соответствуют потребностям Учреждения;
    \item информационный -- часть информации является устаревшей и неактуальной, требует замены;
    \item визуальный -- внешнее оформление сайта не соответствует современным подходам к проектированию и оформлению web-ресурсов.
\end{itemize}

Данная выпускная квалификационная работа будет сосредоточена на исправлении и актуализации приведённых пунтктов относительно сайта Учреждения.

\clearpage
