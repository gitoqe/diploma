\anonsection{Введение}

% // TODO ссылки на главы

Для выполнения данной дипломной работы было установлено требование практического применения (внедрения) результата деятельности на практике в некотором учреждении.
В ходе поисков подходящих предприятий выбор пал на присутствующее в городе Частное учреждение дополнительного профессиональго образования "Учебный центр "Мезон" \cite{meson-uc} (далее - Учреждение, Организация) на базе предприятия ЗАО "Мезон" \cite{meson}.
Среди востребованных со стороны Организации тем, а также одобренных со стороны университета, была выбрана разработка web-ресурса.
Итоговое наименование данной выпускной квалификационной работы на основе выбранной темы и предприятия -- "Разработка Web-ресурса для учебного центра "Мезон".
Рассмотрение предметной области будет приведено в основной части данной работы.
Здесь же будет сконцентрировано внимание на самой теме информатики, автоматизации и обучения, на которой специализируется целевая Организация.

В последние десятилетия всё больше набирает темп всеобщая цифровизация, приводящая к возникновению у общества новых запросов к окружающей инфраструктуре.
Всё большая доступность и распространение персональных компьютеров, смартфонов и прочих устройств влечет за собой увеличение количества пользователей, что в свою очередь влияет на сферу обслуживания этих пользователей, представленную в виде самых разных профилей занятости: разработчиков, дизайнеров, архитекторов информационных решений, техников, инженеров и многих других специалистов.
Возникающий при этом спрос пытаются удовлетворить как частные, так и государственные организации.

Примером государственного влияния на удовлетворение спроса в технических специалистах можно назвать внедрение программ обучения информационным технологиям в государтвенных образовательных учреждениях -- школах, СУЗах, ВУЗах.
Сюда входят как занятия по изучению информатики в средней школе, так и различные специализированные предметы в университетах и институтах.

В процессе внедрения информационных технологий в среду образования производится формирование и утверждение программ обучения.
Данный процесс занимает достаточно большое время в том числе и из-за большого количества сопроводительной документации, требующей тщательного и подробного составления, контроля и наблюдения.
Ввиду этих затрат временных ресурсов возникает эффект наложения сроков на разных уровнях учреждений, ответственных за принятие и утверждение программ.
Результат работы такой системы можно выразить как общее оставание программ обучения информатике разного уровня от современных практикуемых стандартов.
Университеты в большинстве своём концентрируются на фундаментальных знаниях, актуальных вне зависимости от применяемых технологий реализации, из-за чего у студентов и работодателей возникает мнение о неактуальности преподаваемых предметов.
Данная проблема является крайне важной, но разбор её придётся вынести за рамки данной работы.

Резюмируя приведённые выше пункты, получим следущую ситуацию -- есть запрос на подготовку кадров разного уровня при невозможности его удовлетворения государственной системой образования.
Решения данного вопроса возникают на разных уровнях.
Например, широкое распространение имеет практика создания внутри компаний внутренних подразделений для обучения будущих сотрудников, либо прямое взаимодействие с ВУЗами и СУЗами для прохождения стажировок и практик с дальнейшим прицелом на трудоустройство.
Решение другого уровня -- дополнение учебной деятельности в государственной организации занятиями в других учреждениях.
Сюда можно отнести учебные центры, репититорство, менторские программы, олимпиадную деятельность и тренинги.
Также стоит упомянуть способы развития уже работающих специалистов -- форумы, митапы, марафоны, конференции, выставки и онлайн курсы.

Выбранная для выпускной квалификационной работы Организация как раз специализируется на предоставлении услуг дополнительного профессиональго образования.
Спектр предоставляемых услуг захватывает как дошкольников и школьноков, так и взрослых.
Внутреннее устройство Учреждения не лишено проблем образовательной системы в сфере информационных технологий, приведённых ранее, но за счёт меньших масштабов производства, а также иной формы внутренней организации, обладает более адаптируемыми механизмами влияния на образовательный процесс.
Это приводит к возможности упрощенной и ускоренной актуализации программ обучения, предоставляемых Учреждением.

Одной из ключевых задач для ЧУ ДПО "УЦ "Мезон" является взаимодействовие с клиентами.
Сюда входят различные стадии -- привлечение, консультирование, информирование, оформление для обучения, обучение и другие виды деятельности.
Одна из самых важных-- интерактивное вовлечение и информирование, которые могут осуществляться как персоналом Учреждения, так и с помощью различного программного обеспечения.
Наиболее простым и распространённым вариантом интерактивного взаимодействия с клиентами является использование информационных ресурсов -- сайтов, приложений, email-рассылок.
Данные способы подразумевают технические возможности для их осуществления.

Учреждение обладает работающим сайтом \cite{meson-uc}, представляющим из себя многостраничный статический набор документов на языке гипертекстовой разметки HTML \cite{wiki-html} с использованием технологии каскадных таблиц стилей CSS \cite{wiki-css}, а также языка программирования JavaScript \cite{wiki-js}.
Выполняемая им задача включает в себя как информирование имеющихся клиентов, так и привлечение новых.
Потому актуальность его наполнения и внешнего вида являются критичными.

\begin{comment}
    Cайт Учреждения уступает современным трендам по нескольким показателям:
    \begin{itemize}
        \item технический -- исполнение содержимого сайта является неактуальным с точки зрения используемых технологий;
        \item функциональный -- возможности, предоставляемые сайтом не соответствуют потребностям Учреждения;
        \item информационный -- часть информации является устаревшей и неактуальной, требует замены;
        \item визуальный -- внешнее оформление сайта не соответствует современным подходам к проектированию и оформлению web-ресурсов.
    \end{itemize}    
\end{comment}

Данная выпускная квалификационная работа будет сосредоточена на актуализации сайта.

\clearpage
