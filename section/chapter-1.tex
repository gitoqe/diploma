\section{Аналитический обзор информационного обеспечения учебных центров}
\subsection{ИТ-управление учебным центром: цели, задачи и функции}

В статье компании "Инлайн Груп Центр" \cite{inlinegroup-c} дается определение учебного центра как организации, специализирующаяся на предоставлении услуг обучения в очной форме.

Исходя из данного определения, описываются базовые бизнес-процессы учебного центра, такие как: учет ресурсов, использующихся для обучения, ведение графиков доступности ресурсов и инструкторов, составление расписаний с учетом всех действующих ограничений и графиков доступности ресурсов, учет фактически проведенного обучения.

Также приводятся примеры положительного влияния автоматизации деятельности, а именно: значительное повышение качества и скорости планирования, более эффективне использование имеющихся у учебного центра ресурсов, оперативное реагирование на изменения спроса на обучение или выбытие ресурсов путем гибкого перепланирования.

На основе приведенных данных указан выработанный компанией подход к реализации проекта, состоящий из следующих пунктов: проведение анализа бизнес-процессов и разработка Концепции их автоматизации, настройка программного решения, обучение администраторов Учебного центра, разработка и осуществление плана ввода информационной системы управления Учебным центром в действие.

На официальном ресурсе \cite{1c-training-center} фирмы "1С" в описании продукта "1С:Управление учебным центром" указано следующее:

Программный продукт представляет собой комплексное решение задач управления, поддержки бизнес-процессов, учетных функций учебных центров любого профиля, вида финансирования и отраслевой направленности.
Также оно может быть применено для автоматизации работы бизнес-тренеров, центров повышения квалификации и переподготовки, центров корпоративного обучения и дополнительного образования.

Описание данного продукта также содержит основные функциональные возможности, а именно: планирование учебного процесса, составление расписания курсов и занятий, проведение набора слушателей, ведение взаиморасчетов, управление движением контингента слушателей, учет учебного процесса, учет платного питания, сопутствующие торговые операции, заработная плата, настройка конфигурации.

Из приведенных данных от компаний "1С" и "Инлайн Груп Центр" можно выделить некоторые общие пункты, изначально проектируемые для инфраструктуры ИТ-управления учебным центром, на основе которых выстраивается полный функционал итогового решения.

\subsection{Типы сайтов и обзор средств разработки сайтов}

В статье \cite{denisov} интернет-агентства «Студия Вячеслава Денисова» приводится категоризация сайтов по функциональности, отзывчивости дизайна, способу формирования контента и типу контента.

% по функциональности

Существует пять основных видов сайтов: сайты-визитки, сайты электронной коммерции, интернет-порталы, социальные платформы, вики-сайты.

Считается, что визитка — это самый доступный и технически простой веб-сайт. Подобный проект имеет несколько страниц и может использоваться малыми предприятиями, которым нужно просто утвердить свое присутствие в Интернете.

Веб-сайт электронной коммерции — это более сложный инструмент, с помощью которого пользователи могут оплачивать товары и услуги в Интернете.
Сюда относят всевозможные интернет-магазины, аукционы и тому подобное.

Особенность интернет-порталов заключается в том, что они объединяют информацию из множества разных источников.
Сюда можно отнести Яндекс, предлагающий электронную почту, форумы, поисковые системы и новости на домашней странице.

Вики-сайт — это особый проект, который позволяет людям совместно работать онлайн и писать контент вместе.
Самым популярным примером является сама Википедия, которая позволяет любому изменять, добавлять и оценивать содержание статей.

Сайты социальных сетей — это масштабные, дорогостоящие и сложные проекты, которые позволяют обмениваться изображениями, видео или идеями.
Они поощряют интерактивное взаимодействие и обмен информацией между рядовыми пользователями.

% по отзывчивости дизайна

С точки зрения того, насколько хорошо оптимизирован веб-сайт для разных цифровых устройств, макеты, или шаблоны сайта можно классифицировать на четыре категории: статический, жидкий (подвижный), отзывчивый (responsive),адаптивный макет.

Статические, или фиксированные макеты сайта плохо оптимизированы для экрана, отличающегося от стандартного десктопного ПК.
Они имеют фиксированную ширину в пикселях, потому при открытии подобной веб-страницы на мобильном телефоне пользователям придется увеличивать каждый фрагмент для прочтения.

С одной стороны, статические сайты могут загружаться немного быстрее из-за их простоты. Однако они категорически не рекомендуются в эпоху мобильного Интернета из-за ужасного пользовательского опыта для владельцев мобильных телефонов и планшетов.

Более 50\% всех поисковых запросов в настоящее время осуществляется на мобильных устройствах, поэтому mobile friendly – важнейший фактор при разработке сайта.

Веб-сайт, созданный с использованием жидкого дизайна, гарантирует, что страница будет отображаться одинаково с точки зрения пропорций, независимо от размера вашего экрана.
Каждый элемент сайта должен занимать одинаковое место на экране благодаря использованию процентов вместо пикселей.
Например, 5\% от ширины.

При так называемом адаптивном дизайне макет определяет ширину экрана и соответствующим образом приспосабливается к ней с помощью медиа-запросов.

Изначально в адаптивный макет закладываются блоки фиксированного размера, но количество таких фиксированных размеров достаточно для любых экранов.
Медиа-запрос CSS определяет ширину браузера и автоматически выбирает оптимальный для нее размер.

Такое решение идеально подходит для устаревших статических сайтов, которые требуется дешево и сердито модернизировать для поддержки мобильных устройств.
Проблема в том, что отсутствие определенных размеров блоков может приводить к неправильному отображению веб-страниц на новых устройствах.

Преимущество отзывчивого сайта в том, что здесь одновременно используются медиа-запросы CSS о ширине браузера и относительные величины (как в подвижном макете).

В результате становится возможным плавное изменение и корректное отображение информации.
Это наиболее совершенный подход в эпоху мобильного Интернета.

% по способу формирования контента

По способу формирования и обновляемости контента сайты бывают: статические и динамические.

Статические веб-сайты — самые простые, они появились в Интернете первыми.
Их содержимое не зависит от действий пользователя и обновляется относительно редко, только при участии контент-менеджера.
Статические веб-сайты создаются с использованием простого HTML-кода и выполняют только информативную функцию.

Динамический сайт или веб-страница позволяет отображать разный контент при каждом посещении.
В эту категорию входят блоги, сайты электронной коммерции и вообще любой сайт, который регулярно обновляется с участием пользователей.

Динамические веб-сайты также могут быть настроены для демонстрации разного контента разным пользователям, с учетом определенного времени дня и др.
Динамические веб-сайты обеспечивают более личный и интерактивный интерфейс для пользователя, хотя их разработка сложней и дороже, чем статических сайтов.

% по типу контента

При классификации сайтов по содержанию список может продолжаться буквально без конца.
Вот некоторые из наиболее распространенных «тем» для контента веб-сайта, хотя существует вероятность пересечения и совпадения между категориями: блоги, корпоративные сайты, краудфандинговые платформы, сайты электронной коммерции, образовательные ресурсы, новостные сайты, социальные медиа, сайты ТВ или потокового видео.

Блог — это веб-сайт или веб-страница, которая регулярно обновляется.
Как правило, блог будет вести отдельный человек или небольшая группа.
Блог может быть на любую тему, но часто будет писаться в неформальном или разговорном стиле.
В последние годы популярность профессиональных блогов значительно возросла.

Все больше компаний осознают, что у них должен быть хотя бы базовый сайт, чтобы они выглядели заслуживающими доверия и профессиональными.
Компании могут не продавать напрямую через корпоративные веб-сайты, но они могут использовать их с целью предоставления информации о себе и для поддержки клиентов.

В прошлом финансирование нового делового предприятия или проекта предусматривало поиск крупных сумм денег у ограниченного круга состоятельных граждан — спонсоров.
Краудфандинг — новая практика финансирования проектов или предприятий путем привлечения небольших сумм у множества людей.
Она включает создание привлекательного видео и описание преимуществ проекта, постановку целей в надежде достичь ее к установленному сроку (сумма необходимых средств).

Сайт электронной коммерции может пересекаться с блогом или корпоративным сайтом, однако в конечном итоге его целью является продажа продуктов и услуг через Интернет.
Сайт, который является исключительно корпоративным и не имеет функций продажи, все еще побуждает пользователей покупать продукт или услугу.
Вся разница в том, что они не могут совершить сделку через сам сайт.

«Какие бывают виды веб-сайтов?», «Как сварить яйцо?»… очень вероятно, что при вводе в поисковую систему таких запросов вы натолкнетесь на информационные (образовательные) сайты.
Их цель — предоставить пользователю сведения, которые он ищет.

Сайты новостей и журналов нуждаются в небольшом объяснении.
Основная цель новостного сайта — держать читателей в курсе текущих событий.
То же самое можно сказать о сайте онлайн-журнала, где в большей степени внимание уделяется развлечениям.

Сайты социальных сетей уникальны как по функциональности, так и по контенту.
Социальные медиа созданы как место для обмена мыслями, изображениями и видео, и все чаще становятся популярным местом для чтения свежих новостей и даже для обучения.

В последние годы популярность сайтов потокового видео резко пошла вверх.
Netflix и подобные сайты произвели революцию в самом способе просмотра телевизионных программ по всему миру.
Сотни тысяч часов кино и телеэфира, высокое качество просмотра и относительно небольшая плата — этот вид сайтов расширяет горизонты ТВ.

% обзор средств разработки сайтов
В публикации \cite{ualiev-jebegenov-issledovanie} журнала "Вестник науки" за авторством Уалиева Н.С. и Жебегенова Ә.М. проводится следующий обзор технологий для разработки сайтов.

Выбор соответствующего технологического стека особенно сложен образовательных учреждений, поскольку они обычно имеют ограниченные бюджеты и, таким образом, нуждаются в технологическом стеке, который обеспечивает наименьшую финансовую трату, чтобы получить свои проекты бесплатно.
Правильный стек технологий в значительной степени является ключом к успеху проекта, в то время как неправильный выбор технологий веб-разработки может быть причиной неудачи.
Есть несколько критерии чтобы выявить наиболее подходящего стека технологий для веб-приложения.
Прежде чем переходить к критериям выбора современного стека веб-технологий, необходимо четко понимать, что включает в себя процесс веб-разработки.
Не вдаваясь слишком глубоко в детали, есть две стороны веб-разработки: на стороне клиента и на стороне сервера.
Клиентская сторона также называется передней частью.
Программирование на стороне сервера включает в себя приложение (и внутренний язык программирования, который питает его), базу данных и сам сервер.

Программирование на стороне клиента.
Веб-разработка на стороне клиента (т. е. интерфейс) включает в себя все, что пользователи видят на своих экранах. Вот основные компоненты стека интерфейсных технологий:

-- Язык разметки гипертекста (HTML) и Каскадные таблицы стилей (CSS).
HTML сообщает браузеру, как отображать содержимое веб-страниц, в то время как CSS-стили этого содержимого.
Bootstrap, Wordpress, Joomla – это полезная платформа для управления HTML и CSS.

-- JavaScript (JS). JS делает веб-страницы интерактивными.
Существует множество библиотек JavaScript (таких как jQuery, React.js, и Zepto.JS) и фреймворки (такие как Angular, Vue, Backbone и Ember) для более быстрой и легкой веб-разработки.

Программирование на стороне сервера.
Серверная сторона не видна пользователям, но она питает клиентскую сторону, так же, как электростанция генерирует электричество для дома.
Основная проблема заключается в выборе серверных технологий для разработки веб-приложения.
Что касается серверных языков программирования, то они используются для создания логики сайтов и приложений.
Фреймворки для языков программирования предлагают множество инструментов для более простого и быстрого кодирования.
Отметим некоторые популярные языки программирования и их основные фреймворки (в скобках): Ruby (Ruby on Rails), Python (Django, Flask, Pylons), PHP (Laravel), Java (Spring), Scala (Play).
Node.JS, Среда выполнения JavaScript, также используется для внутреннего программирования.

Веб-приложению требуется место для хранения данных, для чего и используется база данных.
Существует два типа баз данных: реляционные и нереляционные (последний подразделяется на несколько категорий), каждая из которых имеет свои плюсы и минусы.
Вот наиболее распространенные базы данных для веб-разработки: MySQL (relational), PostgreSQL (relational), MongoDB (non-relational, document).

Веб-приложению требуется система кэширования для снижения нагрузки на базу данных и обработки больших объемов трафика.
Memcached и Redis являются наиболее распространенными системами кэширования.
Наконец, веб-приложению требуется сервер для обработки запросов от компьютеров клиентов.
В этой области есть два основных домена: Apache и Nginx.

Для разработки веб-приложения необходимо выбрать сервер, базу данных, язык программирования, фреймворк и инструменты интерфейса, которые будут использованы.
Эти технологии веб-разработки основаны друг на друге и, по сути, в совокупности называются стеком \cite{ualiev-jebegenov-issledovanie-1}.

\subsection{Анализ систем управления контентом}

Общий анализ текущей ситуации относительно систем управления контентом приведен в публикации работ \cite{kiyamov-cms} за авторством Киямова Р.В., Хмелева Е.А. и Юнусова И.Ф.

В настоящее время сайты настолько распространены, что практически у каждого третьего имеется свой сайт, начиная с личного блога и заканчивая своим интернетмагазином.
Но всегда есть трудности, с которыми встречаются пользователи при создании сайтов, независимо от того, это интернет-магазин или просто обычный новостной ресурс.
Главной проблемой в этом вопросе является выбор системы управления контентом, проще говоря, выбор CMS, потому что такие системы управления с большим набором возможностей могут позволить себе не все, также стоит учитывать удобный и простой интерфейс пользователя при использовании
такой системы.

CMS - это аббревиатура от Content Management System, что в дословном переводе означает «система управления контентом сайта» или просто «система управления сайтом».
Иногда CMS называют «движок» сайта \cite{kiyamov-cms-1}.
Пользователь, переплачивая за такие системы управления контентом, не всегда получает то, что хотел изначально.
Часто бывает, что пользователь не знает, какую систему управления контентом выбрать при создании своего интернет-ресурса, здесь важным фактором выступает еще и цена, потому что не каждый может позволить себе “CMS”, которая имела бы как простоту и легкость в использовании, так и максимально подходящий функционал для работы над ресурсом с выбранной тематикой.
Можно было бы выбрать что-то из бесплатных, например, Joomla или Wordpress, но, если выскочит какая-нибудь ошибка, связанная с работой сайта на уровне кода, то пользователь сам просто не сможет разобраться с проблемой.
Только на один поиск нужной информации он потратит кучу времени, также стоит отметить безопасность системы, бесплатные системы управления благодаря своей открытости легкодоступны для взломщиков.

Также стоит отметить одну довольно распространенную проблему, связанную с модулями на сайте.
Например, на системе управления контентом “Wordpress” существует модуль, с помощью которого можно реализовать сайт на двух и более языках, но проблема заключается в том, что данный модуль совместим лишь со старыми версиями движка, те, кто обновил свою “CMS”, просто потеряют возможность работы с данным модулем.

Все вышеописанные трудности и проблемы можно решить если не полностью, то частично, реализовав некую универсальную систему управления контентом, которая подошла бы как обычным блоггерам, так и для владельцев интернет-магазинов.
Такая система предусматривается как простая и удобная в использовании, любой новый пользователь сможет с легкостью разобраться в интерфейсе пользователя.
Главным отличием от всех существующих систем будет являться возможность подстраивания системы под определенного владельца, то есть при установке системы пользователь выбирает определенные параметры и тематику своего сайта, и система, исходя из полученных данных, сформирует сайт, максимально подходящий по заданным параметрам.

Например, в окне установки пользователь пишет тематику своего сайта, после выбирает сложность и объемность системы из трех предложенных вариантов, как только система принимает ответ, начинается обработка полученных данных по специально созданному алгоритму.
Первым делом просчитывается выбранная сложность системы, здесь, чем проще вариант, тем меньше возможностей и лишней нагрузки на сайт, далее, как только система подбирает определенный макет, происходит считывание значения из поля «тематика» и начинается подбор наиболее подходящего варианта, после которого система образует главную страницу и макетный шаблон по выбранным пользователем параметрам.
Например, если пользователь выбрал простой вариант сложности, а в тематику написал «блог», то система должна образовать для него сайт лишь с обычным набором возможностей под блог, простым и максимально подходящим дизайном, который в будущем сам пользователь сможет без лишних проблем изменять внешне по своему вкусу. 

Более конкретный анализ систем управления контентом проведен в работе \cite{ivanisheva-cms} А.А. Иванищевой Х.И. Комилова и М.Д. Гехаева.

Сегодня многие пользователи, которые хотят начать свой бизнес, создать портфолио или получить прибыль в интернете начинают раскрутку с создания своего собственного web-сайта.
Им может быть что угодно: блог, информационный сайт, интернет-магазин, сайт услуг, новостной и т.д.
Каждый хочет привлечь к себе больше внимания, посредством красивого оформления, выплывающих объявлений о товарах и услугах, показ последних новостей в разных сферах с сопровождением видео и фото материалов.
Первое, с чем сталкиваются пользователи, это цена сайта.
В основном цена зависит от следующих пунктов: доменное имя и хостинг, система управления контентом, функционал, дизайн web-сайта, контент, услуги специалиста.

Система управления контентом (CMS) – это система управления информацией или компьютерная программа, которая используется для обеспечения и организации процесса по совместному созданию, управлению и редактированию содержимого.
CMS в современном обществе называют движком сайта.
Существуют как платные, так и бесплатные платформы, рассмотрим самые популярные: WordРress, Joomla
и Drupal.

WordРress одна из популярных бесплатных простых платформ по созданию новостных сайтов. При
помощи WordРress можно создать собственный сайт всего за полчаса. Эта платформа имеет ряд простых
шаблонов для установки. Благодаря этому, даже без знания программирования пользователь может создать
интересующую его рубрику, блог, форум, вкладку с добавлением видео, аудио информации, имеет низкий
порог вхождения.
Помимо плюсов у этого движка есть и значительные минусы: требования к объему информации, большая нагрузка на сервер,, риск дублирования страниц, блокировка многих антивирусов к доступу на сайт, малый набор базовых функций.

Joomla вторая по списку популярности система управления контентом.
Она также бесплатна, как и WordРress, но имеет платные улучшенные по дизайну шаблоны, которые могут стать не просто одностраничным сайтом, но и социальной сетью из-за большого выбора плагинов, модулей и компонентов.
Данная CMS востребована, т.к. она обновляется практически каждые 3 месяца, и в связи с этим безопасна от взлома.
Но, как и любая система, Joomla имеет свои минусы: появление копий при редактировании материала страницы, ограниченность SEO, сложность в установке, редактировании и изучении.

Третья известная платформа – Drupal.
Она нацелена на большую аудиторию, подходит для создания крупных интернет-магазинов, используется как каркас для web-приложений, имеет открытый исходный код, очень высокий уровень безопасности системы.
Данная CMS интересна своей функциональностью, которая находится близко к универсальной.
Минусов у этой платформы немного: очень сложна в усвоении, не подходит для новичков, некорректная работа с БД,требователен к хорошему хостингу.

Исходя из перечисленных бесплатных популярных CMS можно сделать вывод, что каждая платформа универсальна в своей сфере применения.
Например, WordРress больше подходит новичкам, которые только начинают осваивать структуру и механику работы сайта.
Платформа Joomla рассчитана на более опытных специалистов в сфере IT, а Drupal на профессиональных разработчиков web-приложений. 

\subsection{Паттерны и фреймворки}

Обзор популярных фреймворков, применяемых в разработке современных веб-приложений проведен в статье \cite{sergacheva-framework} за авторством Сергачевой М.А. и Михалевской К.А.

Язык JavaScript уверенно набирает популярность среди frontend-разработчиков.
С помощью него можно реализовывать интересные решения интерфейсов, создавать интерактивные и эргономичные сайты, а также разрабатывать веб-приложения более просто и понятно.

Существует множество фреймворков, которые регулярно обновляются и совершенствуются.
И выбрать наиболее подходящий фреймворк оказывается нетривиальной задачей.

На начальном этапе создания проекта будет целесообразно произвести анализ, который поможет
выбрать подходящую технологию.
Фреймворк — это платформа, определяющая структуру вебприложения.
Его функционал позволяет осуществить взаимодействие веб-ресурса с сервером.

На веб-сайтах, которые написаны без использования фреймворка, первоначальный контент хранится на сервере.
Поэтому при загрузке нового материала на сайт требуется перезагрузка страницы.
Преимуществом же фреймворка являются неизменные блоки, которые сохраняются от одной
конфигурации к другой.
Это позволяет добиваться мгновенной обратной связи с пользователем при добавлении нового контента.

Данный принцип можно наблюдать при создании одностраничных веб-приложений, электронной
коммерции, облачных сервисов и многих социальных сетей.
Так, переход пользователя из одного раздела в другой осуществляется мгновенно.
Перегрузки страницы не происходит, так как неизменно остается каркас – постоянная часть проекта.

Преимущества использования фреймворка: фреймворки являются полностью бесплатными и имеют открытый исходный код, использование встроенных шаблонов помогает создавать проекты более высокого качества, при этом задействуется меньше строчек кода, высокая скорость разработки, которая достигается за счет открытого доступа к документации и множества форумов.

За реализацией проектов на классическом JS и HTML скрывается много трудностей, что приводит к появлению большого числа фреймворков на рынке.
Но наиболее популярными среди веб-разработчиков остаются Angular, React и Vue.

Angular — это кроссплатформенный фреймворк, разработанный компанией Google.
Он придерживается строгой иерархии и представляет собой большую инфраструктуру для комплексных решений.
Часть серверной службы фреймворка переносится на клиентскую сторону, что уменьшает нагрузку на сервер \cite{sergacheva-framework-1}.

Благодаря строгой типизации языка TypeScript, который используется на Angular, разработка
становится удобнее и проще для понимания.
При компиляции код переводится в JavaScript, то есть TS необходим только на этапе разработки.

К дополнительным преимуществам Angular следует отнести: доступную документацию, мощные инструменты для разработки, поддержку сообщества, актуальность, стабильность.

Слабой стороной фреймворка является высокий порог вхождения, поскольку нужно быть знакомым с подмножеством языка JS – TypeScript.
Также за большие возможности, предусмотренные в Angular, приходится платить нагрузкой на
производительность.

React – это библиотека функций с открытым исходным кодом, которая используется для разработки пользовательских интерфейсов \cite{sergacheva-framework-2}.
Стоит отметить, что данный инструмент будет считаться полноценным фреймворком лишь с подключением сторонних JavaScript библиотек.

Сильные стороны React: поддержка компании-разработчика;, высокая скорость работы, большое сообщество, кроссплатформенность, разработка UI на основе отдельных компонентов, технология Virtual DOM (высокая производительность).

Однако необходимость в сторонних библиотеках для работы делает процесс разработки запутаннее.
Отсутствуют стандарты в написании кода на HTML и CSS, что есть у его конкурентов – Angular и Vue.JS.
Приходится прибегать к дополнительным плагинам, поскольку не все стандартные браузеры поддерживают этот фреймворк \cite{sergacheva-framework-3}.

Имеется доступ к управлению низкоуровневым функционалом, но для новичков это скорее минус.
Поэтому React подходит для более опытных разработчиков.

Vue.js — это прогрессивный JavaScript фреймворк с открытым исходным кодом, который может использоваться и как библиотека.
Он прост в освоении, в отличие от вышеупомянутых фреймворков, при этом по производительности не уступает React.

Примечательно то, что за созданием данного инструмента стоит всего один талантливый разработчик – Эван Ю.
Из-за чего большими корпорациями данный инструмент воспринимается с долей скептицизма.
Тем не менее Vue.js пользуется спросом среди китайских компаний, в частности всем известная Xiaomi.

Данный фреймворк преимущественно отвечает за представление, тем самым позволяя упростить
взаимодействие с другими проектами и библиотеками.
Кроме того, Vue.js больше подходит для небольших проектов \cite{sergacheva-framework-4}.
Сильные стороны: скорость, небольшой вес, лаконичность, приверженность стандартам HTML, CSS, поддержка TS, JSX, низкий порог входа.

Слабыми местами являются недостаточно большое сообщество и отсутствие структуры.
Однако репутация Vue.js со временем растет.


Подходы в использовании определенного набора технологий в профессиональной среде называются "стек" (от английского stack - куча, кипа).
Само название стека зачастую определяется его содержимым - набором инструментов, технологий или операционных систем, обычно отвечающим  задачам формирования клиентской части приложения, взаимодействия сторон клиента и сервера, хранения и взаимодействия с базой данных другим, в зависимости от области и специфики применения.


Подробный разбор веб-стеков приведен в статье Давыдовского М.А. "Выбор веб-стека для реализации цифровой среды предоставления транспортных услуг" \cite{davidovsky-vibor}.

%LAMP

LAMP стек \cite{davidovsky-vibor-1} использует операционную систему Linux (L), веб-сервер Apache (A), базу данных MySQL (M) и язык PHP (P) \cite{davidovsky-vibor-php}.
В настоящее время это самый популярный стек для разработки динамических сайтов и веб-приложений.

Имеются следующие модификации этого стека: 

-- LEMP стек вместо Apache использует nginх;

-- LLMP стек вместо Apache использует lighttpd. Lighttpd -- это быстрый защищенный веб-сервер, который работает как в операционной системе Linux, так и Microsoft Windows;

-- LAPP стек вместо MySQL использует PostgreSQL;

-- WAMP стек вместо Linux использует операционную систему Microsoft Windows;

-- XAMPP стек может работать как в Linux, так и в Windows и использует язык PHP или Perl.

% WISA

WISA стек использует только продукты фирмы Microsoft: операционную систему Windows, веб-сервер Internet Information Services (IIS), систему управления базами данных Microsoft SQL Server, платформу для разработки веб-приложений ASP.NET.

Платформа ASP.NET позволяет писать код приложения на разных языках, входящих в комплект .NET Framework.
Это языки: C\#, VisualBasic.NET, J\#, JScript.NET и другие.
Эта платформа использует технологию AJAX и шаблон проектирования MVC (Model-View-Controller), который позволяет разделить модель данных, внешнее представление и управление данными.

% MAMP

MAMP стек – это стек, работающий в операционной системе macOS фирмы Apple.
Этот стек включает веб-сервера Apache и nginх, систему управления базами данных MySQL, языки программирования PHP, Perl и Python.
Данный стек позволяет работать с системой управления контентом WordPress.

Серверная платформа Node \cite{davidovsky-vibor-NODE} использует скриптовый язык программирования JavaScript.
Она не требует использования Apache или другого веб-сервера, т.к. выполняет его функции.
Поэтому в названии стеков с этой платформой не указывается веб-сервер, а на первом месте стоит система управления базами данных.

% MEAN

MEAN стек \cite{davidovsky-vibor-mean} использует следующие программы:

-- MongoDB (M) -- NoSQL-система управления базами данных, хранящая данные в формате
«ключ-значение»;

-- Express (E) -- фреймворк для разработки серверной части веб-приложений на платформе Node
[1]. Он является стандартным каркасом для разработки приложений на Node и содержит ряд методов, упрощающих код на Node;

-- AngularJS (A) -- фреймворк для разработки клиентской части одностраничных веб-приложений на платформе Node.
Он включает функции обработки данных JSON-формата, использует MVC-шаблон проектирования веб-приложения, позволяет разработать веб-приложение, которое переносит часть нагрузки по обработке данных со стороны сервера на сторону клиента;

-- Node (N) -- платформа разработки серверных приложений.

Модификациями стека MEAN являются:

-- MERN стек, в котором AngularJS заменен на React.
React -- библиотека на JavaScript для разработки пользовательских интерфейсов одностраничных и мобильных приложений;

-- MEEN стек, в котором AngularJS заменен на Ember.js.
Ember.js -- это фреймворк, предоставляющий каркас для разработки веб-приложений с использованием MVC-шаблона.

\begin{comment}
\subsection{Обзор Web-ресурсов для учебных центров}
    Для выполнения работы по созданию веб-ресурса учебного центра можно опираться на уже представленные сайты других организаций.

    Для определения технической стороны можно воспользоваться сервисами-анализаторами, например приложением компании iTrack \cite{iTrack}. 
    При использовании данного сервиса можно определить, что сайт вологодской компании А-Элита \cite{aelita} (Главная страница сайта отображена на Рисунке \ref{aelita}) использует WordPress.

    \addimghere{images/aelita.png}{0.6}{Главная страница сайта компании А-Элита}{aelita}

    Результат проверки отображен на Рисунке \ref{aelita-tr}.

    \addimghere{images/aelita-tr.png}{0.6}{Результат проверки сайта компании А-Элита}{aelita-tr}

    В целом анализ содержимого сайта можно свести к двум выводам.
    Первый - используется CMS WordPress.
    Второй - главная страница представляет собой страницу-лэндинг с размещенными на ней ссылками, перенаправляющими либо на сторониие ресурсы, либо на разделы этой же страницы.

    Пример части главной страницы расположен на Рисунке \ref{aelita2}.

    \addimghere{images/aelita2.png}{0.6}{Часть главной страницы компании А-Элита}{aelita2}

    Данный подхо прост в реализации и позволяет реализовать все необходимые требования не прибегая к созданию множества страниц и переходов между ними.

    Еще одним примером учебного центра будет Учебный центр "Энергетик" \cite{energy}.
    Результат (отображен на Рисунке \ref{energy-tr} его проверки в сервисе компании iTrack - использование CMS Drupal.

    \addimghere{images/energy-tr.png}{0.6}{Результат проверки сайта Учебный центр "Энергетик"}{energy-tr}

    При анализе внешнего вида веб-ресурса можно выявить, что здесь реализуется уже многостраничная система сайта, обрабатываемая собственно CMS Drupal.
    На Рисунке \ref{energy} отображен внешний вид главной страницы с раскрытым меню переходов.

    \addimghere{images/energy.png}{0.6}{Главная страница сайта Учебный центр "Энергетик"}{energy}.

\end{comment}

\clearpage
