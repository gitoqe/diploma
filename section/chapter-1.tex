\section{Аналитический обзор информационного обеспечения учебных центров}


\subsection{ИТ-управление учебным центром: цели, задачи и функции}

В статье компании "Инлайн Груп Центр" \cite{inlinegroup-c} дается определение учебного центра как организации, специализирующаяся на предоставлении услуг обучения в очной форме.

Исходя из данного определения, описываются базовые бизнес-процессы учебного центра, такие как: учет ресурсов, использующихся для обучения, ведение графиков доступности ресурсов и инструкторов, составление расписаний с учетом всех действующих ограничений и графиков доступности ресурсов, учет фактически проведенного обучения.

Также приводятся примеры положительного влияния автоматизации деятельности, а именно: значительное повышение качества и скорости планирования, более эффективное использование имеющихся у учебного центра ресурсов, оперативное реагирование на изменения спроса на обучение или выбытие ресурсов путем гибкого перепланирования.

На основе приведенных данных указан выработанный компанией подход к реализации проекта, состоящий из следующих пунктов: проведение анализа бизнес-процессов и разработка Концепции их автоматизации, настройка программного решения, обучение администраторов Учебного центра, разработка и осуществление плана ввода информационной системы управления Учебным центром в действие.

Описание продукта "1С:Управление учебным центром" на официальном ресурсе фирмы "1С" \cite{1c-training-center} содержит перечисление основных функциональных возможностей, а именно: планирование учебного процесса, составление расписания курсов и занятий, проведение набора слушателей, ведение взаиморасчетов, управление движением контингента слушателей, учет учебного процесса, учет платного питания, сопутствующие торговые операции, заработная плата, настройка конфигурации.

Из приведенных данных от компаний "1С" и "Инлайн Груп Центр" можно выделить некоторые общие пункты, изначально проектируемые для инфраструктуры ИТ-управления учебным центром, на основе которых выстраивается полный функционал итогового решения.


\subsection{Типы сайтов и обзор средств разработки сайтов}

В статье интернет-агентства «Студия Вячеслава Денисова» \cite{denisov} приводится категоризация сайтов по функциональности, отзывчивости дизайна, способу формирования контента и типу контента.

% по функциональности

Первое разделение по функциональности выделяет множество групп.

Простые в техническом исполнении сайты-визитки, применяемые в небольших компаниях.

Коммерческие сайты со сложным внутренним функционалом.
Частым представителем в последние десятилетия является пример интернет-магазина и присущие ему возможности.
Как правило уклон здесь делается на возможность сравнения, оценки и оплаты различных продуктов и услуг.

Порталы или интернет-порталы, являющиеся окном в самый широкий спектр тем.
Зачастую такие сайты являются точкой старта для работы с сетью и включает обширные возможности быстрого и настраиваемого поиска среди проиндексированных веб-ресурсов.
В пример приводится Яндекс.

Информационные сайты или вики-сайты.
Построенные по типу Википедии, обширные картотеки информации самого разного плана и характера.
Функционал часто связан с возможностью пользователей создавать и модерировать контент.
Среди наиболее популярных сайтов этой категории -- Википедия.

Социальные сети.
Активно развивающиеся веб-ресурсы, обладающие высокой сложностью и масштабом.
С помощью них реализуются возможности обмена информацией с окружающими людьми.
Также являются крайне привлекательными со стороны рекламодателей и создателей контента, размещающих у себя рекламу.

% по отзывчивости дизайна

Второе разделение сайтов, предлагаемое в данной статье -- по отзывчивости дизайна.
Удобство пользования ресурсом диктует свои требования формированию его содержимого.
В статье разделяют четыре категории.

Статические макеты, как самые старые, но стабильные представители веб-ресурсов.
Из минусов -- жесткая привязка к конкретному целевому устройству и его характеристикам.
Использование же такого сайта на других платформах может оказаться весьма затруднительным и неудобным.

Следующая категория -- жидкий дизайн, подстраивающийся под целевое устройство.
Настройки содержимого веб-ресурса самостоятельно изменяются в зависимости от используемой платформы, что позволяет добиться одинакового отображения.

Адаптивный дизайн является еще вариантом для решения уже обозначенных проблем.
Особенность работы сайтов данной категории скрыта в медиа-запросах к целевому устройству и реагированию на полученные данные.
Такой подход позволяет заранее задать некоторое количество размеров экрана и подготовить контент под каждый из них.

И последней категорией в данном разделе можно назвать комбинацию предыдущих -- отзывчивый дизайн.
Разработчик попросту использует описанные выше подходы и получает оптимальный вариант работы для большинства целевых устройств.

% по способу формирования контента

Третья группировка сайтов, предлагаемая в статье, берет за основу способ формирования контента.
В данном случае веб-ресурсы подразделяются на две подкатегории.

Первая -- статические веб-ресурсы.
Отличаются простотой структурой и подачей контента.
Они статичны -- не подразумевают активного взаимодействия между пользователем и веб-ресурсом, что приводит к упрощению всего сайта и применяемых в нём технологий.

Вторая -- динамические сайты.
Эта категория сложнее в принципах работы и реализации.
Основная цель -- предоставление интерактивности в взаимодействии пользователя с сайтом.
Например -- предоставление обратной связи, возможности пользоваться встроенным функционалом, изменять контент.
Достигается это благодаря усложнению набора применяемых технологий и, как следствие, увеличении сроков разработки.

% по типу контента

Четвертый способ категоризации сайтов -- по типу контента.
В данном случае само понятие обозначает именно способ подачи содержимого веб-ресурса, а не только тематическое направление.
В статье выделяются как минимум 7 типов.
К ним относятся блоги, социальные сети, агрегаторы потокового видео, корпоративные сайты, краудфандинговые площадки, сайты электронной коммерции, информационные сайты и сайты новостей.

% обзор средств разработки сайтов

В публикации журнала "Вестник науки" за авторством Уалиева Н.С. и Жебегенова Ә.М. \cite{ualiev-jebegenov-issledovanie} проводится следующий обзор технологий разработки сайтов.

В статье ведется рассуждение о важности выбора технологического стека и серьезных последствиях при допущении возможной ошибки на данном этапе.
Здесь задействованы как финансовые, так и временные соображения, что отдельно отмечается применительно к бюджетным и образовательным учреждениям.
Также выделяются критерии для выбора оптимального стека технологий с учетом серверной и клиентской веб-разработки.

Для программирования на клиентской стороне приводятся примеры таких технологий как язык разметки гипертекста (HTML) и каскадные таблицы стилей (CSS) для передачи браузеру инструкций по отображению содержимого сайта и его оформления.
Также упоминается язык программирования JavaScript, позволяющий расширить возможности веб-ресурса интерактивными элементами.

Серверная же сторона веб-разработки раскрывается через указание популярных языков программирования и присущим им фреймворкам.
Например, встречаются такие сочетания как PHP и Laravel, Java и Spring, и Python и Django или Flask соответственно.

Также выделяется необходимость нахождения в стеке технологии, ответственной за работу данными, их хранением и обработкой.
Как правило для таких задач применяются базы данных.
Статья содержит упоминания как реляционных, так и нереляционных баз -- MySQL и MongoDB.

Помимо перечисленных пунктов, в публикации упоминается необходимость наличия систем кэширования -- Memcached и Redis, а также сервер обработки запросов пользователей -- Nginx и Apache.

Как итог выводится перечень нужных для принятия решений касательно применяемого программного обеспечения.
Сюда входят такие пункты как база данных, сервер, язык программирования и привязанный к нему фреймворк.

Данный набор технических средств, плотно связанный между собой называется технологическим стеком \cite{ualiev-jebegenov-issledovanie-1} веб-приложения.


\subsection{Анализ систем управления контентом}

Общий анализ текущей ситуации относительно систем управления контентом приведен в публикации работ за авторством Киямова Р.В., Хмелева Е.А. и Юнусова И.Ф. \cite{kiyamov-cms}.

Начало статьи положено через определение CMS или Content Management System -- система управления контентом, также иногда именуемой <<движком>> сайта \cite{kiyamov-cms-1}.

В тексте работы говорится о широком распространении сайтов, простоте их создания даже без наличия специализированных знаний и умений.
Но даже в таком случае возникает проблема определения инструмента, подходящего для наиболее широкого спектра пользователей.
Как правило, этим инструментом является система управления контентом, выбор которой повлияет на дальнейшие пути взаимодействия с получившимся веб-ресурсом.

Важное замечание касается соотношения цена-качество при выборе CMS, ведь не каждый пользователь или организация может позволить себе покупку мощной и сложной системы с обширным спектром возможностей и качественной поддержкой.
Из-за чего вполне разумным является переход к рассмотрению бесплатных вариантов.
Среди примеров -- Joomla и Wordpress.
Хоть их использование бесплатно, но встречающиеся ошибки, а также проблемы в настройке могут потребовать весьма широкого спектра умений и навыков применительно к информационным технологиям.

Также важно учитывать полную зависимость от разработчика системы управления контентом.
Зачастую при больших обновлениях и добавлении новых возможностей теряется часть устаревшего функционала, что вынуждает либо отказываться от "выпавших модулей", либо переписывать их под новые возможности платформы, что, опять же, доступно далеко не каждому пользователю.

Более конкретный анализ систем управления контентом проведен в работе А.А. Иванищевой Х.И. Комилова и М.Д. Гехаева \cite{ivanisheva-cms}.

Согласно работе, входной точкой для открытия бизнеса в текущее время обычно становится решение о создании собственного веб-ресурса.
Процесс проектирования может обозначить целевой формат сайта в виде блога, интернет-магазина или другого, в зависимости от потребностей заказчика.

Условия конкуренции диктуют свои требования к необычному, красивому оформлению, наличию интерактива с пользователем, показ новостей и медиаматериалов по тематике ресурса.
При этом первичным вопросом, с которым приходится столкнуься, является стоимость разработки.
В данный пункт можно закладывать достаточно большое количество трат.
Например, доменное имя может быть приобретено за сумму в пределах 300 рублей (по состоянию на 2022 год) в зависимости от уникальности требуемого наименования, а также верхнего уровня домена.
Следующие статьи расходов -- хостинг, который будет отвечать за доступ сайта из сети интернет, приобретаемая CMS, оплата разработки дизайна, услуги специалистов для разработки и поддержания работы веб-ресурса.

В работе дается определение CMS.
Систему управления контентом можно охарактеризовать как ответственную за процесс совместного создания, управление и редактирования содержимого ресурса.
CMS подразделяются на платные и бесплатные.

В работе рассматриваются популярные бесплатные системы управления контентом -- Drupal, WordРress и Joomla.

Первой приводится WordРress, которая по праву может называться одной из популярных простых платформ по созданию сайтов, при этом являющейся бесплатной.
Лучшей характеристикой простоты можно назвать временные затраты на полноценное создаение веб-ресурса -- около получаса.
Также в работе отмечается широкий спектр бесплатных простых шаблонов, открывающих возможности по приспособлению внешнего облика сайта поставленной тематике.
Из существенных минусов этого движка выделяются следующие: небольшой набор базовых функций, высокие требования к производительности оборудования сервера, риск дублирования страниц и некоторые другие.

Следующим идет Joomla.
Данную CMS тоже можно назвать весьма популярной и широко применяемой.
Joomla бесплатна, как и WordРress, но обладает некоторыми отличиями.
В этой системе есть платные шаблоны с улучшенным дизайном, которые можно модифицировать с помощью встроенных компонентов, модулей и плагинов.
Joomla часто обновляется, что обеспечивает безопасность от возможного взлома.
Из минусов можно выделить: сложность процесса установки, изучения и редактирования, а также ограниченность системы поисковой оптимизации.

Последним в статье рассматривается Drupal.
Он бесплатен, а его исходный код открыт.
Сфера применения обширна, но особенно выделяются интернет-магазины и возможность применения как каркаса для web-приложений.
Из важных минусов Drupal: высокая сложность овладения, некоторые сложности в работе с базой данных и высокие требования к производительности оборудования сервера, либо хостинга.

Выводом по данной статье можно заявить позицию. что каждая CMS хороша применительно к своей сфере задач и команде разработки, которая готова с ней взаимодействовать.


\subsection{Паттерны и фреймворки}

Обзор популярных фреймворков, применяемых в разработке современных веб-приложений проведен в статье за авторством Сергачевой М.А. и Михалевской К.А \cite{sergacheva-framework}.

В данной работе справедливо отмечается сложность создания проектов с использованием возможностей классического JavaScript и HTML, из-за чего возникают технологические решения -- фреймворки для веб-разработки, среди которых наибольшую популярность обрели Angular, React и Vue.

Первым разобранны в статье является Angular, разработанный компанией Google.
Одной из важных особенностей является то, что часть серверной службы фреймворка переносится на клиентскую сторону и, по итогу, уменьшает нагрузку на сервер \cite{sergacheva-framework-1}.

Angular использует строго типизированный TypeScript, что позволяет добавить в лучший контроль над происходящим в программе на этапе разработки.

Следующим разбирается React, являющийся библиотекой, а не фреймворком.
Он позволяет вести разработку пользовательских интерфейсов \cite{sergacheva-framework-2}.
Из плюсов можно выделить скорость работы, активное сообщество, поддержку множества платформ.

Vue.js последний из рассматриваемых в данной статье фреймворков.
Это прогрессивный JavaScript фреймворк с открытым исходным кодом.
Vue.js проще в освоение, по сравнению с React или Angular.
Из-за низкого порога входа, а также легкости и лаконичности, Vue.js больше подходит для небольших проектов \cite{sergacheva-framework-4}.

Подходы в использовании определенного набора технологий в профессиональной среде называются "стек" (от английского stack -- куча, кипа).
Само название стека зачастую определяется его содержимым -- набором инструментов, технологий или операционных систем, обычно отвечающим  задачам формирования клиентской части приложения, взаимодействия сторон клиента и сервера, хранения и взаимодействия с базой данных другим, в зависимости от области и специфики применения.

Подробный разбор стеков для веб-разработки приведен в статье "Выбор веб-стека для реализации цифровой среды предоставления транспортных услуг" за авторством М.А. Давыдовского \cite{davidovsky-vibor}.

%LAMP
В работе рассматриваются такие стеки, как LAMP, WISA, MAMP и MEAN.
Приводятся их состав и характеристики.

Первый стек -- LAMP \cite{davidovsky-vibor-1}.
Состоит из следующих ключевых элементов: операционная система Linux, веб-сервер Apache, база данных MySQL, язык программирования PHP \cite{davidovsky-vibor-php}.
У данного стека есть множество модификаций и подвидов, отличающихся по составным элементам.

% WISA

Следующий рассматриваемый в статье набор технологий -- WISA.
Данный стек нацелен на компанию Microsoft и её продукты.
Сюда входят операционная система Windows, веб-сервер Internet Information Services, СУБД Microsoft SQL Server и платформа для разработки веб-приложений ASP.NET, позволяющая использовать различные языки программирования -- C\#, VisualBasic.NET, J\#, JScript.NET и другие.

% MAMP

Третий на очереди стек -- MAMP.
В отличии от WISA, применителен к операционной системе macOS от компании Apple.
Как составные части здесь используются веб-сервер Apache и nginх, СУБД MySQL, языки программирования PHP, Perl и Python.

Отдельно в статье выделяется платформа Node \cite{davidovsky-vibor-NODE} применяемая для серверной разработки, выполняющая функции в том числе веб-сервера.
Как следствие название стеков, основанных на этой платформе, не содержит отдельного веб-сервера.
Она применяет скриптовый язык программирования JavaScript.

% MEAN

Следующим разобранным в статье стеком является MEAN \cite{davidovsky-vibor-mean}.
Данный набор технологий относится к использующим Node.js, а значит к ним относится ранее обозначенная особенность такого применения.
MEAN состоит из MongoDB в роли СУБД, Express.js -- фреймворка для разработки серверной части веб-приложений на платформе Node, Angular для клиентской части и непосредственно Node.js.

Любой стек может обладать различными модификациями, получающимися в результате замены составных частей.
Например, в стеке MEAN можно задействовать другое средство работы с клиентской частью.
В результате замены Angular на React получится MERN стек, а при замене на Ember.js -- MEEN.


\subsection{Обзор Web-ресурсов для учебных центров}

Подробный обзор особенностей web-ресурсов, а также специфику их модернизации приводит в своей работе Д.А. Слинкин \cite{slinkin-sovremennie}.

В статье указывается, что модернизация и поддержка веб-ресурса зачастую завязана не только на техническом совершенствовании, но и улучшении дизайна.
При этом эти элементы являются взаимодополняющими.
Редизайн нацелен на сохранение привычного опыта использования у уже имеющихся пользователей, но при этом и получение новых впечатлений и знаний у пришедших впервые.
Техническая сторона касается всех видов веб-ресурсов, позволяет производить и визуальные изменения за счет усовершенствования техники и технологий, но при этом является скрытой от глаз клиентов.

Современные технологии, как правило, требуют более совершенных технических средств.
Без проведения тщательной технической модернизации могут возникнуть непредвиденные накладки по времени и денежным затратам.

В свою очередь можно выделить и обратную зависимость.
Техническая модернизация может вызвать изменение дизайна сайта.
Примером может послужить обновление используемой системы управления контентом, приводящее к отключению или изменению старых функций, используемых на веб-ресурсе.
Такая ситуация приведет к нарушению работы существующего дизайна и потребует доработки в сложившихся условиях.

Более весомый пример -- внедрение новых языковых фреймворков с полной или частичной заменой старых.
Такие манипуляции как правило приводят к изменению кодовой базы и большим трудовым затратам. 

Также в данной работе Д.А. Слинкин рассматривает этапы, проходимые в процессе модернизации веб-ресурсов.
Одним из первых стоит выполнить анализ состояния перерабатываемого веб-ресурса.
Уже после разбора текущего состояния можно приступить к формирование технического задания, по которому будут ориентироваться исполнители и проверяющие.

После составления технического задания идет этап непосредственного технического улучшения.
Здесь необходимо тщательно провести подготовку технологических возможностей веб-ресурса для упрощения дальнейшей разработки.

В данный пункт можно отнести сразу несколько важных шагов.
Первый из них -- проведение смены аппаратного обеспечения, смену или обновление используемой операционной системы, переход на другой тарифный план или хостинг.

Следующим шагом будет обновление сетевого и системного программного обеспечения -- веб-серверов и прочих сопутствующих серверов, применяемых для работы веб-ресурса.

Следом требуется провести переход вспомогательного программного обеспечения на новые версии.
Зачастую крупные проекты могут обрастать сторонними программами, участвующими в некоторых шагах основной деятельности.
Это могут быть конвертеры данных, системы аутентификации и журналирования или другие полезные элементы системы.

При наличии используемой CMS, требуется обратить внимание и на неё.
Зачастую системы управления контентом обновляются и обрастают новым полезным функционалом.
При этом важно учитывать вероятность отказа старых модулей веб-ресурса ввиду отсутствия обратной совместимости, о чем было упоминалось ранее.

Теперь, проведя обновление технических компонентов обеспечения, можно перейти непосредственно к модернизации внешнего облика веб-ресурса.
При этом данный этап идет последним из-за высокой зависимости от прохождения предыдущих.
В случае сильного изменения технического обеспечения веб-ресурса, может потребоваться внести большой объем правок в существующий облик ресурса или вовсе создать его с начала.
При этом потребуется экспертиза задействованных сотрудников в примененных технологиях или, возможно, найм и обучение новых.

Для выполнения работы по созданию веб-ресурса учебного центра можно опираться на уже представленные сайты других организаций.
Также стоит учитывать местоположение, а следовательно сравнение проводить среди локальных компаний, занятых в данной сфере деятельности.
Для определения технической стороны можно воспользоваться сервисами-анализаторами, например, бесплатным приложением компании iTrack \cite{iTrack}.

С помощью данного сервиса можно определить, что сайт вологодской компании А-Элита \cite{aelita} (Главная страница сайта отображена на рисунке \ref{aelita}) использует систему управления контентом WordPress.

\addimghere{images/examples/aelita.png}{0.6}{Главная страница сайта компании А-Элита}{aelita}

Результат проверки отображен на рисунке \ref{aelita-tr}.

\addimghere{images/examples/aelita-tr.png}{0.6}{Результат проверки сайта компании А-Элита}{aelita-tr}

Пример части главной страницы расположен на рисунке \ref{aelita2}.

\addimghere{images/examples/aelita2.png}{0.6}{Часть главной страницы компании А-Элита}{aelita2}

Еще одним примером будет веб-ресурс Учебного центра "Энергетик" \cite{energy}.
Результат его проверки (отображен на рисунке \ref{energy-tr}) в сервисе компании iTrack -- использование CMS Drupal.

\addimghere{images/examples/energy-tr.png}{0.6}{Результат проверки сайта Учебный центр "Энергетик"}{energy-tr}

На рисунке \ref{energy} отображен внешний вид главной страницы с раскрытым меню переходов.

\addimghere{images/examples/energy.png}{0.6}{Главная страница сайта Учебный центр "Энергетик"}{energy}.

% Принимая во внимание указанные в статье \cite{slinkin-sovremennie} Слинкина Д.А. требования к проводимой модернизации веб-ресурса, а также анализ 

\clearpage
