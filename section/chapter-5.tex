\section{РАЗРАБОТКА ДОКУМЕНТАЦИИ ДЛЯ АДМИНИСТРАТОРА И ПОЛЬЗОВАТЕЛЕЙ WEB-РЕСУРСА}

\subsection{Руководство по использованию веб-ресурса пользователем}

Загрузка сайта направляет пользователя на главную страницу.
В закрепленной шапке веб-ресурса расположены все основные разделы, которые можно посетить.
В левом-верхнем углу находится логотип компании Мезон, ведущий на её официальный сайт, рядом с ним находится логотип учебного центра Мезон, ведущий на домашнюю страницу ресурса.

На домашней странице расположен блок кнопок с пояснениями, дублирующий разделы из шапки сайта.
Пользователь имеет возможность ознакомиться с кратким описанием и выбрать подходящий пункт.
Данная страница отображена на рисунке \ref{result-homepage}.

\addimghere{images/examples/result-homepage.png}{0.8}{Домашняя страница разработанного портала}{result-homepage}

В разделе <<Поступление>> пользователь может ознакомиться с правилами записи на курсы учебного центра.
Раздел <<Курсы>> представляет клиенту набор имеющихся для проведения курсов обучения с возможностью подробного рассмотрения особенностей каждого из них в отдельности.
Раздел <<Расписание>> включает в себя расписание занятий групп на учебный год, позволяет оперативно сориентировать клиентов касательно выходных дней в организации.
Раздел <<Работы учеников>> представляет клиентам возможность ознакомиться с выставленными работами студентов разных направлений.
Раздел <<Об организации>> содержит информацию о компании, руководстве, а также обширный набор вспомогательной документации, полезной для ознакомления.
Раздел <<Оплата и аренда>> содержит информацию о способах оплаты услуг учебного центра и предоставляемых в аренду помещениях.

В случае, когда необходимая информация не была найдена, пользователь может воспользоваться контактными данными.
Подвал ресурса содержит все необходимые ссылки, адреса и номера телефонов.
Данный компонент отображается на всех страницах веб-ресурса.
Изображение подвала сайта содержится в рисунке \ref{result-footer}.

\addimghere{images/examples/result-footer.png}{0.7}{Нижняя часть страниц ресурса с контактной информацией учебного центра}{result-footer}

\subsection{Руководство по использованию веб-ресурса администратором}

С точки зрения администратора применительно и руководства для пользователя.
Отличием является лишь предоставление дополнительных возможностей.
Для администратора открыта возможность взаимодействия с исходными файлами ресурса, базой данных.
Помимо этого, для администратора можно завести учетную запись, с помощью которой он может воспользоваться панелью изменения контента сайта.

Для получения доступа редактированию к информации, отображаемой на веб-ресурсе, необходимо в адресной строке после названия сайта написать <<secretlogin>>.
В результате будет произведен переход на страницу логина, отображенную на рисунке \ref{result-secretlogin}.

\addimghere{images/examples/result-secretlogin.png}{0.7}{Страница логина}{result-secretlogin}

Здесь необходимо в подписанных полях <<Username>> и <<Password>> ввести известные данные для входа и нажать на кнопку <<Login>>.
После этого будет проведена проверка переданных значений и, если они оказались корректны, пользователь будет аутентифицирован, произойдет переход на страницу <<Редактор содержимого>>, отображенную на рисунке {result-contenteditor}.

\addimghere{images/examples/result-contenteditor.png}{0.7}{Страница редактора контента}{result-contenteditor}

На данной странице представлены переключатели доступных разделов.
При переходе по ним изменяется содержимое основной части страницы -- таблицы с содержимым из базы данных.
Используя расположенные в первом столбце каждой строки пиктограммы редактирования, сохранения и удаления.
Администратор может активировать режим взаимодействия с содержимым строки с целью его изменения путем нажатия на пиктограмму с изображением листа и ручки.

По завершению редактирования необходимо нажать на пиктограмму облака для сохранения нового состояния строки.

Для удаления строки необходимо нажать на пиктограмму корзины.
При этом будет выдано всплывающее окно с вопросом о подтверждении удаления строки.

Для выхода из режима редактирования достаточно нажать на большую красную кнопку <<Logout>> после чего будет разорван сеанс администратора и произойдет переход на домашнюю страницу веб-ресурса.

\clearpage
