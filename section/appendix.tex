\phantomsection
\addcontentsline{toc}{subsection}{ПРИЛОЖЕНИЕ 1 Исходный код файла content-editor.guard.ts}
\begin{center}
    ПРИЛОЖЕНИЕ 1\\
    %\vspace*{-1cm}
    (обязательное)


    Исходный код файла content-editor.guard.ts
\end{center}

\begin{lstlisting}
    import { Injectable } from '@angular/core';
    import { ActivatedRouteSnapshot, CanActivate, RouterStateSnapshot, Router, UrlTree } from '@angular/router';
    import { Observable } from 'rxjs';
    import { AuthService } from '../services/auth.service';
    @Injectable({
        providedIn: 'root'
    })
    export class AuthGuard implements CanActivate {
        constructor (private _router: Router, private _auth: AuthService) {}
        canActivate(
            route: ActivatedRouteSnapshot,
            state: RouterStateSnapshot): Observable<boolean | UrlTree> | Promise<boolean | UrlTree> | boolean | UrlTree {
            console.log('@ AuthGuard canActivate: ')
            if (this._auth.isLoggedIn()) {
                this._router.navigate(['/contenteditor']);
            }
            return !this._auth.isLoggedIn();
        }
    }
\end{lstlisting}
\clearpage

\phantomsection
\addcontentsline{toc}{subsection}{ПРИЛОЖЕНИЕ 2 Исходный код файла auth.guard.ts}\label{на-приложение-2}
\begin{center}
    ПРИЛОЖЕНИЕ 2\\
    %\vspace*{-1cm}
    (обязательное)


    Исходный код файла auth.guard.ts
\end{center}

\begin{lstlisting}
    import { Injectable } from '@angular/core';
    import { ActivatedRouteSnapshot, CanActivate, RouterStateSnapshot, Router, UrlTree } from '@angular/router';
    import { Observable } from 'rxjs';
    import { AuthService } from '../services/auth.service';
    @Injectable({
        providedIn: 'root'
    })
    export class AuthGuard implements CanActivate {
        constructor (private _router: Router, private _auth: AuthService) {}
        canActivate(
            route: ActivatedRouteSnapshot,
            state: RouterStateSnapshot): Observable<boolean | UrlTree> | Promise<boolean | UrlTree> | boolean | UrlTree {
            console.log('@ AuthGuard canActivate: ')
            if (this._auth.isLoggedIn()) {
                this._router.navigate(['/contenteditor']);
            }
            return !this._auth.isLoggedIn();
        }
    }
\end{lstlisting}
\clearpage

\addcontentsline{toc}{subsection}{ПРИЛОЖЕНИЕ 3 Исходный код файла token.interceptor.ts}\label{на-приложение-3}
\begin{center}
    ПРИЛОЖЕНИЕ 3\\
    %\vspace*{-1cm}
    (обязательное)


    Исходный код файла token.interceptor.ts
\end{center}

\begin{lstlisting}
import { Injectable } from '@angular/core';
import { HttpRequest, HttpHandler, HttpEvent, HttpInterceptor, HttpErrorResponse } from '@angular/common/http';
import { AuthService } from '../services/auth.service';
import { Observable, throwError, BehaviorSubject } from 'rxjs';
import { catchError, filter, take, switchMap } from 'rxjs/operators';
@Injectable()
export class TokenInterceptor implements HttpInterceptor {
    private isRefreshing = false;
    private refreshTokenSubject: BehaviorSubject<any> = new BehaviorSubject<any>(null);
    constructor(public _auth: AuthService) { }
    intercept(request: HttpRequest<any>, next: HttpHandler): Observable<HttpEvent<any>> {
        if (this._auth.getJwtToken()) {
            request = this.addToken(request, this._auth.getJwtToken());
        }
        return next.handle(request).pipe(catchError(error => {
            if (error instanceof HttpErrorResponse && error.status === 401) {
                return this.handle401Error(request, next);
            } else {
                return throwError(error);
            }
        }));
    }
    private addToken(request: HttpRequest<any>, token: string | null) {
        return request.clone({
            setHeaders: {
                'Authorization': `Bearer ${token}`
            }
        });
    }
    private handle401Error(request: HttpRequest<any>, next: HttpHandler) {
        if (!this.isRefreshing) {
            this.isRefreshing = true;
            this.refreshTokenSubject.next(null);

            return this._auth.refreshToken().pipe(
                switchMap((token: any) => {
                    this.isRefreshing = false;
                    this.refreshTokenSubject.next(token.jwt);
                    return next.handle(this.addToken(request, token.jwt));
                }));
        } else {
            return this.refreshTokenSubject.pipe(
                filter(token => token != null),
                take(1),
                switchMap(jwt => {
                    return next.handle(this.addToken(request, jwt));
                }));
        }
    }
}
\end{lstlisting}
\clearpage

\phantomsection
\addcontentsline{toc}{subsection}{ПРИЛОЖЕНИЕ 4 Исходный код файла auth.service.ts}
\begin{center}
    ПРИЛОЖЕНИЕ 4\\
    %\vspace*{-1cm}
    (обязательное)


    Исходный код файла auth.service.ts
\end{center}

\begin{lstlisting}
    import { Injectable } from '@angular/core';
    import { HttpClient } from "@angular/common/http";
    import { Observable, of } from 'rxjs';
    import { tap, map, catchError  } from 'rxjs/operators';
    import { Tokens } from './tokens';
    @Injectable({
        providedIn: 'root'
    })
    export class AuthService {
        constructor(private http: HttpClient) { }
        isLoggedIn() {
            console.log('$ AuthService isLoggedIn():');
            return !!this.getJwtToken();
        }
        readonly ROOT_URL: string = 'http://localhost:3002';
        readonly LOGIN_URL: string = this.ROOT_URL + '/api/login'; 
        readonly REFRESH_URL: string = this.ROOT_URL + '/api/refresh';
        readonly LOGOUT_URL: string = this.ROOT_URL + '/api/logout';
        login(user: UserData): Observable<boolean> {
            console.log('$ AuthService login():');
            return this.http.post<any>(this.LOGIN_URL, user)
                .pipe(
                    tap((tokens) => {
                        console.log('tokens', tokens)
                        this.doLoginUser(user.username, tokens)
                    }),
                    map(() => true),
                    catchError((error) => {
                        console.log(error);
                        return of(false);
                    })
                );
        }
        private loggedUser: string | null;
        private doLoginUser(username: string, tokens: Tokens) {
            console.log('$ AuthService doLoginUser():', username, tokens, typeof tokens);
            this.loggedUser = username;
            this.storeTokens(tokens);
        }
        private readonly JWT_TOKEN = 'JWT_TOKEN';
        private readonly REFRESH_TOKEN = 'REFRESH_TOKEN';
        private storeTokens(tokens: Tokens) {
            console.log('$ AuthService storeTokens():');
            localStorage.setItem(this.JWT_TOKEN, tokens.jwt);
            localStorage.setItem(this.REFRESH_TOKEN, tokens.refreshToken);
        }
        refreshToken() {
            console.log('$ AuthService refreshToken():');
            return this.http.post<any>(this.REFRESH_URL, {
                'refreshToken': this.getRefreshToken()
            }).pipe(tap((tokens: Tokens) => {
                this.storeJwtToken(tokens.jwt);
            }));
        }
        private getRefreshToken() {
            console.log('$ AuthService getRefreshToken():');
            return localStorage.getItem(this.REFRESH_TOKEN);
        }
        private storeJwtToken(jwt: string) {
            console.log('$ AuthService storeJwtToken():');
            localStorage.setItem(this.JWT_TOKEN, jwt);
        }
        getJwtToken() {
            console.log('$ AuthService getJwtToken():');
            return localStorage.getItem(this.JWT_TOKEN);
        }
        private removeTokens() {
            localStorage.removeItem(this.JWT_TOKEN);
            localStorage.removeItem(this.REFRESH_TOKEN);
        }
        private doLogoutUser() {
            this.loggedUser = null;
            this.removeTokens();
        }
        logout() {
            return this.http.post<any>(this.LOGOUT_URL, {
                'refreshToken': this.getRefreshToken()
            }).pipe(
                tap(() => this.doLogoutUser()),
                map(() => true),
                catchError(error => {
                    alert(error.error);
                    return of(false);
                }));
        }
    }
    interface UserData {
        username: string
        password: string;
    }    
\end{lstlisting}
\clearpage