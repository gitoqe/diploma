\begingroup
\renewcommand{\section}[2]{\anonsection{Список использованных источников}}
\begin{thebibliography}{00}

% =-=-=-=-=-=-=-=-=-=-=- ././ =-=-=-=-=-=-=-=-=-=-=-
% =-=-=-=-=-=-=-=-=-=-=- MAIN =-=-=-=-=-=-=-=-=-=-=-
% =-=-=-=-=-=-=-=-=-=-=- ././ =-=-=-=-=-=-=-=-=-=-=-

    \bibitem{inlinegroup-c}
        Автоматизация Учебного центра | Группа компаний «ИНЛАЙН ГРУП»
        --
        URL:
        \href{https://inlinegroup-c.ru/services/upravlenie-uchebnym-centrom/}{https://inlinegroup-c.ru/services/upravlenie-uchebnym-centrom/}
        (дата обращения: 01.05.2022).
        --
        Текст : электронный.

    \bibitem{1c-training-center}
        1С:Управление учебным центром - Возможности продукта
        --
        URL:
        \href{https://solutions.1c.ru/catalog/training-center/features}{https://solutions.1c.ru/catalog/training-center/features}
        (дата обращения: 01.05.2022).
        --
        Текст : электронный.

	\bibitem{denisov}
		Какие бывают сайты? Виды сайтов по контенту, дизайну и функциональности
        --
        URL:
        \href{https://sdvv.ru/articles/testovyy-razdel/kakie-byvayut-sayty-vidy-saytov-po-kontentu-dizaynu-i-funktsionalnosti/}{https://sdvv.ru/articles/testovyy-razdel/kakie-byvayut-sayty-vidy-saytov-po-kontentu-dizaynu-i-funktsionalnosti/}
        (дата обращения: 01.05.2022).
        --
        Текст : электронный.
    

    \bibitem{ualiev-jebegenov-issledovanie}
        Жебегенов, Ә.М. Исследование средств разработки личного кабинета обучающегося на базе ЖГУ Им. И.Жансугурова.
        /
        Н.С. Уалиев, Ә.М. Жебегенов
        //
        Международный научный журнал «Вестник науки»
        --
        2019.
        --
        №5
        --
        С. 376--380.
        % https://cyberleninka.ru/article/n/issledovanie-sredstv-razrabotki-lichnogo-kabineta-obuchayuschegosya-na-baze-zhgu-im-i-zhansugurova

    \bibitem{ualiev-jebegenov-issledovanie-1}
        Уалиев, Н.С. Обзор технологии фреймворк для веб-разработки.
        /
        Н.С. Уалиев, Л.А. Смагулова, А.Т. Турсынова
        //
        Инновационные технологии в науке и образовании: сборник статей VIII Международной научно-практической конференции.
        --
        Пенза: МЦНС «Наука и Просвещение».
        --
        2018.
        --
        С. 54--56.
        % https://naukaip.ru/wp-content/uploads/2018/04/%D0%9C%D0%9A-317-%D0%A1%D0%B1%D0%BE%D1%80%D0%BD%D0%B8%D0%BA.pdf

    \bibitem{kiyamov-cms}
        Киямов, Р. В. Системы управления контентом - сейчас и в дальнейшем.
        /
        Р.В. Киямов, Е.А. Хмелев, И.Ф. Юнусов
        //
        Журнал "Научный журнал"
        --
        2016.
        --
        С. 25--26.
        % https://cyberleninka.ru/article/n/sistemy-upravleniya-kontentom-seychas-i-v-dalneyshem

    \bibitem{kiyamov-cms-1}
        Web технологии и интернет. Что такое CMS?
        --
        URL:
        http://azericms.com/chto-takoe-cms/
        (дата обращения: 12.09.2016).
        --
        Текст : электронный.


    \bibitem{ivanisheva-cms}
        Иванищева, А.А. Анализ бесплатных популярных CMS платформ по созданию web-сайта.
        /
        А.А. Иванищева, Х.И. Комилов, М.Д. Гехаев
        //
        Журнал "Инновационная наука"
        --
        2019.
        --
        №3
        --
        С. 37--39.
        % https://cyberleninka.ru/article/n/analiz-besplatnyh-populyarnyh-cms-platform-po-sozdaniyu-web-sayta

    \bibitem{sergacheva-framework}
        Сергачева, М.А. Анализ фреймворков для разработки современных веб-приложений.
        /
        М.А. Сергачева, К.А. Михалевская
        //
        Журнал "Кронос: естественные и технические науки"
        --
        2020.
        --
        С. 35--39.
        % https://cyberleninka.ru/article/n/analiz-freymvorkov-dlya-razrabotki-sovremennyh-veb-prilozheniy

    \bibitem{sergacheva-framework-1}
        Как работают веб-приложения / Хабр
        --
        URL:
        \href{https://habr.com/ru/post/450282/}{https://habr.com/ru/post/450282/}
        (дата обращения: 01.05.2022).
        --
        Текст : электронный.

    \bibitem{sergacheva-framework-2}
        Что лучше выбрать в 2020 году — React или Vue? / Хабр
        --
        URL:
        \href{https://habr.com/ru/company/ruvds/blog/470413/}{https://habr.com/ru/company/ruvds/blog/470413/}
        (дата обращения: 01.05.2022).
        --
        Текст : электронный.

    \bibitem{sergacheva-framework-3}
        Самые популярные фреймворки и языки программирования среди стартапов-единорогов | Techrocks.
        --
        URL:
        \href{https://techrocks.ru/2019/07/20/most-popular-frameworks-and-languages-in-startups}{https://techrocks.ru/2019/07/20/most-popular-frameworks-and-languages-in-startups}
        (дата обращения: 01.05.2022).
        --
        Текст : электронный.


    \bibitem{sergacheva-framework-4}
        React или Vue или Angular. Что Выбрать?
        --
        URL:
        \href{https://www.youtube.com/watch?v=Nm8GpLCAgwk}{https://www.youtube.com/watch?v=Nm8GpLCAgwk}
        (дата обращения: 01.05.2022).
        --
        Текст : электронный.

    \bibitem{davidovsky-vibor}
        Давыдовский, М.А. Выбор веб-стека для реализации цифровой среды предоставления транспортных услуг
        /
        М.А. Давыдовский
        //
        Журнал "Образовательные ресурсы и технологии"
        --
        2019.
        --
        №4
        --
        С. 34--41.
        % https://cyberleninka.ru/article/n/vybor-veb-steka-dlya-realizatsii-tsifrovoy-sredy-predostavleniya-transportnyh-uslug

    \bibitem{davidovsky-vibor-1}
        Никсон, Р. Создаем динамические веб-сайты с помощью PHP, MySQL, JavaScript, CSS и HTML5
        /
        Р. Никсон
        --
        СПб.: Питер,
        2019.
        --
        816 с.

    \bibitem{davidovsky-vibor-php}
        PHP: Hypertext Preprocessor
        --
        URL:
        \href{https://www.php.net}{https://www.php.net}
        (дата обращения: 01.05.2022).
        --
        Текст: электронный.

    \bibitem{davidovsky-vibor-NODE}
        Node.js
        --
        URL:
        \href{https://www.nodejs.org}{https://www.nodejs.org}
        (дата обращения: 01.05.2022).
        --
        Текст: электронный.


    \bibitem{davidovsky-vibor-mean}
        MEAN.JS - Full-Stack JavaScript Using MongoDB, Express, AngularJS, and Node.js
        --
        URL:
        \href{http://www.meanjs.org}{http://www.meanjs.org}
        (дата обращения: 01.05.2022).
        --
        Текст: электронный.

    \bibitem{slinkin-sovremennie}
        Слинкин, Д.А. Современные подходы к модернизации веб-ресурсов образовательной организации
        /
        Д.А. Слинкин
        //
        Журнал "Вестник Шадринского государственного педагогического университета"
        --
        2019.
        --
        С. 103--108.

    \bibitem{iTrack}
        Определить CMS (движок) сайта - Компания iTrack
        --
        URL:
        \href{https://itrack.ru/whatcms/check}{https://itrack.ru/whatcms/check}
        (дата обращения: 01.05.2022).
        --
        Текст: электронный.

    \bibitem{aelita}
        А-Элита | Курсы программирования и робототехники в Вологде
        --
        URL:
        \href{http://aelita35.ru/}{http://aelita35.ru/}
        (дата обращения: 01.05.2022).
        --
        Текст: электронный.

    \bibitem{energy}
        Компьютерная школа для детей и взрослых | Учебный центр "Энергетик"
        --
        URL:
        \href{https://uc-energetik.ru/activities-hub/compschool}{https://uc-energetik.ru/activities-hub/compschool}
        (дата обращения: 01.05.2022).
        --
        Текст: электронный.

    \bibitem{bulma}
        Bulma
        --
        URL:
        \href{https://bulma.io/}{https://bulma.io/}
        (дата обращения: 26.05.2022).
        --
        Текст: электронный.

    \bibitem{eslint}
        ESLint
        --
        URL:
        \href{https://eslint.org/}{https://eslint.org/}
        (дата обращения: 01.05.2022).
        --
        Текст: электронный.

    \bibitem{prettier}
        Prettier
        --
        URL:
        \href{https://prettier.io/}{https://prettier.io/}
        (дата обращения: 01.05.2022).
        --
        Текст: электронный.
        


% =-=-=-=-=-=-=-=-=-=-=- ./././././ =-=-=-=-=-=-=-=-=-=-=-
% =-=-=-=-=-=-=-=-=-=-=- DEPRECATED =-=-=-=-=-=-=-=-=-=-=-
% =-=-=-=-=-=-=-=-=-=-=- ./././././ =-=-=-=-=-=-=-=-=-=-=-

\begin{comment}

    \bibitem{livejournal}
        Главное - ЖЖ
        [Электронный ресурс] //
        LiveJournal — сервис для ведения блога и развития сообщества
        --
        Режим доступа:
        \href{https://www.livejournal.com/}{https://www.livejournal.com/}
        
    \bibitem{kickstarter}
        Kickstarter
        [Электронный ресурс] //
        Kickstarter - краудфандинговая платформа.
        --
        Режим доступа:
        \href{https://www.kickstarter.com/}{https://www.kickstarter.com/}

    \bibitem{ozon}
        OZON — интернет-магазин. Миллионы товаров по выгодным ценам
        [Электронный ресурс] //
        OZON. Интернет магазин
        --
        Режим доступа:
        \href{https://www.ozon.ru/}{https://www.ozon.ru/}

    \bibitem{stepik}
        Stepik
        [Электронный ресурс] //
        Stepik - интернет платформа для создания и распространения обучающего контента
        --
        Режим доступа:
        \href{https://stepik.org}{https://stepik.org/}

    \bibitem{vk}
        ВКонтакте
        [Электронный ресурс] //
        ВКонтакте - социальная сеть
        --
        Режим доступа:
        \href{https://vk.com/}{https://vk.com/}

    \bibitem{wiki}
        Википедия
        [Электронный ресурс] //
        Википедия. Свободная энциклопедия
        --
        Режим доступа:
        \href{https://ru.wikipedia.org/}{https://ru.wikipedia.org/}

    \bibitem{wiki-CMS}
        CMS -- Википедия 
        [Электронный ресурс] //
        Википедия. Свободная энциклопедия
        --
        Режим доступа:
        \href{https://ru.wikipedia.org/wiki/CMS}{https://ru.wikipedia.org/wiki/CMS}

    \bibitem{cmsmagazine}
        Движки для сайтов, платные и бесплатные CMS системы, каталог систем управления сайтами
        [Электронный ресурс] //
        CMS Magazine - digital-журнал
        --
        Режим доступа:
        \href{https://cmsmagazine.ru/cms/}{https://cmsmagazine.ru/cms/}

    \bibitem{wiki-WordPress}
        WordPress -- Википедия 
        [Электронный ресурс] //
        Википедия. Свободная энциклопедия
        --
        Режим доступа:
        \href{https://ru.wikipedia.org/wiki/WordPress}{https://ru.wikipedia.org/wiki/WordPress}

    \bibitem{wiki-1cb}
        1С-Битрикс: Управление сайтом 
        [Электронный ресурс] //
        Википедия. Свободная энциклопедия
        --
        Режим доступа:
        \href{https://ru.wikipedia.org/wiki/1С-Битрикс}{https://ru.wikipedia.org/wiki/1С-Битрикс}

    \bibitem{wiki-joomla}
        Joomla!
        [Электронный ресурс] //
        Википедия. Свободная энциклопедия
        --
        Режим доступа:
        \href{https://ru.wikipedia.org/wiki/Joomla}{https://ru.wikipedia.org/wiki/Joomla}

    \bibitem{wiki-drupal}
        Drupal
        [Электронный ресурс] //
        Википедия. Свободная энциклопедия
        --
        Режим доступа:
        \href{https://ru.wikipedia.org/wiki/Drupal}{https://ru.wikipedia.org/wiki/Drupal}
        
    \bibitem{wiki-tilda}
        Tilda Publishing
        [Электронный ресурс] //
        Википедия. Свободная энциклопедия
        --
        Режим доступа:
        \href{https://ru.wikipedia.org/wiki/Tilda_Publishing}{$https://ru.wikipedia.org/wiki/Tilda_Publishing$}
        
    \bibitem{wiki-MongoDB}
        MongoDB -- Википедия 
        [Электронный ресурс] //
        Википедия. Свободная энциклопедия
        --
        Режим доступа:
        \href{https://ru.wikipedia.org/wiki/MongoDB}{https://ru.wikipedia.org/wiki/MongoDB}

    \bibitem{wiki-Express.js}
        Express.js -- Википедия 
        [Электронный ресурс] //
        Википедия. Свободная энциклопедия
        --
        Режим доступа:
        \href{https://ru.wikipedia.org/wiki/Express.js}{https://ru.wikipedia.org/wiki/Express.js}

    \bibitem{wiki-angular.js}
        Angular.js -- Википедия 
        [Электронный ресурс] //
        Википедия. Свободная энциклопедия
        --
        Режим доступа:
        \href{https://ru.wikipedia.org/wiki/Angular}{https://ru.wikipedia.org/wiki/Angular}
        

    \bibitem{wiki-nodejs}
        Node.js -- Википедия 
        [Электронный ресурс] //
        Википедия. Свободная энциклопедия
        --
        Режим доступа:
        \href{https://ru.wikipedia.org/wiki/Node.js}{https://ru.wikipedia.org/wiki/Node.js}

    \bibitem{wiki-react}
        React -- Википедия 
        [Электронный ресурс] //
        Википедия. Свободная энциклопедия
        --
        Режим доступа:
        \href{https://ru.wikipedia.org/wiki/React}{https://ru.wikipedia.org/wiki/React}


    \bibitem{wiki-rest}
        REST -- Википедия 
        [Электронный ресурс] //
        Википедия. Свободная энциклопедия
        --
        Режим доступа:
        \href{https://ru.wikipedia.org/wiki/REST}{https://ru.wikipedia.org/wiki/REST}
    
\end{comment}

% =-=-=-=-=-=-=-=-=-=-=- ./././././ =-=-=-=-=-=-=-=-=-=-=-
% =-=-=-=-=-=-=-=-=-=-=-=-= PDP =-=-=-=-=-=-=-=-=-=-=-=-=-
% =-=-=-=-=-=-=-=-=-=-=- ./././././ =-=-=-=-=-=-=-=-=-=-=-

\begin{comment}

    \bibitem{kiyamov-cms}
Киямов, Р. В. Системы управления контентом - сейчас и в дальнейшем.
/
Р.В. Киямов, Е.А. Хмелев, И.Ф. Юнусов
//
Журнал "Научный журнал"
--
2016.
--
С. 25--26.
% https://cyberleninka.ru/article/n/sistemy-upravleniya-kontentom-seychas-i-v-dalneyshem

\bibitem{ivanisheva-cms}
Иванищева, А.А. Анализ бесплатных популярных CMS платформ по созданию web-сайта.
/
А.А. Иванищева, Х.И. Комилов, М.Д. Гехаев
//
Журнал "Инновационная наука"
--
2019.
--
№3
--
С. 37--39.
% https://cyberleninka.ru/article/n/analiz-besplatnyh-populyarnyh-cms-platform-po-sozdaniyu-web-sayta

\bibitem{davidovsky-vibor}
Давыдовский, М.А. Выбор веб-стека для реализации цифровой среды предоставления транспортных услуг
/
М.А. Давыдовский
//
Журнал "Образовательные ресурсы и технологии"
--
2019.
--
№4
--
С. 34--41.
% https://cyberleninka.ru/article/n/vybor-veb-steka-dlya-realizatsii-tsifrovoy-sredy-predostavleniya-transportnyh-uslug

\end{comment}

\end{thebibliography}
\endgroup

\clearpage
